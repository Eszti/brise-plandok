\documentclass{article}
\usepackage[margin=3cm]{geometry}
\usepackage[utf8]{inputenc}
\usepackage[T1]{fontenc}
\usepackage[ngerman]{babel}
\usepackage[usenames]{color}
\usepackage{siunitx}
\usepackage{makeidx}
\usepackage{etoolbox}
% \usepackage{longtable}
\usepackage{fancyhdr}
\usepackage{hyperref}


%% Macros

% Format for attributes:
\newcommand{\mml}[1]{\texttt{#1}}

% Format for attribute entries:
% \merkmal{Name_of_Merkmal}{
%   \erlaeuterung{Explanation where present}
%   \wert{Type of value possibly with clarification}
%   \beispiel{Example where present}
%   \rechtsmaterie{Law text the Merkmal has been extracted from where
%      applicable}
%   \notiz{Notes and clarifications, e.g., "don't confuse with Merkmal
%   xyz"}
%   \quelle{Source, only in long version}
%   \frage{Potential questions, only in long version}
% }
% NOTE: Latex doesn't like "\_" in labels. So avoid it in Merkmal names!
\newcommand{\merkmal}[2]{\subsubsection{\mml{#1}}\label{merkmal:#1}\index{\category! #1}\begin{description}#2\end{description}\medskip}
\newcommand{\erlaeuterung}[1]{\item[Erläuterung]:\\ #1 }
\newcommand{\wert}[1]{\item[Wert]:\\ #1 }
\newcommand{\beispiel}[1]{\item[Beispiele]:\\ #1 }
\newcommand{\rechtsmaterie}[1]{\item[Rechtsmaterie]:  \\#1 }
\newcommand{\notiz}[1]{\item[Bemerkungen]:  \\#1 }
\newcommand{\quelle}[1]{\inVersion{long}{\item[Quelle]: \\#1 }}
\newcommand{\frage}[1]{\inVersion{long}{\item[{\color{red}Frage}]: \\#1 }}


% for referring to a different Merkmal in the text
% Use the name of the Merkmal, e.g., \merkmalref{AnteilBaumbepflanzung}
\newcommand{\merkmalref}[1]{\hyperref[merkmal:#1]{\mml{#1}}}

% for typesetting "[...]"
\newcommand{\etc}{{[\dots]}}

% for indexing the Merkmale according to their categories
\newcommand{\category}{none}


% Versionlength: set to "short" for short (standard), "long" for long (including sources),
% "all" for both
\newcommand{\versionlength}{short}

% prints the text only in the appropriate version
% requires etoolbox package
\newcommand{\inVersion}[2]{\ifdefstring{\versionlength}{#1}{#2}{
\ifdefstring{\versionlength}{all}{\noindent \textbf{In #1 Version: }
  #2 \textbf{End of  #1 Version}\\}{}
}}



\makeindex
\patchcmd{\theindex}{\thispagestyle{plain}}
  {\thispagestyle{plain}\phantomsection\label{index}}{}{}




\begin{document}
\title{BRISE-Vienna - Annotationsrichtlinien}
\date{Version vom \today}
\maketitle

% \renewcommand{\chaptermark}[1]{ \markboth{#1}{} }
% \renewcommand{\sectionmark}[1]{ \markright{#1}{} }
% \pagestyle{myheadings}
% \markright{\thesection \hfill}

\lfoot[{\hyperref[index]{Merkmalesindex} der
  Merkmale}]{\hyperref[index]{Merkmalsindex}}
\pagestyle{fancy}
% \renewcommand{\chaptermark}[1]{ \markboth{#1}{} }
% \renewcommand{\sectionmark}[1]{ \markright{#1}{} }




\section{Einführung}
\label{sec:einfuhrung}

Dieses Dokument beinhaltet die Annotationsrichtlinien für die
Annotation der den im Wiener Flächen\-wid\-mungs- und Bebauungsplan
ausgewiesenen Plangebieten angehängten textuellen Bestimmungen im
Rahmen des Projektes
\href{https://digitales.wien.gv.at/site/projekt/brisevienna/}{BRISE-Vienna}. Ziel
dieser Annotationen ist es, ein Modell für maschinelles Lernen so zu trainieren,
daß es die zu einer textuellen Bestimmung gehörigen Merkmale
automatisch extrahieren kann.

Für den Index der Merkmale siehe den
\hyperref[index]{Merkmalsindex}. Für ausgemusterte Merkmale und
sonstige große Änderungen siehe Anhang~\ref{sec:changes}.


\section{Die % Erläuterung der einzelnen
  Merkmale}
\label{sec:merkmalsliste}



Es folgt eine Auflistung der verfügbaren Merkmale, aufgegliedert
in die folgenden Kategorien:
\begin{itemize}
\item Ausgestaltung und Sonstiges (Abschn.~\ref{sec:kateg-ausg-und})
\item Dach (Abschn.~\ref{sec:kategorie:dach})
\item Einfriedungen (Abschn.~\ref{sec:kateg-einfr})
\item Fläche (Abschn.~\ref{sec:kategorie:flache})
\item Geschosse (Abschn.~\ref{sec:kategorie:geschosse})
\item Grossbauvorhaben, Hochhäuser, Einkaufszentren,
  Geschäftsstrassen (Abschn.~\ref{sec:kateg-grossb-hochh})
\item Höhe (Abschn.~\ref{sec:kategorie:hohe})
\item Lage, Gelände und Planzeichen (Abschn.~\ref{sec:kateg-lage-gelande})
\item Laubengänge, Durchfahrten, Arkaden (Abschn.~\ref{sec:kateg-laub-durchf})
\item Nutzung und Widmung (Abschn.~\ref{sec:kateg-nutz-und})
\item Stellplätze, Garagen, Parkgebäude (Abschn.~\ref{sec:kateg-stellpl-garag})
\item Strasse und Gehsteige (Abschn.~\ref{sec:kateg-strasse-und})
\item Volumen (Abschn.~\ref{sec:kategorie:volumen})
\item Vorbauten (Abschn.~\ref{sec:kategorie:vorbauten})
\end{itemize}

Die Beschreibung der einzelnen Merkmale folgt der folgenden Struktur:

\merkmal{Merkmalsname}{
\erlaeuterung{Eine kurze Erläuterung des Merkmales.}
\wert{Eine Beschreibung des zugehörigen Wertes und der
  Werteeinheit.}
\beispiel{Beispiele für Vorkommnisse des Merkmales in den
  Textuellen Bestimmungen.}
\rechtsmaterie{Die zugrundeliegende Rechtsmaterie für das Merkmal,
sofern vorhanden.}
\notiz{Etwaige sonstige Bemerkungen.}
\quelle{Die Quelle des Merkmals (nur in der langen Version der Richtlinien).}
\frage{Etwaige Fragen (nur in der langen Version der Richtlininen).}
}

%%%%%%%  Merkmale: Template  %%%%%%%
% \merkmal{}{
%   \erlaeuterung{}
%   \wert{}
%   \beispiel{}
%   \rechtsmaterie{}
%   \notiz{}
%   \quelle{}
%   \frage{}
% }
%%%%%%%  End of Merkmale template  %%%%%%%

\subsection{Kategorie: Ausgestaltung und Sonstiges}
\label{sec:kateg-ausg-und}

\renewcommand{\category}{Ausgestaltung und Sonstiges}

Enthält Merkmale bezüglich der Ausgestaltung von Gebäuden
sowie sonstige Merkmale.\smallskip



\merkmal{AnordnungGaertnerischeAusgestaltung}{
  \erlaeuterung{Wahr, wenn es eine Anordnung für gärtnerische Ausgestaltung unbebauter Flächen gibt.}
  \wert{Wahrheitswert}
  \beispiel{"`Die in den gärtnerisch auszugestaltenden Flächen (G) nach
    der Bauordnung für Wien zulässigen Nebengebäude dürfen, mit
    Ausnahme der mit BB3 bezeichneten Flächen, je Bauplatz eine
    bebaute Gesamtfläche von 30 $m^2$ nicht überschreiten."'\\
    (7792\_11)\smallskip\\
    "`Nicht bebaute, aber bebaubare Flächen im Bauland sind, soweit
    Anlagen zum Einstellen von Kraftfahrzeugen darauf nicht errichtet
    werden, gärtnerisch auszugestalten, wobei die Errichtung von
    betrieblich benötigten Rangier- und Zufahrtsflächen zulässig
    ist."'\\
    (7792\_12)
  }
  \rechtsmaterie{\href{https://www.ris.bka.gv.at/NormDokument.wxe?Abfrage=LrW&Gesetzesnummer=20000006&FassungVom=2020-11-05&Artikel=&Paragraf=5&Anlage=&Uebergangsrecht=}{WBO \S 5/4/p}: \\
(4) Über die Festsetzungen nach Abs. 2 und 3 hinaus können die
Bebauungspläne zusätzlich enthalten:\\
\etc\\
p)
die Anordnung der gärtnerischen Ausgestaltung unbebauter Grundflächen;
}
\notiz{Für spezifische Anordnungen wie die Schaffung von Erdkernen
  für Bäume siehe \merkmalref{VorkehrungBepflanzung}. % Oft auch
  % im Zusammenhang mit \merkmalref{FlaecheBebaubar} und \merkmalref{FlaecheBebaut}.
}
\quelle{Aus \href{run:./files/200526_BRISE_RIM-IDM_Auszug AI-BB.xlsx}{200626\_BRISE\_RIM-IDM\_Auszug\_AI-BB}}
% \frage{Wie genau ist die Erklärung gemeint?}
}


\merkmal{AnordnungGaertnerischeAusgestaltungProzentual}{
  \erlaeuterung{Spezifiziert den Mindestanteil einer Fläche
    welcher gärtnerisch auszugestalten ist.}
  \wert{Eine positive Zahl [Prozent]}
  \beispiel{"`Die mit P BB 4 bezeichneten und für das Abstellen von
    Kraftfahrzeugen bestimmten Grundflächen sind bis zu einem Ausmaß
    von mindestens 10 v. H. dieser durch Fluchtlinien bestimmten
    Flächen gärtnerisch auszugestalten."'\\
    (7020\_26)
  }
  \notiz{Siehe \merkmalref{AnordnungGaertnerischeAusgestaltung}
    für allgemeinere Anordnungen zur gärtnerischen Ausgestaltung.}
  \quelle{Aus Kleinworkshop 2021-01-22}
}


\merkmal{AnteilBaumbepflanzung}{
  \erlaeuterung{Spezifiziert den Anteil der Fläche, für den Vorkehrungen für die Pflanzung von Bäumen zu treffen sind.}
  \wert{Eine positive Zahl [\%]}
  \beispiel{"`Für 10 \% der mit BB4 bezeichneten Fläche sind technische Vorkehrungen für Baumpflanzungen (Erdkerne) zu treffen."'\\(7545\_13)}
  % \rechtsmaterie{}
  \quelle{Aus den Beispielen sowie
    \href{run:./files/Merkmale_neu_liste-WP4-2020-11-06.xlsx}{Merkmale\_neu\_liste
    WP4 (Version von 2020-11-06)}
    }
}

\merkmal{AnzahlGebaeudeMax}{
  \erlaeuterung{Spezifiziert die maximal Anzahl an Gebäuden welche
    errichtet werden dürfen.}
  \wert{Eine positive Zahl [Anzahl]}
  \beispiel{"`Innerhalb der mit BB 3 bezeichneten und als Bauland/ Gartensiedlungsgebiet gewidmeten Grundflächen darf nur ein Gebäude mit einer bebauten Fläche von maximal 80 $m^2$ errichtet werden."' \\(7412\_16)}
  \quelle{Annotations Kleinworkshop 2021-03-01}
}


\merkmal{AufbautenZulaessig}{
  \erlaeuterung{Wahr wenn Aufbauten zulässig sind,
    beispielsweise technische Aufbauten oder Aufbauten welche der
    Belichtung dienen.}
  \wert{Wahrheitswert}
  \beispiel{"`Für 10 \% der mit BB4 bezeichneten Fläche sind
    technische Vorkehrungen für Baumpflanzungen (Erdkerne) zu
    treffen. Technisch erforderliche Aufbauten sind
    zulässig."'\\(7545\_13)\smallskip\\
    "`Flachdächer bis zu einer Dachneigung von fünf
    Grad sind entsprechend dem Stand der technischen Wissenschaften zu
    begrünen. Technische bzw. der Belichtung dienende Aufbauten sind im
    erforderlichen Ausmaß zulässig"'\\(7181\_6)\smallskip\\
    "`Weitere technische Aufbauten sind nur auf dem mit BB14 bezeichneten und als Bauklasse VI festgesetzten Gebäudeteil zulässig."'\\(7990\_15)
  }
  % \rechtsmaterie{}
  \notiz{Vergleiche auch
    \merkmalref{TechnischeAufbautenHoeheMax} für
    die Beschränkung der Höhe technischer Aufbauten.}
  \quelle{Sammelmerkmal für \merkmalref{TechnischeAufbautenZulaessig} und
    \merkmalref{TechnischeUndBelichtungsaufbautenZulaessig}.
    Aus den Beispielen sowie
    \href{run:./files/Merkmale_neu_liste-WP4-2020-11-06.xlsx}{Merkmale\_neu\_liste
    WP4 (Version von 2020-11-06)} (für
  \merkmalref{TechnischeAufbautenZulaessig} sowie
  \merkmalref{TechnischeUndBelichtungsaufbautenZulaessig}).
    }
}


\merkmal{AusnahmeGaertnerischAuszugestaltende}{
  \erlaeuterung{Spezifiziert Ausnahmen zur Anordnung der
    gärtnerischen Ausgestaltung.}
  \wert{Text}
  \beispiel{"`Alle bebaubaren, jedoch unbebaut bleibenden
    Grundflächen sind, mit Ausnahme von unbedingt erforderlichen
    Park-, Rangier- und Manipulationsflächen, gärtnerisch
    auszugestalten."' % "mit Ausnahme unbedingt erforderlicher Park-, Rangier-
    % oder Manipulationsflächen"
    \\(7545\_7)}
  % \rechtsmaterie{}
  \quelle{Aus den Beispielen sowie
    \href{run:./files/Merkmale_neu_liste-WP4-2020-11-06.xlsx}{Merkmale\_neu\_liste
    WP4 (Version von 2020-11-06)}
    }
}

\merkmal{BauweiseID}{
  \erlaeuterung{Spezifiziert die Bauweise (Optionen: "`offen"' (o),
    "`gekuppelt"' (gk), "`offen oder gekuppelt"' (ogk), "`gruppenbauweise"' (gr), "`geschlossen"' (g)).}
  \wert{Text}
  \beispiel{"`Die Dächer der auf den mit der Festsetzung WGV I 5
    m g bzw. GBGV I 5 m g bezeichneten Grundflächen zur Errichtung
    gelangenden Gebäude sind entsprechend dem Stand der Technik
    als begrünte Flachdächer auszubilden \etc.% , sofern es sich nicht um Glasdächer handelt.
    "'\\(7408\_11)}
  % \rechtsmaterie{}
  \quelle{Aus den Beispielen, \href{run:./files/200929_BRISE_IDM-REM-LOI.xlsx}{200929\_BRISE\_IDM REM LOI} sowie
    \href{run:./files/Merkmale_neu_liste-WP4-2020-11-06.xlsx}{Merkmale\_neu\_liste
    WP4 (Version von 2020-11-06)}
    }

}

% \merkmal{BBallgemein}{
%   % \erlaeuterung{Allgemein auf das gesamte Plangebiet (vs spezifisch
%   % etc) wahrscheinlich nicht fuer uns relevant (beschreibt, ob output
%   % aus AI ins Referenzmodell kommt er klaert es nochmal ab. ->
%   % rauswerfen }
%   \wert{Text}
%   % \beispiel{}
%   % \rechtsmaterie{}
%   \notiz{aus WienBV\_Bebauungsbestimmungen}
%   \quelle{Aus \href{run:./files/200929_BRISE_IDM-REM-LOI.xlsx}{200929\_BRISE\_IDM REM LOI}
%   }
%   \frage{Erklärung? RAUSWERFEN!}
% }

% \merkmal{BBspezifisch}{
%   % \erlaeuterung{}
%   \wert{Text}
%   % \beispiel{}
%   \rechtsmaterie{\href{https://www.ris.bka.gv.at/NormDokument.wxe?Abfrage=LrW&Gesetzesnummer=20000006&FassungVom=2020-11-05&Artikel=&Paragraf=7d&Anlage=&Uebergangsrecht=}{WBO
%       \S 7d};
%     \href{https://www.ris.bka.gv.at/NormDokument.wxe?Abfrage=LrW&Gesetzesnummer=20000006&FassungVom=2020-11-05&Artikel=&Paragraf=7e&Anlage=&Uebergangsrecht=}{WBO
%       \S 7e};
%     \href{https://www.ris.bka.gv.at/NormDokument.wxe?Abfrage=LrW&Gesetzesnummer=20000006&FassungVom=2020-11-05&Artikel=&Paragraf=7f&Anlage=&Uebergangsrecht=}{WBO
%       \S 7f}}
%   \notiz{aus WienBV\_Bebauungsbestimmungen}
%   \quelle{Aus \href{run:./files/200929_BRISE_IDM-REM-LOI.xlsx}{200929\_BRISE\_IDM REM LOI}
%   }
%   \frage{Erklärung?  RAUSWERFEN!}
% }

% \merkmal{BBTBvorhanden}{
%   % \erlaeuterung{}
%   \wert{Wahrheitswert}
%   % \beispiel{}
%   % \rechtsmaterie{}
%   \notiz{aus WienBV\_Bebauungsbestimmungen}
%   \quelle{Aus \href{run:./files/200929_BRISE_IDM-REM-LOI.xlsx}{200929\_BRISE\_IDM REM LOI}
%   }
%   \frage{Erklärung?  RAUSWERFEN!}
% }


\merkmal{BegruenungFront}{
  \erlaeuterung{Beschreibung in welcher Art die Begrünung der Front zu erfolgen hat.}
  \wert{Text}
  % \beispiel{}
  \rechtsmaterie{\href{https://www.ris.bka.gv.at/NormDokument.wxe?Abfrage=LrW&Gesetzesnummer=20000006&FassungVom=2020-11-05&Artikel=&Paragraf=5&Anlage=&Uebergangsrecht=}{WBO \S 5/4/k}: \\
(4) Über die Festsetzungen nach Abs. 2 und 3 hinaus können die
Bebauungspläne zusätzlich enthalten:\\
\etc\\
k)
Bestimmungen über die Ausbildung der Fronten und Dächer der Gebäude, insbesondere über die Begrünung der Straßenfronten und der Dächer, sowie über die Dachneigungen, die auch mit mehr als 45 Grad, im Gartensiedlungsgebiet auch mit mehr als 25 Grad festgesetzt werden können;
}
\quelle{Aus \href{run:./files/200526_BRISE_RIM-IDM_Auszug
    AI-BB.xlsx}{200626\_BRISE\_RIM-IDM\_Auszug\_AI-BB}
}
}

% \merkmal{Eckbauplatz}{
%   % Eckbauplatz geometrisch erkennen -> nur damit das Programm weiss,
%   % ob es ein Eckbauplatz ist -> rauswerfen
%   % \erlaeuterung{}
%   \wert{Wahrheitswert}
%   % \beispiel{}
%   \rechtsmaterie{\href{https://www.ris.bka.gv.at/NormDokument.wxe?Abfrage=LrW&Gesetzesnummer=20000006&FassungVom=2020-11-05&Artikel=&Paragraf=75&Anlage=&Uebergangsrecht=}{WBO
%       \S 75/5}:\\
%   (5) Ergibt sich bei Anwendung der Bestimmung des Abs. 4 für
%   Eckbauplätze eine verschiedene Höhe der Hauptfronten, so ist die
%   größere Höhe auf eine Länge von höchstens 15 m auch für die andere
%   Hauptfront zulässig.
% }
%   \notiz{aus WienBV\_Bebauungsbestimmungen}
%   \quelle{Aus \href{run:./files/200929_BRISE_IDM-REM-LOI.xlsx}{200929\_BRISE\_IDM REM LOI}
%   }
%   \frage{Erklärung?  RAUSWERFEN!}
% }

\merkmal{Einbautrasse}{
  \erlaeuterung{Bestimmung zu Einbautrassen in Durchfahrten, Durchgängen oder Arkaden.}
  \wert{Text} % tendenziell text
  % \beispiel{}
  
\rechtsmaterie{\href{https://www.ris.bka.gv.at/NormDokument.wxe?Abfrage=LrW&Gesetzesnummer=20000006&FassungVom=2020-11-05&Artikel=&Paragraf=5&Anlage=&Uebergangsrecht=}{WBO \S 5/4/f}: \\
(4) Über die Festsetzungen nach Abs. 2 und 3 hinaus können die
Bebauungspläne zusätzlich enthalten:\\
\etc\\
f)
die Anordnung von Laubengängen, Durchfahrten, Durchgängen oder Arkaden;
}
\quelle{Aus \href{run:./files/200526_BRISE_RIM-IDM_Auszug
    AI-BB.xlsx}{200626\_BRISE\_RIM-IDM\_Auszug\_AI-BB}
}
% \frage{Wert des Merkmals ist ein Text? TENDENZIELL TEXT -> geklaert}
}

\merkmal{EinleitungNiederschlagswaesser}{
  \erlaeuterung{Bestimmungen über die Einleitung von
    Niederschlagswässern}
  \wert{Text}
  \beispiel{"`Die Einleitungsmenge von Niederschlagswässern in den
    Kanal darf im Neubaufall 0,012 l/s pro $\mathrm{m}^2$ der Fläche des
    jeweiligen Bauplatzes, Bauloses bzw. Trennstückes nicht
    überschreiten."'\\(8159\_14)\smallskip\\
  "`Im gesamten Antragsgebiet ist die Einleitung von
  Niederschlagswässern in den Kanal nicht zulässig."'\\(7736\_14)}
  \quelle{Aus den Beispielen und Übungen 2021-03-19}
}

\merkmal{ErrichtungGebaeude}{
  \erlaeuterung{Spezifizert, ob die Errichtung von Gebäuden
    erlaubt ist.}
  \wert{Wahrheitswert [wahr wenn erlaubt, falsch wenn verboten]}
  \beispiel{"`Auf den mit L BB5 bezeichneten Grundflächen ist die
    Errichtung von Gebäuden untersagt."'\\ (7181\_13)\smallskip\\
    "`Die Errichtung von Nebengebäuden ist unzulässig."'\\(6963\_24)\smallskip\\
    "`Der durch Baulinien bzw. Baufluchtlinien begrenzte Bereich darf unmittelbar bebaut werden."'\\(7857\_32)
  }
  % \rechtsmaterie{}
  \notiz{Für eine genauere Spezifizierung des Gebäudetyps
    (z.B. "`Nebengebäude"'), siehe \merkmalref{GebaeudeBautyp}.}
  \quelle{Aus den Beispielen sowie
    \href{run:./files/Merkmale_neu_liste-WP4-2020-11-06.xlsx}{Merkmale\_neu\_liste
    WP4 (Version von 2020-11-06)}
    }
    \frage{Wihann: checkt, ob es das gleiche wie verbotbebauung
      ist. => sagt es mir am Montag.}
}

\merkmal{GebaeudeBautyp}{
  \erlaeuterung{Spezifiziert den Bautyp eines Gebäudes,
    beispielsweise "`Glashaus"', "`Flügelbau"',
    "`Nebengebäude"', "`Hauptgebäude"', "`Flugdach"', etc}
  \wert{Text}
  \beispiel{"`Auf den als ländliches Gebiet (L) gewidmeten
    Flächen ist die Errichtung von Glashäusern
    unzulässig."'\\(7181\_8)\smallskip\\
   "`Die Errichtung von Nebengebäuden ist unzulässig."'\\(6963\_24)\smallskip\\
   "`Weiters darf das Ausmaß der durch Hauptgebäude bebauten Fläche
   20\% der Bauplatzfläche nicht überschreiten."'\\(7160\_8)\smallskip\\
   "`Flugdächer sind von diesen Beschränkungen ausgenommen."'\\(7813\_21)}
  % \rechtsmaterie{}
  \quelle{Aus den Beispielen sowie
    \href{run:./files/Merkmale_neu_liste-WP4-2020-11-06.xlsx}{Merkmale\_neu\_liste
    WP4 (Version von 2020-11-06)}
    }
}

\merkmal{Kleinhaeuser}{
  \erlaeuterung{Festlegung betrifft Kleinhäuser gem. WBO §5/4/u =
    Wahr. Also, "`wahr"', wenn die textliche Bestimmung sich auf
    Kleinhäuser bezieht.}
  % Wahr, wenn in textlichen Bestimmungen "Kleinhaeuser" vorkommt
  \wert{Wahrheitswert}
  % \beispiel{}
  \rechtsmaterie{\href{https://www.ris.bka.gv.at/NormDokument.wxe?Abfrage=LrW&Gesetzesnummer=20000006&FassungVom=2020-11-05&Artikel=&Paragraf=5&Anlage=&Uebergangsrecht=}{WBO \S 5/4/u}: \\
(4) Über die Festsetzungen nach Abs. 2 und 3 hinaus können die
Bebauungspläne zusätzlich enthalten:\\
\etc\\
u)
Gebiete, die der Errichtung von Wohngebäuden mit einer Gebäudehöhe von höchstens 7,50 m, die nicht mehr als zwei Wohnungen enthalten und bei denen für Betriebs- oder Geschäftszwecke höchstens ein Geschoß in Anspruch genommen wird (Kleinhäuser) und Reihenhäusern vorbehalten bleiben;
}
\notiz{Für die Beschränkung der maximalen Höhe von
  Kleinhäusern, siehe Merkmal \merkmalref{MaxHoeheWohngebaeude}.}
\quelle{Aus \href{run:./files/200526_BRISE_RIM-IDM_Auszug
    AI-BB.xlsx}{200626\_BRISE\_RIM-IDM\_Auszug\_AI-BB}
}
% \frage{Genauere Erläuterung? WAHR, wenn in textlichen Bestimmungen
%   "Kleinhaeuser" vorkommt. }
}

\merkmal{Massengliederung}{
  \erlaeuterung{Beschreibung der Massengliederung an der Gebäudefront.}
  \wert{Text [Verteilung der Massengliederung]}
  \beispiel{"`Soweit die zulässige Gebäudehöhe nach § 81 Abs. 2 der BO für Wien zu ermitteln ist, wird für die Gliederung der Baumassen bestimmt, dass der obere Abschluss der Gebäudefronten an keiner Stelle höher als das um 1,5 m vermehrte Ausmaß der zulässigen Gebäudehöhe über dem anschließenden Gelände liegen darf."'\\(7531\_7)}
  \rechtsmaterie{\href{https://www.ris.bka.gv.at/NormDokument.wxe?Abfrage=LrW&Gesetzesnummer=20000006&FassungVom=2020-11-05&Artikel=&Paragraf=5&Anlage=&Uebergangsrecht=}{WBO \S 5/4/i}: \\
(4) Über die Festsetzungen nach Abs. 2 und 3 hinaus können die
Bebauungspläne zusätzlich enthalten:\\
\etc\\
i)
die Massengliederung, die Anordnungen oder das Verbot der Staffelung der Baumassen und die Beschränkung oder das Verbot der Herstellung von Vorbauten;
}
\quelle{Aus \href{run:./files/200526_BRISE_RIM-IDM_Auszug
    AI-BB.xlsx}{200626\_BRISE\_RIM-IDM\_Auszug\_AI-BB}
}
}

% \merkmal{ObjektPosition}{
%   % \erlaeuterung{}
%   \wert{Text}
%   % \beispiel{}
%   \rechtsmaterie{\href{https://www.ris.bka.gv.at/NormDokument.wxe?Abfrage=LrW&Gesetzesnummer=20000006&FassungVom=2020-11-05&Artikel=&Paragraf=70a&Anlage=&Uebergangsrecht=}{WBO \S 70/a/1/ff}}
%   \notiz{aus WienBV\_Bebauungsbestimmungen}
%   \quelle{Aus \href{run:./files/200929_BRISE_IDM-REM-LOI.xlsx}{200929\_BRISE\_IDM REM LOI}
%   }
%   \frage{Erklärung?  RAUSWERFEN!}
% }

\merkmal{TechnischeAufbautenHoeheMax}{
  \erlaeuterung{Spezifiziert die maximale Höhe, bis zu der
    technische Aufbauten zulässig sind.}
  \wert{Eine positive Zahl [m]}
  \beispiel{"`Technische Aufbauten sind bis zu einer Höhe von 6,5m über
    der tatsächlich erreichten Gebäudehöhe
    zulässig."'\\(7545\_22)\smallskip\\
    "`Für die mit BB2 gekennzeichneten Grundflächen wird bestimmt: Über die zulässige Gebäudehöhe hinaus sind technische Aufbauten, wie ein Glockenturm bis zu einer insgesamt maximalen Höhe von 25 m zulässig."'\\(8286\_16)
  }
  % \rechtsmaterie{}
  \quelle{Aus den Beispielen sowie
    \href{run:./files/Merkmale_neu_liste-WP4-2020-11-06.xlsx}{Merkmale\_neu\_liste
    WP4 (Version von 2020-11-06)}
    }
}

% \merkmal{TechnischeAufbautenZulaessig}{
%   \erlaeuterung{Wahr wenn technisch erforderliche Aufbauten zulässig sind.}
%   \wert{Wahrheitswert}
%   \beispiel{"`Für 10 \% der mit BB4 bezeichneten Fläche sind
%     technische Vorkehrungen für Baumpflanzungen (Erdkerne) zu
%     treffen. Technisch erforderliche Aufbauten sind zulässig."'\\(7545\_13)}
%   % \rechtsmaterie{}
%   \notiz{Vergleiche auch
%     \merkmalref{TechnischeUndBelichtungsAufbautenZulaessig} für
%     technische und
%     Belichtungsaufbauten auf Dächern.}
%   \quelle{Aus den Beispielen sowie
%     \href{run:./files/Merkmale_neu_liste-WP4-2020-11-06.xlsx}{Merkmale\_neu\_liste
%     WP4 (Version von 2020-11-06)}
%     }
% }

\merkmal{UnterbrechungGeschlosseneBauweise}{
  \erlaeuterung{Für den Bauplatz bzw.\ die Fläche ist das Unterbrechen der geschlossenen Bauweise zulässig.}
  \wert{Wahrheitswert}
  \beispiel{"`Auf den mit BB1 bezeichneten Flächen ist die Unterbrechung der geschlossenen Bauweise zulässig."'\\(7160\_7)}
  \rechtsmaterie{\href{https://www.ris.bka.gv.at/NormDokument.wxe?Abfrage=LrW&Gesetzesnummer=20000006&FassungVom=2020-11-05&Artikel=&Paragraf=5&Anlage=&Uebergangsrecht=}{WBO \S 5/4/j}: \\
(4) Über die Festsetzungen nach Abs. 2 und 3 hinaus können die
Bebauungspläne zusätzlich enthalten:\\
\etc\\
j)
die Berechtigung zur Unterbrechung der geschlossenen Bauweise;
}
\quelle{Aus \href{run:./files/200526_BRISE_RIM-IDM_Auszug
    AI-BB.xlsx}{200626\_BRISE\_RIM-IDM\_Auszug\_AI-BB}
}
}

\merkmal{VerbotFensterZuOeffentlichenVerkehrsflaechen}{
  \erlaeuterung{Verbot von Fenstern von Aufenthaltsräumen zu öffentlichen Verkehrsflächen.}
  \wert{Text}
  \beispiel{"`Entlang der mit BB2 bezeichneten Baulinien dürfen im Erdgeschoß keine Fenster von Aufenthaltsräumen von Wohnungen zu den Verkehrsflächen hin orientiert werden."'\\(8036\_19)}
  \rechtsmaterie{\href{https://www.ris.bka.gv.at/NormDokument.wxe?Abfrage=LrW&Gesetzesnummer=20000006&FassungVom=2020-11-05&Artikel=&Paragraf=5&Anlage=&Uebergangsrecht=}{WBO \S 5/4/w}: \\
(4) Über die Festsetzungen nach Abs. 2 und 3 hinaus können die
Bebauungspläne zusätzlich enthalten:\\
\etc\\
w)
die Unzulässigkeit der Errichtung von Büro- und Geschäftsgebäuden, die Beschränkung des Rechtes, Fenster von Aufenthaltsräumen von Wohnungen zu öffentlichen Verkehrsflächen herzustellen, sowie in Wohnzonen die Verpflichtung, nicht weniger als 80 vH der Summe der Nutzflächen der Hauptgeschosse eines Gebäudes, jedoch unter Ausschluss des Erdgeschosses oder jener höchstens zulässigen Zahl von Geschossen, in denen die Nutzung für ein Einkaufszentrum zulässig ist, Wohnzwecken vorzubehalten;
}
\notiz{Vergleiche auch \merkmalref{VerbotAufenthaltsraum} für das
  Verbot von Aufenthaltsräumen generell. Für die
  Spezifizierung des Stockwerkes (z.B. "`im Erdgeschoss"') siehe \merkmalref{Stockwerk}.}
\quelle{Für das ursprüngliche
  \mml{UnzulaessigkeitFensterZuOeffentlichenVerkehrsflaechen} aus \href{run:./files/200526_BRISE_RIM-IDM_Auszug
    AI-BB.xlsx}{200626\_BRISE\_RIM-IDM\_Auszug\_AI-BB}
}
}


\merkmal{VerbotStaffelung}{
  \erlaeuterung{Wahr, wenn die Staffelung der Baumasse untersagt ist.}
  \wert{Wahrheitswert}
  % \rechtsmaterie{}
  \beispiel{Die Errichtung von Staffelgeschoßen and den zu den
    Baulinien orientierten Schauseiten der Gebäude ist untersagt.\\(7101\_13)}
  \quelle{Aus Kleinworkshop \# 1}
}


\merkmal{VerbotUnterirdischeBauwerkeUeberBaufluchtlinie}{
  \erlaeuterung{Unzulässigkeit mit unterirdischen Bauteilen über die Baufluchtlinie vorzuragen.}
  \wert{Wahrheitswert [wahr falls unzulässig]}
  % \beispiel{}
  \rechtsmaterie{\href{https://www.ris.bka.gv.at/NormDokument.wxe?Abfrage=LrW&Gesetzesnummer=20000006&FassungVom=2020-11-05&Artikel=&Paragraf=5&Anlage=&Uebergangsrecht=}{WBO \S 5/4/q}: \\
(4) Über die Festsetzungen nach Abs. 2 und 3 hinaus können die
Bebauungspläne zusätzlich enthalten:\\
\etc\\
q)
die Unzulässigkeit, mit unterirdischen Bauwerken über Baufluchtlinien vorzuragen;
}
\notiz{Siehe auch \merkmalref{VonBebauungFreizuhalten} für das
  generelle Verbot von unterirdischen Bauwerken.}
\quelle{Für das ursprüngliche Merkmal
  \mml{UnzulaessigkeitUnterirdischeBauwerke} aus \href{run:./files/200526_BRISE_RIM-IDM_Auszug
    AI-BB.xlsx}{200626\_BRISE\_RIM-IDM\_Auszug\_AI-BB}.
}
}

\merkmal{VonBebauungFreizuhalten}{
  \erlaeuterung{Die Fläche ist von gewisser Bebauung
    freizuhalten. Von welcher Bebauung ist im Wert angegeben.}
  \wert{Text [Art der Bebauung von welcher die Fläche freizuhalten
  ist]}
  \beispiel{"`Die mit Esp BB5 bezeichneten Flächen sind von jeder
    Bebauung freizuhalten."'\\(8063\_22)\smallskip\\
  "`Auf der mit BB16 bezeichneten Grundfläche ist die Errichtung von
  oberirdischen Bauten untersagt."'\\(7813\_30)\smallskip\\
"`Auf den mit G/BB4 bezeichneten Grundflächen dürfen keine unterirdischen Bauten errichtet werden."'\\(7408\_18)}
  \rechtsmaterie{\href{https://www.ris.bka.gv.at/NormDokument.wxe?Abfrage=LrW&Gesetzesnummer=20000006&FassungVom=2020-11-05&Artikel=&Paragraf=5&Anlage=&Uebergangsrecht=}{WBO \S 5/4/g}: \\
(4) Über die Festsetzungen nach Abs. 2 und 3 hinaus können die
Bebauungspläne zusätzlich enthalten:\\
\etc\\
g)
Grundflächen und Räume, die zur Errichtung und Duldung von öffentlichen Durchfahrten und Durchgängen, Verkehrsbauwerken und öffentlichen Aufschließungsleitungen durch die Gemeinde von jeder Bebauung freizuhalten sind und Bestimmungen über die sich daraus ergebenden Einschränkungen der Bebaubarkeit und Nutzung;
}
	\notiz{Siehe auch \merkmalref{ErrichtungGebaeude} für Verbote und Zulassungen, die sich auf Gebäude beziehen.}
  \quelle{Aus WienBV\_Bebauungsbestimmungen in \href{run:./files/200929_BRISE_IDM-REM-LOI.xlsx}{200929\_BRISE\_IDM REM LOI}
  }
\frage{Erläuterung? DRINNENLASSEN, aber eher in WienBV\_TB
  verorten, d.h. u.u. ein neues Merkmal?}
}


% \merkmal{VonBebauungFreizuhaltenAusnahme}{
%   \erlaeuterung{Ausnahmen zu: von Bebauung freizuhaltenden Bereichen (öffentlichen Durchfahrten und Durchgängen, Verkehrsbauwerken und öffentlichen Aufschließungsleitungen).}
%   \wert{Text}
%   \beispiel{
%     "`Auf der mit BB6 bezeichneten Fläche wird im Niveau der
%     anschließenden Verkehrsfläche ein Durchgang mit einer lichten Höhe
%     von 9,0 m angeordnet. Die Anordnung statisch erforderlicher
%     Konstruktionselemente ist zulässig."'\\(7990\_22)
%   }
%   \rechtsmaterie{\href{https://www.ris.bka.gv.at/NormDokument.wxe?Abfrage=LrW&Gesetzesnummer=20000006&FassungVom=2020-11-05&Artikel=&Paragraf=5&Anlage=&Uebergangsrecht=}{WBO \S 5/4/g}: \\
% (4) Über die Festsetzungen nach Abs. 2 und 3 hinaus können die
% Bebauungspläne zusätzlich enthalten:\\
% \etc\\
% g)
% Grundflächen und Räume, die zur Errichtung und Duldung von öffentlichen Durchfahrten und Durchgängen, Verkehrsbauwerken und öffentlichen Aufschließungsleitungen durch die Gemeinde von jeder Bebauung freizuhalten sind und Bestimmungen über die sich daraus ergebenden Einschränkungen der Bebaubarkeit und Nutzung;
% }
% \notiz{Siehe Merkmal \merkmalref{VonBebauungFreizuhalten} für das
%   Verbot der Bebauung.}
% \notiz{Wurde entfernt, da Kontextsensitiv. Stattdessen
%   \merkmalref{WeitereBestimmungPruefungErforderlich} oder
%   \merkmalref{AusnahmePruefungErforderlich} verwenden.}
% \quelle{Aus \href{run:./files/200526_BRISE_RIM-IDM_Auszug
%     AI-BB.xlsx}{200626\_BRISE\_RIM-IDM\_Auszug\_AI-BB}
% }
% % \frage{Ist der Wert ein Text?}
% }


\merkmal{VorkehrungBepflanzung}{
  \erlaeuterung{Spezifiziert Vorkehrungen für die Bepflanzung von
    Flächen.}
  \wert{Text}
  \beispiel{"`Für alle Flächen, für die die gärtnerische Ausgestaltung
    (G) vorgeschrieben ist, sind bei unterirdischen Baulichkeiten
    Vorkehrungen zu treffen, daß für das Pflanzen von Bäumen
    ausreichende Erdkerne vorhanden bleiben."'\\(6963\_13)\smallskip\\
	  "`In den Verkehrsflächen [\dots] sind Vorkehrungen zu treffen, dass die Pflanzung einer Baumreihe möglich ist."'\\(7181\_3)}
  \notiz{Vergleiche auch
    Merkmal~\merkmalref{AnordnungGaertnerischeAusgestaltung} für allgemeinere Anordnungen zur gärnterischen Ausgestaltung.}
  \quelle{Aus Kleinworkshop \# 1}
  \frage{DONE Was genau ist der Unterschied zu
    \merkmalref{AnordnungGaertnerischeAusgestaltung}? Ist es das
    gleiche? => Ist eher spezifischer -> wenn man es pruefen kann
    (Baeume sind im Modell eingezeichnet), dann dieses, wenn
    allgemein, dann eher anordnunggaertnerischeausgestaltung DONE}
}



% \merkmal{VerbotBebauung}{
%   \erlaeuterung{}
%   \wert{}
%   \beispiel{}
%   \rechtsmaterie{}
% }

% \merkmal{BefestigungFuerSchuleSpielplatz}{
%   \erlaeuterung{}
%   \wert{}
%   \beispiel{}
%   \rechtsmaterie{}
% }

% \merkmal{VorgabeTyp}{
%   % \erlaeuterung{}
%   \wert{Text}
%   % \beispiel{}
%   \rechtsmaterie{\href{https://www.ris.bka.gv.at/NormDokument.wxe?Abfrage=LrW&Gesetzesnummer=20000006&FassungVom=2020-11-05&Artikel=&Paragraf=70a&Anlage=&Uebergangsrecht=}{WBO \S 70a/1/ff}}
%   \notiz{aus WienBV\_Bebauungsbestimmungen}
%   \quelle{Aus \href{run:./files/200929_BRISE_IDM-REM-LOI.xlsx}{200929\_BRISE\_IDM REM LOI}
%   }
%   \frage{Erläuterung? Und ist das relevant? NEIN!}
% }

% \merkmal{AusnahmeRangierflaechen}{
%   \erlaeuterung{}
%   \wert{}
%   \beispiel{}
%   \rechtsmaterie{}
% }

% \merkmal{VorkehrungPflanzungBaeume}{
%   \erlaeuterung{}
%   \wert{}
%   \beispiel{}
%   \rechtsmaterie{}
% }

% \merkmal{UnterirdischeBauten}{
%   \erlaeuterung{}
%   \wert{}
%   \beispiel{}
%   \rechtsmaterie{}
% }


\subsection{Kategorie: Dach}
\label{sec:kategorie:dach}

\renewcommand{\category}{Dach}

Enthält Merkmale welche das Dach betreffen.

% \merkmal{AbschlussDachMax}{
%   \erlaeuterung{Festsetzung des obersten Dachabschlusses im Strukturgebiet.}
%   \wert{Eine positive Zahl [m]}
%   \beispiel{"`Im gesamten Plangebiet darf bei den zur Errichtung
%     gelangenden Gebäuden der höchste Punkt des Daches maximal 4,5 m
%     über der tatsächlich errichteten Gebäudehöhe liegen."'\\(6963\_17)}
%   \rechtsmaterie{\href{https://www.ris.bka.gv.at/NormDokument.wxe?Abfrage=LrW&Gesetzesnummer=20000006&FassungVom=2020-11-05&Artikel=&Paragraf=77&Anlage=&Uebergangsrecht=}{WBO
%       \S 77/3/c}:\\
%   (3) Über jede Struktureinheit hat der Bebauungsplan folgende Festsetzung zu enthalten:\\
%   \etc\\
%   c)
%   die maximale Gebäudehöhe oder der maximale oberste Abschluss des Daches.
%   }
%   \notiz{Siehe auch \merkmalref{GebaeudeHoeheArt} für die
%     genauere Spezifikation der Gebäudehöhe.}
% \quelle{Aus \href{run:./files/200526_BRISE_RIM-IDM_Auszug
%     AI-BB.xlsx}{200626\_BRISE\_RIM-IDM\_Auszug\_AI-BB}
% }
% \frage{2021-01-29: Koennen wir "im strukturgebiet" aus erklaerung
%   streichen? => urspruenglich war es nur fuer strukturgebiet
%   gedacht. kann es rausnehmen, weil durch die widmung abgedeckt.}
% \frage{2021-01-29: ist das "ueber dem anschliessenden Gelaende" oder
%   "ueber der Gebaeudehoehe"? Muessen wir es unterscheiden? => brauchen
% mehrere Merkmale...}
% }

\merkmal{AbschlussDachMaxBezugGebaeude}{
  \erlaeuterung{Festsetzung des obersten Dachabschlusses bezogen auf
    die Gebäudehöhe.}
  \wert{Eine positive Zahl [m]}
  \beispiel{"`Im gesamten Plangebiet darf bei den zur Errichtung
    gelangenden Gebäuden der höchste Punkt des Daches maximal 4,5 m
    über der tatsächlich errichteten Gebäudehöhe
    liegen."'\\(6963\_17)\smallskip\\
  "`Der höchste Punkt der zur Errichtung gelangenden Dächer darf die tatsächlich ausgeführte Gebäudehöhe um höchstens 4,5 m überragen."'\\(7799\_7)}
  \rechtsmaterie{\href{https://www.ris.bka.gv.at/NormDokument.wxe?Abfrage=LrW&Gesetzesnummer=20000006&FassungVom=2020-11-05&Artikel=&Paragraf=77&Anlage=&Uebergangsrecht=}{WBO
      \S 77/3/c}:\\
  (3) Über jede Struktureinheit hat der Bebauungsplan folgende Festsetzung zu enthalten:\\
  \etc\\
  c)
  die maximale Gebäudehöhe oder der maximale oberste Abschluss des Daches.
  }
  \notiz{Siehe auch \merkmalref{GebaeudeHoeheArt} für die
    genauere Spezifikation der Gebäudehöhe. Für
    Einschränkungen relativ zum angrenzenden Gelände siehe
    \merkmalref{AbschlussDachMaxBezugGelaende}.}
  \quelle{War zunächst AbschlussDachMax aus \href{run:./files/200526_BRISE_RIM-IDM_Auszug
      AI-BB.xlsx}{200626\_BRISE\_RIM-IDM\_Auszug\_AI-BB}. Nach den
    Workshops und Treffen mit Alexander Wihann 2021-01-29 aufgeteilt
    in dieses und das Merkmal \merkmalref{AbschlussDachMaxBezugGelaende}.
  }
  % \frage{2021-01-29: Koennen wir "im strukturgebiet" aus erklaerung
  %   streichen? => urspruenglich war es nur fuer strukturgebiet
  %   gedacht. kann es rausnehmen, weil durch die widmung abgedeckt.}
  % \frage{2021-01-29: ist das "ueber dem anschliessenden Gelaende" oder
  %   "ueber der Gebaeudehoehe"? Muessen wir es unterscheiden? => brauchen
  %   mehrere Merkmale...}
}


\merkmal{AbschlussDachMaxBezugGelaende}{
  \erlaeuterung{Festsetzung des obersten Dachabschlusses bezogen auf
    das umliegende Gelände.}
  \wert{Eine positive Zahl [m]}
  \beispiel{"`Der höchste Punkt des Daches darf nicht mehr als 6,5 m über dem angrenzenden Niveau liegen."'\\(7062\_11\_2)}
  \rechtsmaterie{\href{https://www.ris.bka.gv.at/NormDokument.wxe?Abfrage=LrW&Gesetzesnummer=20000006&FassungVom=2020-11-05&Artikel=&Paragraf=77&Anlage=&Uebergangsrecht=}{WBO
      \S 77/3/c}:\\
  (3) Über jede Struktureinheit hat der Bebauungsplan folgende Festsetzung zu enthalten:\\
  \etc\\
  c)
  die maximale Gebäudehöhe oder der maximale oberste Abschluss des Daches.
  }
  \notiz{Für Einschränkungen relativ zur Gebäudehöhe
    siehe \merkmalref{AbschlussDachMaxBezugGebaeude}.
  }
  \quelle{War zunächst AbschlussDachMax aus \href{run:./files/200526_BRISE_RIM-IDM_Auszug
      AI-BB.xlsx}{200626\_BRISE\_RIM-IDM\_Auszug\_AI-BB}.  Nach den
    Workshops und Treffen mit Alexander Wihann 2021-01-29 aufgeteilt
    in dieses und das Merkmal \merkmalref{AbschlussDachMaxBezugGebaeude}.
}
% \frage{2021-01-29: Koennen wir "im strukturgebiet" aus erklaerung
%   streichen? => urspruenglich war es nur fuer strukturgebiet
%   gedacht. kann es rausnehmen, weil durch die widmung abgedeckt.}
% \frage{2021-01-29: ist das "ueber dem anschliessenden Gelaende" oder
%   "ueber der Gebaeudehoehe"? Muessen wir es unterscheiden? => brauchen
% mehrere Merkmale...}
}


\merkmal{AnteilDachbegruenung}{
  \erlaeuterung{Spezifiziert den minimalen Anteil der begrünten Dachfläche relativ zur gesamten Dachfläche.}
  \wert{Eine positive Zahl [\%]}
  \beispiel{"`Auf den als Gemischtes Baugebiet/Betriebsbaugebiet und
    mit BB3 bezeichneten Flächen sind Dächer von Gebäuden
    [\dots] % , die als
    % Flachdächer mit einer Dachneigung bis zu 5 Grad ausgebildet
    % werden,
    im Ausmaß von mindestens 40\% entsprechend dem Stand der Technik zu begrünen."'\\(7443\_10)}
  % \rechtsmaterie{}
  \notiz{Für allgemeinere Bestimmungen zur Dachbegrünung siehe
  \merkmalref{BegruenungDach}.}
  \quelle{Aus den Beispielen sowie
    \href{run:./files/Merkmale_neu_liste-WP4-2020-11-06.xlsx}{Merkmale\_neu\_liste
    WP4 (Version von 2020-11-06)}
    }
    \frage{DONE 2021-01-29: Absorbiert durch BegruenungDach? Wenn nein ->
      Hinweis auf Unterschied! => macht sinn ein neues zu haben wegen
      konkreter zahl }
}

% \merkmal{BBDachneigungMax}{
%   \erlaeuterung{Festlegung der maximal möglichen Dachneigung.}
%   \wert{Eine positive Zahl [Gradzahl]}
%   \beispiel{"`In den mit BB5 und als Bauland/Gartensiedlungsgebiet gewidmeten Flächen ist eine Dachneigung von maximal 45 Grad zulässig."'\\(7443\_13)}
%   \rechtsmaterie{\href{https://www.ris.bka.gv.at/NormDokument.wxe?Abfrage=LrW&Gesetzesnummer=20000006&FassungVom=2020-11-05&Artikel=&Paragraf=5&Anlage=&Uebergangsrecht=}{WBO \S 5/4/k}:\\
% (4) Über die Festsetzungen nach Abs. 2 und 3 hinaus können die
% Bebauungspläne zusätzlich enthalten:\\
% \etc\\
% k)
% Bestimmungen über die Ausbildung der Fronten und Dächer der Gebäude,
% insbesondere über die Begrünung der Straßenfronten und der Dächer,
% sowie über die Dachneigungen, die auch mit mehr als 45 Grad, im
% Gartensiedlungsgebiet auch mit mehr als 25 Grad festgesetzt werden
% können;\\
% \href{https://www.ris.bka.gv.at/NormDokument.wxe?Abfrage=LrW&Gesetzesnummer=20000006&FassungVom=2020-11-05&Artikel=&Paragraf=81&Anlage=&Uebergangsrecht=}{WBO
%   \S 81/4}:\\
% (4) Durch das Gebäude darf jener Umriss nicht überschritten werden, der sich daraus ergibt, dass in dem nach Abs. 1 bis 3 für die Bemessung der Gebäudehöhe maßgeblichen oberen Abschluss der Gebäudefront ein Winkel von \ang{45}, im Gartensiedlungsgebiet von \ang{25}, von der Waagrechten gegen das Gebäudeinnere ansteigend, angesetzt wird. Dies gilt auch für den Fall, dass im Bebauungsplan eine besondere Bestimmung über die Höhe der Dächer festgesetzt ist. Ist im Bebauungsplan eine besondere Bestimmung über die Neigung der Dächer festgesetzt, ist der dieser Festsetzung entsprechende Winkel für die Bildung des Gebäudeumrisses maßgebend.
% }
% \notiz{Vergleiche auch Merkmal \merkmalref{DachneigungMax} für die
% maximale Dachneigung, für welche eine Bestimmung gilt.}
% \quelle{Aus \href{run:./files/200526_BRISE_RIM-IDM_Auszug
%     AI-BB.xlsx}{200626\_BRISE\_RIM-IDM\_Auszug\_AI-BB}
% }
% }

% \merkmal{BBDachneigungMin}{
%   \erlaeuterung{Festlegung der minimal möglichen Dachneigung.}
%   \wert{Eine positive Zahl [Gradzahl]}
%   \beispiel{"`Die Dachneigung der innerhalb der Schutzzone zur
%     Errichtung gelangenden Gebäuden hat nicht weniger als 25 Grad \etc\
%     % und nicht mehr als 55 Grad
%     zu betragen."'\\(7181\_7)}
%   \rechtsmaterie{\href{https://www.ris.bka.gv.at/NormDokument.wxe?Abfrage=LrW&Gesetzesnummer=20000006&FassungVom=2020-11-05&Artikel=&Paragraf=5&Anlage=&Uebergangsrecht=}{WBO
%       \S 5/4/k}:\\
% (4) Über die Festsetzungen nach Abs. 2 und 3 hinaus können die
% Bebauungspläne zusätzlich enthalten:\\
% \etc\\
% k)
% Bestimmungen über die Ausbildung der Fronten und Dächer der Gebäude,
% insbesondere über die Begrünung der Straßenfronten und der Dächer,
% sowie über die Dachneigungen, die auch mit mehr als 45 Grad, im
% Gartensiedlungsgebiet auch mit mehr als 25 Grad festgesetzt werden
% können;\\
%   \href{https://www.ris.bka.gv.at/NormDokument.wxe?Abfrage=LrW&Gesetzesnummer=20000006&FassungVom=2020-11-05&Artikel=&Paragraf=81&Anlage=&Uebergangsrecht=}{WBO
%       \S 81/4}:\\
% (4) Durch das Gebäude darf jener Umriss nicht überschritten werden,
% der sich daraus ergibt, dass in dem nach Abs. 1 bis 3 für die
% Bemessung der Gebäudehöhe maßgeblichen oberen Abschluss der
% Gebäudefront ein Winkel von \ang{45}, im Gartensiedlungsgebiet von
% \ang{25}, von der Waagrechten gegen das Gebäudeinnere ansteigend,
% angesetzt wird. Dies gilt auch für den Fall, dass im Bebauungsplan
% eine besondere Bestimmung über die Höhe der Dächer festgesetzt
% ist. Ist im Bebauungsplan eine besondere Bestimmung über die Neigung
% der Dächer festgesetzt, ist der dieser Festsetzung entsprechende
% Winkel für die Bildung des Gebäudeumrisses maßgebend.
% }
% \quelle{Aus \href{run:./files/200526_BRISE_RIM-IDM_Auszug
%     AI-BB.xlsx}{200626\_BRISE\_RIM-IDM\_Auszug\_AI-BB}
% }
% }

\merkmal{BegruenungDach}{
  \erlaeuterung{Wahr, wenn ein Dach zu begrünen ist.}
  \wert{Wahrheitswert}
  \beispiel{"`Auf der mit BB2 bezeichneten Fläche sind die Dachflächen der Gebäude als begrünte Flachdächer auszubilden."'\\(7272\_10)}
  \rechtsmaterie{\href{https://www.ris.bka.gv.at/NormDokument.wxe?Abfrage=LrW&Gesetzesnummer=20000006&FassungVom=2020-11-05&Artikel=&Paragraf=5&Anlage=&Uebergangsrecht=}{WBO \S 5/4/k}: \\
(4) Über die Festsetzungen nach Abs. 2 und 3 hinaus können die
Bebauungspläne zusätzlich enthalten:\\
\etc\\
k)
Bestimmungen über die Ausbildung der Fronten und Dächer der Gebäude, insbesondere über die Begrünung der Straßenfronten und der Dächer, sowie über die Dachneigungen, die auch mit mehr als 45 Grad, im Gartensiedlungsgebiet auch mit mehr als 25 Grad festgesetzt werden können;
}
\notiz{Für die Festlegung des Anteils der zu begrünenden
  Dachfläche siehe \merkmalref{AnteilDachbegruenung}.
}
\quelle{Aus \href{run:./files/200526_BRISE_RIM-IDM_Auszug
    AI-BB.xlsx}{200626\_BRISE\_RIM-IDM\_Auszug\_AI-BB}
}
}

\merkmal{Dachart}{
  \erlaeuterung{Spezifiziert die Dachart, beispielsweise "Flachdach",
    "Pultdach", "Glasdach", etc.}
  \wert{Text}
  \beispiel{"`Flachdächer bis zu einer Dachneigung von fünf Grad sind entsprechend dem Stand der technischen Wissenschaften zu begrünen."'\\(7181\_6)}
  % \rechtsmaterie{}
  \quelle{Aus den Beispielen sowie
    \href{run:./files/Merkmale_neu_liste-WP4-2020-11-06.xlsx}{Merkmale\_neu\_liste
    WP4 (Version von 2020-11-06)}
    }
}

\merkmal{DachflaecheMin}{
  \erlaeuterung{Spezifiziert die minimale Dachfläche.}
  \wert{Eine positive Zahl [$m^2$]}
  \beispiel{"`Die Dächer dieser Gebäude sind ab einer
    Größe von 5 m2 entsprechend dem Stand der Technik als begrünte Flachdächer auszubilden, sofern es sich nicht um Glasdächer handelt."'\\ (7408\_10)}
  % \rechtsmaterie{}
  \quelle{Aus den Beispielen sowie
    \href{run:./files/Merkmale_neu_liste-WP4-2020-11-06.xlsx}{Merkmale\_neu\_liste
    WP4 (Version von 2020-11-06)}
    }
}

% \merkmal{DachneigungMax}{
%   \erlaeuterung{Spezifiziert den maximalen Dachwinkel für
%     welchen eine Bestimmung gilt.}
%   \wert{Eine positive Zahl [Gradzahl]}
%   \beispiel{"{}Die zur Errichtung gelangenden Dächer von Gebäuden mit einer bebauten Fläche von mehr als $12 \mathrm{m}^2$ sind bis zu einer Dachneigung von 15 Grad entsprechend dem Stand der Technik zu begrünen."\\(8159\_11)% 7443\_10\_0
%   }
%   % \rechtsmaterie{}
%   \notiz{Vergleiche auch Merkmal \merkmalref{BBDachneigungMax} für
%     die Festlegung der maximal erlaubten Dachneigung.}
%   \quelle{Aus den Beispielen sowie
%     \href{run:./files/Merkmale_neu_liste-WP4-2020-11-06.xlsx}{Merkmale\_neu\_liste
%     WP4 (Version von 2020-11-06)}
%     }
%     \frage{Neu: koennen wir das mit BBDachneigungMax gleichsetzen?
%       (Wenn ja, dann auch fuer "min"?) => Er fragt nochmal nach.}
% }

\merkmal{DachneigungMax}{
  \erlaeuterung{Spezifiziert einen maximalen Dachwinkel.}
  \wert{Eine positive Zahl [Gradzahl]}
  \beispiel{"`Die zur Errichtung gelangenden Dächer von
    Gebäuden mit einer bebauten Fläche von mehr als $12
    \mathrm{m}^2$ sind bis zu einer Dachneigung von 15 Grad
    entsprechend dem Stand der Technik zu begrünen."'\\(8159\_11)\smallskip% 7443\_10\_0
    \\
    "`In den mit BB5 und als Bauland/Gartensiedlungsgebiet gewidmeten Flächen ist eine Dachneigung von maximal 45 Grad zulässig."'\\(7443\_13)
  }
\rechtsmaterie{\href{https://www.ris.bka.gv.at/NormDokument.wxe?Abfrage=LrW&Gesetzesnummer=20000006&FassungVom=2020-11-05&Artikel=&Paragraf=5&Anlage=&Uebergangsrecht=}{WBO \S 5/4/k}:\\
(4) Über die Festsetzungen nach Abs. 2 und 3 hinaus können die
Bebauungspläne zusätzlich enthalten:\\
\etc\\
k)
Bestimmungen über die Ausbildung der Fronten und Dächer der Gebäude,
insbesondere über die Begrünung der Straßenfronten und der Dächer,
sowie über die Dachneigungen, die auch mit mehr als 45 Grad, im
Gartensiedlungsgebiet auch mit mehr als 25 Grad festgesetzt werden
können;\\
\href{https://www.ris.bka.gv.at/NormDokument.wxe?Abfrage=LrW&Gesetzesnummer=20000006&FassungVom=2020-11-05&Artikel=&Paragraf=81&Anlage=&Uebergangsrecht=}{WBO
  \S 81/4}:\\
(4) Durch das Gebäude darf jener Umriss nicht überschritten werden, der sich daraus ergibt, dass in dem nach Abs. 1 bis 3 für die Bemessung der Gebäudehöhe maßgeblichen oberen Abschluss der Gebäudefront ein Winkel von \ang{45}, im Gartensiedlungsgebiet von \ang{25}, von der Waagrechten gegen das Gebäudeinnere ansteigend, angesetzt wird. Dies gilt auch für den Fall, dass im Bebauungsplan eine besondere Bestimmung über die Höhe der Dächer festgesetzt ist. Ist im Bebauungsplan eine besondere Bestimmung über die Neigung der Dächer festgesetzt, ist der dieser Festsetzung entsprechende Winkel für die Bildung des Gebäudeumrisses maßgebend.
}
  % \notiz{Vergleiche auch Merkmal \merkmalref{BBDachneigungMax} für
  %   die Festlegung der maximal erlaubten Dachneigung.}
  \quelle{Das ursprüngliche Merkmal \mml{BBDachneigungMax} aus \href{run:./files/200526_BRISE_RIM-IDM_Auszug
    AI-BB.xlsx}{200626\_BRISE\_RIM-IDM\_Auszug\_AI-BB}. Das Merkmal
  \mml{DachneigungMax} für Bedingungen aus den Beispielen sowie
    \href{run:./files/Merkmale_neu_liste-WP4-2020-11-06.xlsx}{Merkmale\_neu\_liste
    WP4 (Version von 2020-11-06)}
    }
}

\merkmal{DachneigungMin}{
  \erlaeuterung{Spezifiziert eine minimale Dachneigung.}
  \wert{Eine positive Zahl [Gradzahl]}
  \beispiel{"`Die Dachneigung der innerhalb der Schutzzone zur
    Errichtung gelangenden Gebäuden hat nicht weniger als 25 Grad \etc\
    % und nicht mehr als 55 Grad
    zu betragen."'\\(7181\_7)}
  \rechtsmaterie{\href{https://www.ris.bka.gv.at/NormDokument.wxe?Abfrage=LrW&Gesetzesnummer=20000006&FassungVom=2020-11-05&Artikel=&Paragraf=5&Anlage=&Uebergangsrecht=}{WBO
      \S 5/4/k}:\\
(4) Über die Festsetzungen nach Abs. 2 und 3 hinaus können die
Bebauungspläne zusätzlich enthalten:\\
\etc\\
k)
Bestimmungen über die Ausbildung der Fronten und Dächer der Gebäude,
insbesondere über die Begrünung der Straßenfronten und der Dächer,
sowie über die Dachneigungen, die auch mit mehr als 45 Grad, im
Gartensiedlungsgebiet auch mit mehr als 25 Grad festgesetzt werden
können;\\
  \href{https://www.ris.bka.gv.at/NormDokument.wxe?Abfrage=LrW&Gesetzesnummer=20000006&FassungVom=2020-11-05&Artikel=&Paragraf=81&Anlage=&Uebergangsrecht=}{WBO
      \S 81/4}:\\
(4) Durch das Gebäude darf jener Umriss nicht überschritten werden,
der sich daraus ergibt, dass in dem nach Abs. 1 bis 3 für die
Bemessung der Gebäudehöhe maßgeblichen oberen Abschluss der
Gebäudefront ein Winkel von \ang{45}, im Gartensiedlungsgebiet von
\ang{25}, von der Waagrechten gegen das Gebäudeinnere ansteigend,
angesetzt wird. Dies gilt auch für den Fall, dass im Bebauungsplan
eine besondere Bestimmung über die Höhe der Dächer festgesetzt
ist. Ist im Bebauungsplan eine besondere Bestimmung über die Neigung
der Dächer festgesetzt, ist der dieser Festsetzung entsprechende
Winkel für die Bildung des Gebäudeumrisses maßgebend.
}
\quelle{Das ursprüngliche Merkmal \mml{BBDachneigungMin} aus \href{run:./files/200526_BRISE_RIM-IDM_Auszug
    AI-BB.xlsx}{200626\_BRISE\_RIM-IDM\_Auszug\_AI-BB}
}
}


% \merkmal{TechnischeUndBelichtungsAufbautenZulaessig}{
%   \erlaeuterung{Wahr, wenn technische bzw.\ der Belichtung dienende
%     Dachaufbauten im erforderlichen Ausmaß zulässig sind.}
%   \wert{Wahrheitswert}
%   \beispiel{"`Flachdächer bis zu einer Dachneigung von fünf
%     Grad sind entsprechend dem Stand der technischen Wissenschaften zu
%     begrünen. Technische bzw. der Belichtung dienende Aufbauten sind im
%     erforderlichen Ausmaß zulässig"'\\(7181\_6)}
%   % \rechtsmaterie{}
%   \notiz{Vergleiche auch Merkmal
%     \merkmalref{TechnischeAufbautenZulaessig} für technische
%     Aufbauten auf der Grundfläche. Siehe auch Merkmal
%     \merkmalref{TechnischeAufbautenHoeheMax} für die
%     Beschränkung der Höhe technischer Aufbauten.}
%   \quelle{Aus den Beispielen sowie
%     \href{run:./files/Merkmale_neu_liste-WP4-2020-11-06.xlsx}{Merkmale\_neu\_liste
%     WP4 (Version von 2020-11-06)}
%     }
% }

% \merkmal{}{
%   \erlaeuterung{}
%   \wert{}
%   \beispiel{}
%   \rechtsmaterie{}
% }

% \merkmal{}{
%   \erlaeuterung{}
%   \wert{}
%   \beispiel{}
%   \rechtsmaterie{}
% }

% \merkmal{}{
%   \erlaeuterung{}
%   \wert{}
%   \beispiel{}
%   \rechtsmaterie{}
% }


\subsection{Kategorie: Einfriedungen}
\label{sec:kateg-einfr}

\renewcommand{\category}{Einfriedungen}

Enthält Merkmale, welche Einfriedungen betreffen.

\merkmal{EinfriedungAusgestaltung}{
  \erlaeuterung{Beschreibung der Ausgestaltung der zulässigen Einfriedung.}
  \wert{Text}
  \beispiel{"`Einfriedungen \etc % an seitlichen und hinteren Grundgrenzen von Liegenschaften im Bauland, für welche die gärtnerische Ausgestaltung unbebauter Grundflächen angeordnet ist,
    dürfen den freien Durchblick nicht hindern."'\\(7020\_20)}
  \rechtsmaterie{\href{https://www.ris.bka.gv.at/NormDokument.wxe?Abfrage=LrW&Gesetzesnummer=20000006&FassungVom=2020-11-05&Artikel=&Paragraf=5&Anlage=&Uebergangsrecht=}{WBO \S 5/4/s}:\\
  (4) Über die Festsetzungen nach Abs. 2 und 3 hinaus können die
  Bebauungspläne zusätzlich enthalten:\\
  \etc\\
  s)
Bestimmungen über die Ausgestaltung von Einfriedungen oder das Verbot ihrer Herstellung sowie über die Zulässigkeit, Ausgestaltung, Höhe und Lage von Lärmschutzeinrichtungen;}
\notiz{Siehe auch die Merkmale \merkmalref{EinfriedungHoeheGesamt}, \merkmalref{EinfriedungHoeheSockel} und
\merkmalref{EinfriedungLage} für Bestimmungen zur Höhe bzw.\
Lage der Einfriedungen.}
\quelle{Aus \href{run:./files/200526_BRISE_RIM-IDM_Auszug
    AI-BB.xlsx}{200626\_BRISE\_RIM-IDM\_Auszug\_AI-BB}
}
}

\merkmal{EinfriedungHoeheGesamt}{
  \erlaeuterung{Beschränkung der Gesamthöhe der Einfriedung. }
  \wert{Eine positive Zahl [m]}
  \beispiel{"`Zwischen dem mit den Buchstaben C und D gekennzeichneten Bereich der Grenz- bzw. Baulinie ist die Errichtung von vollflächigen Einfriedungen mit maximal 2,5 m Höhe zulässig."'\\(7443\_21)}
  \rechtsmaterie{\href{https://www.ris.bka.gv.at/NormDokument.wxe?Abfrage=LrW&Gesetzesnummer=20000006&FassungVom=2020-11-05&Artikel=&Paragraf=62a&Anlage=&Uebergangsrecht=}{WBO \S 62a/1/21}:\\
  § 62a (1) Bei folgenden Bauführungen ist weder eine Baubewilligung
  noch eine Bauanzeige erforderlich:\\
  \etc\\
21.
Einfriedungen bis zu einer Höhe von 2,50 m, soweit sie nicht gegen öffentliche Verkehrsflächen, Friedhöfe oder Grundflächen für öffentliche Zwecke gerichtet sind; gegen öffentliche Verkehrsflächen gerichtete Einfriedungen bis zu einer Höhe von 2,50 m, wenn sie bloß als Ersatz für Einfriedungen, die im Zuge des Ausbaus dieser Verkehrsfläche abgebrochen wurden, errichtet werden;
}
\quelle{Aus \href{run:./files/200526_BRISE_RIM-IDM_Auszug
    AI-BB.xlsx}{200626\_BRISE\_RIM-IDM\_Auszug\_AI-BB}\\
  Ebenso aus den Beispielen sowie
    \href{run:./files/Merkmale_neu_liste-WP4-2020-11-06.xlsx}{Merkmale\_neu\_liste
    WP4 (Version von 2020-11-06)}
}
% \frage{Ist das das gleiche wie \merkmalref{EinfriedungMaxHoehe}? JA,
%   siehe dort}
}

\merkmal{EinfriedungHoeheSockel}{
  \erlaeuterung{Beschränkung der Höhe des Sockels der Einfriedung.}
  \wert{Eine positive Zahl [m]}
  % \beispiel{}
  \rechtsmaterie{\href{https://www.ris.bka.gv.at/NormDokument.wxe?Abfrage=LrW&Gesetzesnummer=20000006&FassungVom=2020-11-05&Artikel=&Paragraf=62a&Anlage=&Uebergangsrecht=}{WBO
      \S 62a/1/21}:\\
  § 62a (1) Bei folgenden Bauführungen ist weder eine Baubewilligung
  noch eine Bauanzeige erforderlich:\\
  \etc\\
21.
Einfriedungen bis zu einer Höhe von 2,50 m, soweit sie nicht gegen öffentliche Verkehrsflächen, Friedhöfe oder Grundflächen für öffentliche Zwecke gerichtet sind; gegen öffentliche Verkehrsflächen gerichtete Einfriedungen bis zu einer Höhe von 2,50 m, wenn sie bloß als Ersatz für Einfriedungen, die im Zuge des Ausbaus dieser Verkehrsfläche abgebrochen wurden, errichtet werden;
}
\quelle{Aus \href{run:./files/200526_BRISE_RIM-IDM_Auszug
    AI-BB.xlsx}{200626\_BRISE\_RIM-IDM\_Auszug\_AI-BB}
}
}

\merkmal{EinfriedungLage}{
  \erlaeuterung{Beschreibung der Lage der Einfriedung.}
  \wert{Text}
  \beispiel{"`Einfriedungen an seitlichen und hinteren Grundgrenzen
    von Liegenschaften im Bauland \etc %, für welche die gärtnerische Ausgestaltung unbebauter Grundflächen angeordnet ist,
    dürfen den freien Durchblick nicht hindern."'\\(7020\_20)}
  \rechtsmaterie{\href{https://www.ris.bka.gv.at/NormDokument.wxe?Abfrage=LrW&Gesetzesnummer=20000006&FassungVom=2020-11-05&Artikel=&Paragraf=5&Anlage=&Uebergangsrecht=}{WBO \S 5/4/s}:\\
  (4) Über die Festsetzungen nach Abs. 2 und 3 hinaus können die
  Bebauungspläne zusätzlich enthalten:\\
  \etc\\
  s)
Bestimmungen über die Ausgestaltung von Einfriedungen oder das Verbot ihrer Herstellung sowie über die Zulässigkeit, Ausgestaltung, Höhe und Lage von Lärmschutzeinrichtungen;}
\quelle{Aus \href{run:./files/200526_BRISE_RIM-IDM_Auszug
    AI-BB.xlsx}{200626\_BRISE\_RIM-IDM\_Auszug\_AI-BB}
}
}

% \merkmal{EinfriedungMaxHoehe}{
%   \erlaeuterung{Spezifiziert die maximale Höhe einer Einfriedung.}
%   \wert{Eine positive Zahl [m]}
%   \beispiel{"`Zwischen dem mit den Buchstaben C und D gekennzeichneten Bereich der Grenz- bzw. Baulinie ist die Errichtung von vollflächigen Einfriedungen mit maximal 2,5 m Höhe zulässig."'\\(7443\_21)}
%   % \rechtsmaterie{}
%   \quelle{Aus den Beispielen sowie
%     \href{run:./files/Merkmale_neu_liste-WP4-2020-11-06.xlsx}{Merkmale\_neu\_liste
%     WP4 (Version von 2020-11-06)}
%     }
%     \frage{Ist das das gleiche wie
%       \merkmalref{EinfriedungHoeheGesamt}? JA, also KOMBINIEREN in eins!}
% }

\merkmal{EinfriedungZulaessig}{
  \erlaeuterung{Ist die Errichtung von Einfriedungen zulässig oder nicht.}
  \wert{Wahrheitswert}
  \beispiel{"`Zwischen dem mit den Buchstaben C und D gekennzeichneten Bereich der Grenz- bzw. Baulinie ist die Errichtung von vollflächigen Einfriedungen mit maximal 2,5 m Höhe zulässig."'\_(7443\_21)}
  \rechtsmaterie{\href{https://www.ris.bka.gv.at/NormDokument.wxe?Abfrage=LrW&Gesetzesnummer=20000006&FassungVom=2020-11-05&Artikel=&Paragraf=5&Anlage=&Uebergangsrecht=}{WBO \S 5/4/s}:\\
  (4) Über die Festsetzungen nach Abs. 2 und 3 hinaus können die
  Bebauungspläne zusätzlich enthalten:\\
  \etc\\
  s)
Bestimmungen über die Ausgestaltung von Einfriedungen oder das Verbot ihrer Herstellung sowie über die Zulässigkeit, Ausgestaltung, Höhe und Lage von Lärmschutzeinrichtungen;
}
\quelle{Aus \href{run:./files/200526_BRISE_RIM-IDM_Auszug
    AI-BB.xlsx}{200626\_BRISE\_RIM-IDM\_Auszug\_AI-BB}
}
}

% \merkmal{}{
%   \erlaeuterung{}
%   \wert{}
%   \beispiel{}
%   \rechtsmaterie{}
% }

\subsection{Kategorie: Fläche}
\label{sec:kategorie:flache}

\renewcommand{\category}{Fläche}

Enthält Merkmale, welche eine Fläche betreffen. % Siehe auch
% Abschnitt~\ref{sec:kateg-grossb-hochh} für Merkmale bezüglich
% der Fläche speziell bei Grossbauvorhaben, Einkaufszentren, etc.
\smallskip

\merkmal{Flaechen}{
  \erlaeuterung{Sammelmerkmal für alle Bestimmungen, welche sich auf eine
    Fläche beziehen.}
  \beispiel{
    "`Innerhalb der als Bauland gewidmeten und mit G
    bezeichneten Flächen dürfen unterirdische Bauten oder Bauteile nur
    in einem Ausmaß von maximal 20 v. H. des Bauplatzes errichtet
    werden."' \\(7408\_13)\smallskip\\
    "`Auf den mit BB 1 bezeichneten Flächen des Wohngebietes
    dürfen bei einer Bauplatzgröße über 300 $m^2$ max. 25\% des
    Bauplatzes bebaut werden."'\\(7412\_14) \smallskip\\
    "`Im Bauland/Wohngebiet darf die Bruttogeschoßfläche aller
    Geschoße, die ganz oder teilweise über dem anschließenden Gelände
    liegen, insgesamt höchstens 11.500 $\mathrm{m}^2$
    betragen."'\\(8025\_15) \smallskip\\
    "`Die mit BB 15 bezeichnete und als Grünland/
    Erholungsgebiet Sport- und Spielplätze gewidmete Grundfläche darf
    bis zu einem Ausmaß von höchstens 20 v. H. dieser durch
    Fluchtlinien bestimmten Fläche bebaut werden."'\\
    (7020\_38) \smallskip\\
    "`Innerhalb der mit BB 13 bezeichneten und als Grünland/
    Parkschutzgebiet gewidmeten Grundfläche dürfen zwei Gebäude mit
    einer gesamten bebauten Fläche von max. 400 $m^2$ und einer
    Gebäudehöhe von max. 9 m errichtet werden."'\\(7020\_35) \smallskip\\
    "`Auf den mit BB 6 bezeichneten und als Erholungsgebiet/
    Kleingartengebiet gewidmeten Flächen darf das Ausmaß der bebauten
    Fläche 25 $m^2$ je Kleingarten nicht überschreiten."'\\(7020\_47) \smallskip\\
    "`Auf der mit Esp BB2 bezeichneten Fläche dürfen nur
    Gebäude und bauliche Anlagen in einem Ausmaß von insgesamt maximal
    10 \% der Grundfläche \etc\ % und einer Gebäudehöhe von maximal 7,5 m
    errichtet werden."'\\(7545\_11) \smallskip\\
    "`In den als Wohnzone ausgewiesenen Bereichen ist \etc\ nur die
    Errichtung von Wohngebäuden zulässig, in denen nicht weniger als
    80 v.H. der Summe der Nutzflächen der Hauptgeschoße, jedoch unter
    Ausschluß des Erdgeschoßes, Wohnzwecken vorbehalten
    sind."'\\(6963\_29) \smallskip\\
    "`Die mit Spk/BB3 bezeichnete Fläche darf im Ausmaß von höchstens
    40 \% bebaut werden."'\\(7774\_11) \smallskip\\
    "`Auf den mit BB2 bezeichneten Teilen des Wohn- oder gemischten
    Baugebietes darf die bebaute Fläche maximal 20\% der Bauplatzgröße
    betragen."'\\(7408\_16) \smallskip\\
    "`Das Ausmaß der bebaubaren Fläche wird mit 100 $m^2$
    limitiert."'\\(7443\_14) \smallskip\\
    "`Die Gebäudehöhe darf 7,5 m nicht überschreiten, die bebaute
    Fläche hat maximal 250 $m^2$ zu betragen."'\\(7167\_15)\smallskip\\
    "`Die mit Nebengebäuden bebaute Grundfläche darf höchstens
    30m2 je Bauplatz betragen"'\\(7531\_9) \smallskip\\
    "`Nicht bebaute, jedoch bebaubare Grundflächen sind gärtnerisch
    auszugestalten."'\\(7020\_16) \smallskip\\
    "`Auf den mit BB3 bezeichneten Grundflächen sind die nicht
    bebauten Grundflächen gärtnerisch auszugestalten."'\\(7702\_9) \smallskip\\
    "`Auf den mit EKZ BB4 bezeichneten Flächen ist die Errichtung
    eines Einkaufszentrums zulässig, wobei die von Räumen gemäß § 7c
    Abs. 1 BO für Wien in Anspruch genommene Gesamtfläche 4.300 $m^2$
    nicht überschreiten darf."'\\(7443\_12)
  }
  \notiz{
    Fasst etliche vorherige Merkmale bezüglich bebaubarer
    Fläche, flächenmäßiger Ausnützbarkeit und
    anderer Aspekte der Grundflächen zusammen,
    siehe Anhang~\ref{changes:2021-04-12}
    für die Details.
  }
  \quelle{Aus Diskussion 2021-04-12. Meta-Merkmal.}
}


% \merkmal{BauplatzUnterirdischeBebauungMax}{
%   \erlaeuterung{Spezifiziert das maximale Ausmaß der Fläche unterirdischer Bauten/Bauteile relativ zur Fläche des Bauplatzes}
%   \wert{Eine positive Zahl [\%]}
%   \beispiel{"`Innerhalb der als Bauland gewidmeten und mit G
%     bezeichneten Flächen dürfen unterirdische Bauten oder Bauteile nur
%     in einem Ausmaß von maximal 20 v. H. des Bauplatzes errichtet
%     werden."' \\(7408\_13)}
%   % \rechtsmaterie{}
%   \quelle{Aus den Beispielen sowie
%     \href{run:./files/Merkmale_neu_liste-WP4-2020-11-06.xlsx}{Merkmale\_neu\_liste
%     WP4 (Version von 2020-11-06)}
%     }
% }

% \merkmal{BauplatzGroesseMin}{
%   \erlaeuterung{Spezifiziert eine Mindestgröße für einen Bauplatz.}
%   \wert{Eine positive Zahl [$\mathrm{m}^2$]}
%   \beispiel{"`Auf den mit BB 1 bezeichneten Flächen des Wohngebietes
%     dürfen bei einer Bauplatzgröße über 300 $m^2$ max. 25\% des Bauplatzes bebaut werden."'\\(7412\_14)}
%   \quelle{Annotations Kleinworkshop 2021-03-01.}
%   % \notiz{}
% }

% \merkmal{AusnuetzbarkeitFlaecheBGF}{
%   \label{merkmal:BBAusnuetzbarkeitFlaecheBGF}
%   \erlaeuterung{Maximale BGF-bezogene Ausnützbarkeit (m2).}
%   \wert{Eine positive Zahl [$m^2$]}
%   \beispiel{"`Im Bauland/Wohngebiet darf die Bruttogeschoßfläche aller Geschoße, die ganz oder teilweise über dem anschließenden Gelände liegen, insgesamt höchstens 11.500 $\mathrm{m}^2$ betragen."'\\(8025\_15)}
%   \rechtsmaterie{\href{https://www.ris.bka.gv.at/NormDokument.wxe?Abfrage=LrW&Gesetzesnummer=20000006&FassungVom=2020-11-05&Artikel=&Paragraf=5&Anlage=&Uebergangsrecht=}{WBO \S 5/4/d,e}: \\
% (4) Über die Festsetzungen nach Abs. 2 und 3 hinaus können die
% Bebauungspläne zusätzlich enthalten:\\
% \etc\\
% d)
% Bestimmungen über die flächenmäßige beziehungsweise volumenbezogene Ausnützbarkeit der Bauplätze und der Baulose oder von Teilen davon; in Gebieten für geförderten Wohnbau Bestimmungen über den Anteil der Wohnnutzfläche der auf einem Bauplatz geschaffenen Wohnungen und Wohneinheiten in Heimen, die hinsichtlich der Grundkostenangemessenheit dem Wiener Wohnbauförderungs- und Wohnhaussanierungsgesetz – WWFSG 1989 entsprechen müssen;\\
% e)
% Bestimmungen über die bauliche Ausnützbarkeit von ländlichen Gebieten,
% Parkanlagen, Freibädern, Parkschutzgebieten und Grundflächen für
% Badehütten, bei Gewässern auch die Ausweisung der von jeder Bebauung
% freizuhaltenden Uferzonen; Bestimmungen über die bauliche
% Ausnützbarkeit von Sport- und Spielplätzen, bei Sportplätzen auch in
% bezug auf Sporthallen, sowie eine höchstens zulässige bebaubare
% Fläche, bezogen auf eine durch Grenzlinien bestimmte Grundfläche;
% Bestimmungen über die Ausnützbarkeit der Sondernutzungsgebiete
% hinsichtlich der Art, des Zweckes, ihres Umfanges und ihrer Abgrenzung
% zu Nutzungen anderer Art sowie hinsichtlich der endgültigen Gestaltung
% ihrer Oberflächen unter Festsetzung der beabsichtigten Wirkung auf das
% örtliche Stadt- bzw. Landschaftsbild nach der endgültigen Widmung der
% Widmungskategorie Grünland für die endgültige Nutzung der Grundflächen
% durch Bestimmung von Geländehöhen (Überhöhungen und Vertiefungen),
% Böschungswinkeln, Bepflanzungen der endgültigen baulichen
% Ausnützbarkeit und ähnlichem; die Festsetzung eines Zeitpunktes für
% die Herstellung der endgültigen Widmung ist zulässig;
% }
% \quelle{Aus \href{run:./files/200526_BRISE_RIM-IDM_Auszug
%     AI-BB.xlsx}{200626\_BRISE\_RIM-IDM\_Auszug\_AI-BB}
% }
% }

% \merkmal{AusnuetzbarkeitFlaecheBGFRelativ}{
%   \erlaeuterung{Maximale BGF-bezogene Ausnützbarkeit (relativ/prozentual).}
%   \wert{Eine positive Zahl [\%]}
%   % \beispiel{}
%   \rechtsmaterie{\href{https://www.ris.bka.gv.at/NormDokument.wxe?Abfrage=LrW&Gesetzesnummer=20000006&FassungVom=2020-11-05&Artikel=&Paragraf=5&Anlage=&Uebergangsrecht=}{WBO
%       \S 5/4/d,e}\\siehe Merkmal \merkmalref{AusnuetzbarkeitFlaecheBGF} % 
%   }
% \quelle{Aus \href{run:./files/200526_BRISE_RIM-IDM_Auszug
%     AI-BB.xlsx}{200626\_BRISE\_RIM-IDM\_Auszug\_AI-BB}
% }
% }


% \merkmal{AusnuetzbarkeitFlaecheFluchtlinienbezugRelativ}{
%   \erlaeuterung{Maximale flächenmäßige Ausnützbarkeit
%     bezogen auf eine von Fluchtlinien bestimmten Fläche.}
%   \wert{Eine positive Zahl [\%]}
%   \beispiel{"`Die mit BB 15 bezeichnete und als Grünland/
%     Erholungsgebiet Sport- und Spielplätze gewidmete Grundfläche darf
%     bis zu einem Ausmaß von höchstens 20 v. H. dieser durch
%     Fluchtlinien bestimmten Fläche bebaut werden."'\\
%     (7020\_38)
%   }
%   % \rechtsmaterie{}
%   \notiz{}
%   \quelle{Kleinworkshop 2021-01-22}
% }


% \merkmal{AusnuetzbarkeitFlaecheGrundflaechenbezug}{
%   \erlaeuterung{Maximale flächenmäßige Ausnützbarkeit bezogen auf die Grundfläche (m2).}
%   \wert{Eine positive Zahl [$m^2$]}
%   \beispiel{"`Innerhalb der mit BB 13 bezeichneten und als Grünland/
%     Parkschutzgebiet gewidmeten Grundfläche dürfen zwei Gebäude mit
%     einer gesamten bebauten Fläche von max. 400 $m^2$ und einer
%     Gebäudehöhe von max. 9 m errichtet werden."'\\(7020\_35)\\
%   "`Auf den mit BB 6 bezeichneten und als Erholungsgebiet/ Kleingartengebiet gewidmeten Flächen darf das Ausmaß der bebauten Fläche 25 $m^2$ je Kleingarten nicht überschreiten."'\\(7020\_47)}
%   \rechtsmaterie{\href{https://www.ris.bka.gv.at/NormDokument.wxe?Abfrage=LrW&Gesetzesnummer=20000006&FassungVom=2020-11-05&Artikel=&Paragraf=5&Anlage=&Uebergangsrecht=}{WBO
%       \S 5/4/d,e}:\\siehe Merkmal \merkmalref{AusnuetzbarkeitFlaecheBGF}% 
%   }
%   \notiz{Siehe auch \merkmalref{AnordnungGaertnerischeAusgestaltung}
%     für den Bezug auf die gärtnerische Ausgestaltung.
%   }
% \quelle{Aus \href{run:./files/200526_BRISE_RIM-IDM_Auszug
%     AI-BB.xlsx}{200626\_BRISE\_RIM-IDM\_Auszug\_AI-BB}
% }
% \frage{DONE Neu (workshop 2020-12-11): Bezieht sich das auch auf
%   gaertnerische Ausgestaltung etc? (siehe 6963\_26: "mindestens X
%   prozent gaertnerisch auszugestalten"). Ist das maximal oder minimal?
% => GaertnerischeAusgestaltungProzentual (neues Merkmal) => Verweis auf
% das Merkmal; ist die maximale ausnuetzbarkeit! => Ueberall bei
% Ausnuetzbarkeit reinschreiben}
% }

% \merkmal{AusnuetzbarkeitFlaecheGrundflaechenbezugRelativ}{
%   \erlaeuterung{Maximale flächenmäßige Ausnützbarkeit bezogen auf die Grundfläche (relativ/prozentual).}
%   \wert{Eine positive Zahl [\%]}
%   \beispiel{"`Auf der mit Esp BB2 bezeichneten Fläche dürfen nur
%     Gebäude und bauliche Anlagen in einem Ausmaß von insgesamt maximal
%     10 \% der Grundfläche \etc\ % und einer Gebäudehöhe von maximal 7,5 m
%     errichtet werden."'\\(7545\_11)}
%   \rechtsmaterie{\href{https://www.ris.bka.gv.at/NormDokument.wxe?Abfrage=LrW&Gesetzesnummer=20000006&FassungVom=2020-11-05&Artikel=&Paragraf=5&Anlage=&Uebergangsrecht=}{WBO
%       \S 5/4/d,e}}:\\siehe Merkmal \merkmalref{AusnuetzbarkeitFlaecheBGF}% 
% \quelle{Aus \href{run:./files/200526_BRISE_RIM-IDM_Auszug
%     AI-BB.xlsx}{200626\_BRISE\_RIM-IDM\_Auszug\_AI-BB}
% }
% }

% \merkmal{AusnuetzbarkeitFlaecheNutzflaeche}{
%   \erlaeuterung{Maximale nutzflächenmäßige Ausnützbarkeit bezogen auf die Grundfläche (m2).}
%   \wert{Eine positive Zahl [$m^2$]}
%   % \beispiel{}
%   \rechtsmaterie{\href{https://www.ris.bka.gv.at/NormDokument.wxe?Abfrage=LrW&Gesetzesnummer=20000006&FassungVom=2020-11-05&Artikel=&Paragraf=5&Anlage=&Uebergangsrecht=}{WBO
%       \S 5/4/d,e}\\siehe Merkmal
%     \merkmalref{AusnuetzbarkeitFlaecheBGF} 
%   }
% \quelle{Aus \href{run:./files/200526_BRISE_RIM-IDM_Auszug
%     AI-BB.xlsx}{200626\_BRISE\_RIM-IDM\_Auszug\_AI-BB}
% }
% }

% \merkmal{AusnuetzbarkeitFlaecheNutzflaecheRelativ}{
%   \erlaeuterung{Maximale nutzflächenmäßige Ausnützbarkeit bezogen auf die Grundfläche (relativ/prozentual).}
%   \wert{Eine positive Zahl [\%]}
%   % \beispiel{}
%   \rechtsmaterie{\href{https://www.ris.bka.gv.at/NormDokument.wxe?Abfrage=LrW&Gesetzesnummer=20000006&FassungVom=2020-11-05&Artikel=&Paragraf=5&Anlage=&Uebergangsrecht=}{WBO
%       \S 5/4/d,e}\\siehe Merkmal \merkmalref{AusnuetzbarkeitFlaecheBGF}
%   }
% \quelle{Aus \href{run:./files/200526_BRISE_RIM-IDM_Auszug
%     AI-BB.xlsx}{200626\_BRISE\_RIM-IDM\_Auszug\_AI-BB}
% }
% }

% \merkmal{AusnuetzbarkeitFlaecheWohnnutzflaeche}{
%   \erlaeuterung{Maximale wohnnutzflächenmäßige Ausnützbarkeit bezogen auf die Grundfläche (m2).}
%   \wert{Eine positive Zahl [$m^2$]}
%   % \beispiel{}
%   \rechtsmaterie{\href{https://www.ris.bka.gv.at/NormDokument.wxe?Abfrage=LrW&Gesetzesnummer=20000006&FassungVom=2020-11-05&Artikel=&Paragraf=5&Anlage=&Uebergangsrecht=}{WBO
%       \S 5/4/d,e}\\siehe Merkmal \merkmalref{AusnuetzbarkeitFlaecheBGF}% 
%   }
% \quelle{Aus \href{run:./files/200526_BRISE_RIM-IDM_Auszug
%     AI-BB.xlsx}{200626\_BRISE\_RIM-IDM\_Auszug\_AI-BB}
% }
% }

% \merkmal{AusnuetzbarkeitFlaecheWohnnutzflaecheRelativMax}{
%   \erlaeuterung{Maximale wohnnutzflächenmäßige Ausnützbarkeit bezogen auf die Grundfläche (relativ/prozentual).}
%   \wert{Eine positive Zahl [\%]}
%   % \beispiel{}
%   \rechtsmaterie{\href{https://www.ris.bka.gv.at/NormDokument.wxe?Abfrage=LrW&Gesetzesnummer=20000006&FassungVom=2020-11-05&Artikel=&Paragraf=5&Anlage=&Uebergangsrecht=}{WBO
%       \S 5/4/d,e}\\siehe Merkmal \merkmalref{AusnuetzbarkeitFlaecheBGF}% 
%   }
% \quelle{Aus \href{run:./files/200526_BRISE_RIM-IDM_Auszug
%     AI-BB.xlsx}{200626\_BRISE\_RIM-IDM\_Auszug\_AI-BB}
% }
% \frage{DONE Neu: (workshop 2020-12-11): Ist das minimal oder maximal? In
%   6963\_29 ist es "minimal". => Maximal, aber koennte auch minimal
%   sein => aufteilen in Max und Min (aber nur Wohnnutzflaeche)}
% }

% \merkmal{AusnuetzbarkeitFlaecheWohnnutzflaecheRelativMin}{
%   \erlaeuterung{Minimale wohnnutzflächenmäßige Ausnützbarkeit bezogen auf die Grundfläche (relativ/prozentual).}
%   \wert{Eine positive Zahl [\%]}
%   \beispiel{"`In den als Wohnzone ausgewiesenen Bereichen ist \etc\ nur die Errichtung von Wohngebäuden zulässig, in denen nicht weniger als 80 v.H. der Summe der Nutzflächen der Hauptgeschoße, jedoch unter Ausschluß des Erdgeschoßes, Wohnzwecken vorbehalten sind."'\\(6963\_29)
%   }
%   \rechtsmaterie{\href{https://www.ris.bka.gv.at/NormDokument.wxe?Abfrage=LrW&Gesetzesnummer=20000006&FassungVom=2020-11-05&Artikel=&Paragraf=5&Anlage=&Uebergangsrecht=}{WBO
%       \S 5/4/d,e}\\siehe Merkmal \merkmalref{AusnuetzbarkeitFlaecheBGF}% 
%   }
% \quelle{Aus Workshop 2020-12-11
% }
% \frage{DONE Neu: (workshop 2020-12-11): Ist das minimal oder maximal? In
%   6963\_29 ist es "minimal". => Maximal, aber koennte auch minimal
%   sein => aufteilen in Max und Min (aber nur Wohnnutzflaeche)}
% }

% \merkmal{BebaubareFlaecheAbgegrenzt}{
%   \erlaeuterung{Die nach § 5 Abs. 4 lit. d durch den Bebauungsplan beschränkte bebaubare Fläche des Bauplatzes (relativ/prozentual).}
%   % \erlaeuterung{Beschränkt die bebaubare Fläche pro Fläche
%   % (relativ).}
%   \wert{Eine positive Zahl [\%]}
%   \beispiel{"`Die mit Spk/BB3 bezeichnete Fläche darf im Ausmaß von höchstens 40 \% bebaut werden."'\\(7774\_11)}
%   \rechtsmaterie{\href{https://www.ris.bka.gv.at/NormDokument.wxe?Abfrage=LrW&Gesetzesnummer=20000006&FassungVom=2020-11-05&Artikel=&Paragraf=82&Anlage=&Uebergangsrecht=}{WBO \S 82/5}:\\
%   (5) Die durch Nebengebäude in Anspruch genommene Grundfläche ist auf die nach den gesetzlichen Ausnutzbarkeitsbestimmungen bebaubare Fläche und die die nach § 5 Abs. 4 lit. d durch den Bebauungsplan beschränkte bebaubare Fläche des Bauplatzes anzurechnen. Im Gartensiedlungsgebiet ist die mit einem Nebengebäude bebaute Grundfläche auf die Ausnutzbarkeitsbestimmungen eines Bauloses dann anzurechnen, wenn die bebaubare Fläche im Bebauungsplan mit mindestens 100 m2 festgesetzt ist.
% }
% \quelle{Aus \href{run:./files/200526_BRISE_RIM-IDM_Auszug
%     AI-BB.xlsx}{200626\_BRISE\_RIM-IDM\_Auszug\_AI-BB}
% }
% % \frage{Die Rechtsmaterie scheint sich nur auf Nebengebäude zu
% %   beziehen. Das Merkmal auch? NEIN! -> geklaert}
% }

% \merkmal{BebaubareFlaecheGesamterBauplatz}{
%   \erlaeuterung{Einschränkung der bebaubaren Fläche des gesamten Bauplatzes (relativ/prozentual).}
%   % \erlaeuterung{Beschränkt die bebaubare Fläche auf dem
%   %   Bauplatz relativ zur Fläche des gesamten Bauplatzes.}
%   \wert{Eine positive Zahl [\%]}
%   \beispiel{"`Auf den mit BB2 bezeichneten Teilen des Wohn- oder gemischten Baugebietes darf die bebaute Fläche maximal 20\% der Bauplatzgröße betragen."'\\(7408\_16)}
%   \rechtsmaterie{\href{https://www.ris.bka.gv.at/NormDokument.wxe?Abfrage=LrW&Gesetzesnummer=20000006&FassungVom=2020-11-05&Artikel=&Paragraf=82&Anlage=&Uebergangsrecht=}{WBO
%       \S 82/5}}:\\
%   (5) Die durch Nebengebäude in Anspruch genommene Grundfläche ist auf die nach den gesetzlichen Ausnutzbarkeitsbestimmungen bebaubare Fläche und die die nach § 5 Abs. 4 lit. d durch den Bebauungsplan beschränkte bebaubare Fläche des Bauplatzes anzurechnen. Im Gartensiedlungsgebiet ist die mit einem Nebengebäude bebaute Grundfläche auf die Ausnutzbarkeitsbestimmungen eines Bauloses dann anzurechnen, wenn die bebaubare Fläche im Bebauungsplan mit mindestens 100 m2 festgesetzt ist.
% \quelle{Aus \href{run:./files/200526_BRISE_RIM-IDM_Auszug
%     AI-BB.xlsx}{200626\_BRISE\_RIM-IDM\_Auszug\_AI-BB}
% }
% % \frage{Die Rechtsmaterie scheint sich nur auf Nebengebäude zu
% %   beziehen. Das Merkmal auch? NEIN -> geklaert}
% }

% \merkmal{BebaubareFlaecheJeBauplatz}{
%   \erlaeuterung{Einschränkung der bebaubaren Fläche je Bauplatz (m2)}
%   % \erlaeuterung{Beschränkt die bebaubare Fläche auf dem
%   %   Bauplatz absolut.}
%   \wert{Eine positive Zahl [$m^2$]}
%   \beispiel{"`Das Ausmaß der bebaubaren Fläche wird mit 100 $m^2$ limitiert."'\\(7443\_14)}
%   \rechtsmaterie{\href{https://www.ris.bka.gv.at/NormDokument.wxe?Abfrage=LrW&Gesetzesnummer=20000006&FassungVom=2020-11-05&Artikel=&Paragraf=82&Anlage=&Uebergangsrecht=}{WBO \S 82/5}:\\
%   (5) Die durch Nebengebäude in Anspruch genommene Grundfläche ist auf die nach den gesetzlichen Ausnutzbarkeitsbestimmungen bebaubare Fläche und die die nach § 5 Abs. 4 lit. d durch den Bebauungsplan beschränkte bebaubare Fläche des Bauplatzes anzurechnen. Im Gartensiedlungsgebiet ist die mit einem Nebengebäude bebaute Grundfläche auf die Ausnutzbarkeitsbestimmungen eines Bauloses dann anzurechnen, wenn die bebaubare Fläche im Bebauungsplan mit mindestens 100 m2 festgesetzt ist.
% }
% \notiz{Vergleiche auch
%   \merkmalref{BebauteFlaechefuerNebengebaeudeJeBauplatzMax} für
%   die Beschränkung der bebaubare Fläche für Nebengebäude.}
% \quelle{Aus \href{run:./files/200526_BRISE_RIM-IDM_Auszug
%     AI-BB.xlsx}{200626\_BRISE\_RIM-IDM\_Auszug\_AI-BB}
% }
% % \frage{Die Rechtsmaterie scheint sich nur auf Nebengebäude zu
% %   beziehen. Das Merkmal auch? NEIN -> geklaert}
% \frage{DONE Sind die Beschränkungen "je Bauplatz" auch "je
%   Bauplatzteil" relevant? Das kommt wohl in älteren Verordnungen
%   vor. => Er fragt. Aber aufnehmen}
% }

% \merkmal{BebaubareFlaecheJeBauplatzteil}{
%   \erlaeuterung{Einschränkung der bebaubaren Fläche je Bauplatzteil (m2)}
%   % \erlaeuterung{Beschränkt die bebaubare Fläche auf dem
%   %   Bauplatz absolut.}
%   \wert{Eine positive Zahl [$m^2$]}
%   % \beispiel{}
%   % \rechtsmaterie{\href{https://www.ris.bka.gv.at/NormDokument.wxe?Abfrage=LrW&Gesetzesnummer=20000006&FassungVom=2020-11-05&Artikel=&Paragraf=82&Anlage=&Uebergangsrecht=}{WBO \S 82/5}:\\
% %   (5) Die durch Nebengebäude in Anspruch genommene Grundfläche ist auf die nach den gesetzlichen Ausnutzbarkeitsbestimmungen bebaubare Fläche und die die nach § 5 Abs. 4 lit. d durch den Bebauungsplan beschränkte bebaubare Fläche des Bauplatzes anzurechnen. Im Gartensiedlungsgebiet ist die mit einem Nebengebäude bebaute Grundfläche auf die Ausnutzbarkeitsbestimmungen eines Bauloses dann anzurechnen, wenn die bebaubare Fläche im Bebauungsplan mit mindestens 100 m2 festgesetzt ist.
% % }
% \quelle{Aus workshop 
% }
% % \frage{DONE Die Rechtsmaterie scheint sich nur auf Nebengebäude zu
% %   beziehen. Das Merkmal auch? NEIN -> geklaert}
% % \frage{Sind die Beschränkungen "je Bauplatz" auch "je
% %   Bauplatzteil" relevant? Das kommt wohl in älteren Verordnungen
% %   vor. => Er fragt. Aber aufnehmen}
% }

% \merkmal{BebaubareFlaecheJeGebaeude}{
%   \erlaeuterung{Einschränkung der bebaubaren Fläche je Gebäude (m2).}
%   \wert{Eine positive Zahl [$m^2$]}
%   \beispiel{"`Die Gebäudehöhe darf 7,5 m nicht überschreiten, die bebaute Fläche hat maximal 250 $m^2$ zu betragen."'\\(7167\_15)}
%   \rechtsmaterie{\href{https://www.ris.bka.gv.at/NormDokument.wxe?Abfrage=LrW&Gesetzesnummer=20000006&FassungVom=2020-11-05&Artikel=&Paragraf=82&Anlage=&Uebergangsrecht=}{WBO \S 82/5}:\\
%   (5) Die durch Nebengebäude in Anspruch genommene Grundfläche ist auf die nach den gesetzlichen Ausnutzbarkeitsbestimmungen bebaubare Fläche und die die nach § 5 Abs. 4 lit. d durch den Bebauungsplan beschränkte bebaubare Fläche des Bauplatzes anzurechnen. Im Gartensiedlungsgebiet ist die mit einem Nebengebäude bebaute Grundfläche auf die Ausnutzbarkeitsbestimmungen eines Bauloses dann anzurechnen, wenn die bebaubare Fläche im Bebauungsplan mit mindestens 100 m2 festgesetzt ist.}
% \quelle{Aus \href{run:./files/200526_BRISE_RIM-IDM_Auszug
%     AI-BB.xlsx}{200626\_BRISE\_RIM-IDM\_Auszug\_AI-BB}
% }
% % \frage{Die Rechtsmaterie scheint sich nur auf Nebengebäude zu
% %   beziehen. Das Merkmal auch? NEIN -> geklaert}
% }

% % \merkmal{BefestigungFuerSchuleOderSpielplatz}{
% %   \erlaeuterung{}
% %   \wert{}
% %   \beispiel{}
% %   \rechtsmaterie{}
% % }

% % \merkmal{BebaubareFlaeche}{
% %   % \erlaeuterung{}
% %   \wert{Eine positive Zahl [$m^2$]}
% %   % \beispiel{}
% %   % \rechtsmaterie{}
% %   \notiz{aus WienBV\_Bebauungsbestimmungen}
% %   \quelle{Aus \href{run:./files/200929_BRISE_IDM-REM-LOI.xlsx}{200929\_BRISE\_IDM REM LOI}
% %   }
% %   \frage{Erläuterung? Ist das relevant? EHER NEIN}
% % }

% % \merkmal{BebaubareFlaecheBeschraenkung}{
% %   % \erlaeuterung{}
% %   \wert{Eine positive Zahl [$m^2$]}
% %   % \beispiel{}
% %   % \rechtsmaterie{}
% %   \notiz{aus WienBV\_Bebauungsbestimmungen}
% %   \quelle{Aus \href{run:./files/200929_BRISE_IDM-REM-LOI.xlsx}{200929\_BRISE\_IDM REM LOI}
% %   }
% %   \frage{Erläuterung? Ist das relevant? EHER NEIN}
% %   \frage{Ist das das gleiche wie \merkmalref{BebaubareFlaeche}?}
% % }

% % \merkmal{BebauteFlaechefuerNebengebaeudeMax}{
% %   \erlaeuterung{Spezifiziert das maximale Ausmaß der mit
% %     Nebengebäuden bebaubaren Grundfläche.}
% %   \wert{Eine positive Zahl [$m^2$]}
% %   \beispiel{"`Die mit Nebengebäuden bebaute Grundfläche darf höchstens
% %     30m2 je Bauplatz betragen"'\\(7531\_9)}
% %   % \rechtsmaterie{}
% %   \quelle{Aus den Beispielen sowie
% %     \href{run:./files/Merkmale_neu_liste-WP4-2020-11-06.xlsx}{Merkmale\_neu\_liste
% %     WP4 (Version von 2020-11-06)}
% %     }
% % }

% \merkmal{BebauteFlaechefuerNebengebaeudeJeBauplatzMax}{
%   \erlaeuterung{Spezifiziert das maximale Ausmaß der mit
%     Nebengebäuden bebaubaren Grundfläche auf einem Bauplatz.}
%   \wert{Eine positive Zahl [$m^2$]}
%   \beispiel{"`Die mit Nebengebäuden bebaute Grundfläche darf höchstens
%     30m2 je Bauplatz betragen"'\\(7531\_9)}
%   % \rechtsmaterie{}
%   \quelle{Aus den Beispielen sowie
%     \href{run:./files/Merkmale_neu_liste-WP4-2020-11-06.xlsx}{Merkmale\_neu\_liste
%     WP4 (Version von 2020-11-06)}. Die Unterscheidung Bauplatz/Baulos
%   aus Annotation Kleinworkshop 2020-11-27.
%     }
% }

% \merkmal{BebauteFlaechefuerNebengebaeudeJeBaulosMax}{
%   \erlaeuterung{Spezifiziert das maximale Ausmaß der mit
%     Nebengebäuden bebaubaren Grundfläche auf einem Baulos.}
%   \wert{Eine positive Zahl [$m^2$]}
%   % \beispiel{"`Die mit Nebengebäuden bebaute Grundfläche darf höchstens
%   %   30m2 je Bauplatz betragen"'\\(7531\_9)}
%   % \rechtsmaterie{}
%   \quelle{Aus den Beispielen sowie
%     \href{run:./files/Merkmale_neu_liste-WP4-2020-11-06.xlsx}{Merkmale\_neu\_liste
%     WP4 (Version von 2020-11-06)}. Die Unterscheidung Bauplatz/Baulos
%   aus Annotation Kleinworkshop 2020-11-27.
%     }
%     \frage{DONE Neu: Unterscheidung Bauplatz/Baulos -> neues Merkmal. Passt
%     das? => ja.}
% }

% % \merkmal{ReferenzFlaecheFWBP}{
% %   % \erlaeuterung{}
% %   \wert{Wahrheitswert}
% %   % \beispiel{}
% %   % \rechtsmaterie{}
% %   \notiz{aus WienBV\_Bebauungsbestimmungen}
% %   \quelle{Aus \href{run:./files/200929_BRISE_IDM-REM-LOI.xlsx}{200929\_BRISE\_IDM REM LOI}
% %   }
% %   \frage{Erläuterung? Ist das relevant? NEIN}
% % }

% % \merkmal{WidmungAusnutzbarkeitFlaeche}{
% %   % \erlaeuterung{}
% %   \wert{Eine positive Zahl [$m^2$]}
% %   % \beispiel{}
% %   \rechtsmaterie{\href{https://www.ris.bka.gv.at/NormDokument.wxe?Abfrage=LrW&Gesetzesnummer=20000006&FassungVom=2020-11-05&Artikel=&Paragraf=5&Anlage=&Uebergangsrecht=}{WBO
% %       \S 5/4/d}:\\
% %     (4) Über die Festsetzungen nach Abs. 2 und 3 hinaus können die
% % Bebauungspläne zusätzlich enthalten:\\
% % \etc\\
% % d)
% % Bestimmungen über die flächenmäßige beziehungsweise volumenbezogene Ausnützbarkeit der Bauplätze und der Baulose oder von Teilen davon; in Gebieten für geförderten Wohnbau Bestimmungen über den Anteil der Wohnnutzfläche der auf einem Bauplatz geschaffenen Wohnungen und Wohneinheiten in Heimen, die hinsichtlich der Grundkostenangemessenheit dem Wiener Wohnbauförderungs- und Wohnhaussanierungsgesetz – WWFSG 1989 entsprechen müssen;
% % }
% %   \notiz{aus WienBV\_Bebauungsbestimmungen}
% %   \quelle{Aus \href{run:./files/200929_BRISE_IDM-REM-LOI.xlsx}{200929\_BRISE\_IDM REM LOI}
% %   }
% %   \frage{Erläuterung? Ist das relevant? EHER NEIN}
% % }

% \merkmal{FlaecheBebaubar}{
%   \erlaeuterung{Gibt an, ob eine Grundfläche bebaubar ist. % ("wahr",
%     % wenn bebaubar).
%   }
%   \wert{Wahrheitswert ["`wahr"' wenn bebaubar]}
%   \beispiel{"`Nicht bebaute, jedoch bebaubare Grundflächen sind gärtnerisch auszugestalten."'\\(7020\_16)}
%   % \rechtsmaterie{}
%   \notiz{Oft in Verbindung mit Merkmal \merkmalref{FlaecheBebaut}.}
%   \quelle{Aus Kleinworkshop 2020-11-27.}
%   % \frage{DONE 2020-11-27: Neues Merkmal => drinnenlassen}
% }


% \merkmal{FlaecheBebaut}{
%   \erlaeuterung{Gibt an, ob eine Grundfläche bebaut ist. %  ("wahr"
%     % wenn bebaut).
%   }
%   \wert{Wahrheitswert ["`wahr wenn bebaut"']}
%   \beispiel{"`Nicht bebaute, jedoch bebaubare Grundflächen sind
%     gärtnerisch auszugestalten."'\\(7020\_16)\\
%   "`Auf den mit BB3 bezeichneten Grundflächen sind die nicht bebauten Grundflächen gärtnerisch auszugestalten."'\\(7702\_9)}
%   % \rechtsmaterie{}
%   \notiz{Oft in Verbindung mit Merkmal \merkmalref{FlaecheBebaubar}.}
%   \quelle{Aus Kleinworkshop 2020-11-27.}
%   % \frage{DONE 2020-11-27: Neues Merkmal  => drinnenlassen}
% }

% % \merkmal{}{
% %   \erlaeuterung{}
% %   \wert{}
% %   \beispiel{}
% %   \rechtsmaterie{}
% % }




\subsection{Kategorie: Geschosse}
\label{sec:kategorie:geschosse}

\renewcommand{\category}{Geschosse}

Enthält Merkmale welche die Anzahl der Geschosse betreffen.


\merkmal{MaxAnzahlDachgeschosse}{
  \erlaeuterung{Beschränkung der maximal zulässigen Anzahl für Dachgeschosse.}
  \wert{Eine positive Zahl [Anzahl]}
  \beispiel{"`Dachgeschosse sind nicht zulässig."'\\(7443\_15) \smallskip\\
  "`Die Gebäude dürfen mit nur einem Dachgeschoß errichtet werden, wobei der höchste Punkt des Daches nicht mehr als 4,5 m über der tatsächlich errichteten Gebäudehöhe liegen darf."'\\(7400\_18)}
  \rechtsmaterie{\href{https://www.ris.bka.gv.at/NormDokument.wxe?Abfrage=LrW&Gesetzesnummer=20000006&FassungVom=2020-11-05&Artikel=&Paragraf=5&Anlage=&Uebergangsrecht=}{WBO
      \S 5/4/h}:\\
  (4) Über die Festsetzungen nach Abs. 2 und 3 hinaus können die
  Bebauungspläne zusätzlich enthalten:\\
  \etc\\
  h)
Bestimmungen über die Gebäudehöhe, im Bauland bei Festsetzung einer Bauklasse nur bis zu deren Grenzen, ferner über die Höhe von sonstigen Bauwerken, sowie über die höchstens zulässige Zahl der Geschosse, die zur Gänze oder zu einem Teil über dem anschließenden Gelände liegen;}
\quelle{Für das ursprüngliche Merkmal
  \mml{MaxAnzahlGeschosseOberirdischDachgeschoss} aus \href{run:./files/200526_BRISE_RIM-IDM_Auszug
    AI-BB.xlsx}{200626\_BRISE\_RIM-IDM\_Auszug\_AI-BB}
}
}


\merkmal{MaxAnzahlGeschosseOberirdisch}{
  \erlaeuterung{Beschränkung der maximal zulässigen Anzahl an Geschossen für das gesamte Gebäude.}
  \wert{Eine positive Zahl [Anzahl]}
  \beispiel{"`Innerhalb der mit BB5 bezeichneten und als
    Bauland/Wohngebiet gewidmeten Grundflächen sind maximal 4
    Hauptgeschoße zulässig."'\\(7792\_22)\smallskip\\
    "`Für die mit BB13 bezeichneten Grundflächen wird bestimmt: Gebäude dürfen höchstens mit einem Geschoß, das zur Gänze oder zum Teil über dem anschließenden Gelände liegt, errichtet werden."'\\(8038\_31)
  }
  \rechtsmaterie{\href{https://www.ris.bka.gv.at/NormDokument.wxe?Abfrage=LrW&Gesetzesnummer=20000006&FassungVom=2020-11-05&Artikel=&Paragraf=5&Anlage=&Uebergangsrecht=}{WBO \S 5/4/h}:\\
  (4) Über die Festsetzungen nach Abs. 2 und 3 hinaus können die
  Bebauungspläne zusätzlich enthalten:\\
  \etc\\
  h)
Bestimmungen über die Gebäudehöhe, im Bauland bei Festsetzung einer Bauklasse nur bis zu deren Grenzen, ferner über die Höhe von sonstigen Bauwerken, sowie über die höchstens zulässige Zahl der Geschosse, die zur Gänze oder zu einem Teil über dem anschließenden Gelände liegen;
}
\notiz{Für die Beschränkung von Geschossen für den
  Verwendungszweck "`Einkaufszentrum"', siehe Merkmal \merkmalref{ZulaessigeGeschossanzahlEinkaufszentrum}}
\quelle{Aus \href{run:./files/200526_BRISE_RIM-IDM_Auszug
    AI-BB.xlsx}{200626\_BRISE\_RIM-IDM\_Auszug\_AI-BB}
}
\frage{2021-01-29: Sind "Geschosse" dasselbe wie "Hauptgeschosse"?
  Muessen wir das differenzieren? (sonst Beispiel hinzufuegen) =>
  wahrscheinlich bezieht es sich nur auf hauptgeschosse => er gibt bescheid.}
}

\merkmal{MaxAnzahlGeschosseOberirdischOhneDachgeschoss}{
  \erlaeuterung{Beschränkung der maximal zulässigen Anzahl an Geschossen ohne Berücksichtigung des Dachgeschosse.}
  \wert{Eine positive Zahl [Anzahl]}
  % \beispiel{}
  \rechtsmaterie{\href{https://www.ris.bka.gv.at/NormDokument.wxe?Abfrage=LrW&Gesetzesnummer=20000006&FassungVom=2020-11-05&Artikel=&Paragraf=5&Anlage=&Uebergangsrecht=}{WBO
      \S 5/4/h}:\\
  (4) Über die Festsetzungen nach Abs. 2 und 3 hinaus können die
  Bebauungspläne zusätzlich enthalten:\\
  \etc\\
  h)
Bestimmungen über die Gebäudehöhe, im Bauland bei Festsetzung einer Bauklasse nur bis zu deren Grenzen, ferner über die Höhe von sonstigen Bauwerken, sowie über die höchstens zulässige Zahl der Geschosse, die zur Gänze oder zu einem Teil über dem anschließenden Gelände liegen;}
\quelle{Aus \href{run:./files/200526_BRISE_RIM-IDM_Auszug
    AI-BB.xlsx}{200626\_BRISE\_RIM-IDM\_Auszug\_AI-BB}
}
}

% \merkmal{ZulaessigeGeschossanzahl}{
%   \erlaeuterung{}
%   \wert{}
%   \beispiel{}
%   \rechtsmaterie{}
% }

\merkmal{Stockwerk}{
  \erlaeuterung{Spezifiziert ein Stockwerk, z.B. "`im Erdgeschoss"'.}
  \wert{Text}
  \beispiel{"`Auf den mit BB5 bezeichneten Flächen [\dots] % des Gemischten Baugebiets
    dürfen im Erdgeschoß keine Fenster von Aufenthaltsräumen von Wohnungen gegen die öffentlichen Verkehrsflächen hergestellt werden."'\\(7272\_13)}
  % \rechtsmaterie{}
  \quelle{Aus den Beispielen sowie
    \href{run:./files/Merkmale_neu_liste-WP4-2020-11-06.xlsx}{Merkmale\_neu\_liste
    WP4 (Version von 2020-11-06)}
    }
}


\merkmal{UnterirdischeBaulichkeiten}{
  \erlaeuterung{Wahr, wenn unterirdische Baulichkeiten vorhanden sind.}
  \wert{Wahrheitswert}
  \beispiel{"`Für alle Flächen, für die die gärtnerische Ausgestaltung (G) vorgeschrieben ist, sind bei unterirdischen Baulichkeiten Vorkehrungen zu treffen, daß für das Pflanzen von Bäumen ausreichende Erdkerne vorhanden bleiben."'\\(6963\_17)}
  % \rechtsmaterie{}
  \quelle{Kleinworkshop \# 1}
}


% \merkmal{VerbotUnterirdischeBauwerke}{
%   \erlaeuterung{Wahr, wenn die Errichtung von unterirdischen Bauten
%     untersagt ist.}
%   \wert{Wahrheitswert}
%   \beispiel{"`Auf den mit G/BB4 bezeichneten Grundflächen dürfen keine unterirdischen Bauten errichtet werden."'\\(7408\_18)}
%   \quelle{Kleinworkshop 2020-12-11. Jetzt absorbiert in \merkmalref{VonBebauungFreizuhalten}.}
%   \notiz{Siehe auch
%     \merkmalref{VerbotUnterirdischeBauwerkeUeberBaufluchtlinie}
%     für das Verbot mit unterirdischen Baulichkeiten über die
%     Baufluchtlinie zu ragen.}
% }

\merkmal{ZulaessigeGeschossanzahlEinkaufszentrum}{
  \erlaeuterung{Maximal zulässige Anzahl an Geschoßen für die Nutzung Einkaufzentrum bei Geschäftsstraßen.}
  \wert{Eine positive Zahl [Anzahl]}
%  \beispiel{}
  \rechtsmaterie{WBO \S 5/4/c}
  \quelle{Aus \href{run:./files/200526_BRISE_RIM-IDM_Auszug
      AI-BB.xlsx}{200626\_BRISE\_RIM-IDM\_Auszug\_AI-BB}
  }
  \notiz{Bezieht sich nur auf Einkaufszentren (z.B. EKZ nur auf den ersten
    zwei/drei Stockwerken). Für die Maximalanzahl von Geschossen
    allgemein siehe Merkmale
    \merkmalref{MaxAnzahlGeschosseOberirdisch},
    \merkmalref{MaxAnzahlDachgeschosse} oder
    \merkmalref{MaxAnzahlGeschosseOberirdischOhneDachgeschoss}.}
  % \frage{Erläuterung? OK}
}


% \merkmal{}{
%   \erlaeuterung{}
%   \wert{}
%   \beispiel{}
%   \rechtsmaterie{}
% }




\subsection{Kategorie: Grossbauvorhaben, Hochhäuser,
  Einkaufszentren, Geschäftsstrassen}
\label{sec:kateg-grossb-hochh}

\renewcommand{\category}{Grossbauvorhaben, Hochhäuser,
  Einkaufszentren, Geschäftsstrassen}

Enthält Merkmale, welche Grossbauvorhaben, Hochhäuser,
Einkaufszentren oder Geschäftsstrassen betreffen.

\merkmal{BestimmungenFuerHochhausUndGrossbauvorhaben}{
  \erlaeuterung{Beschreibung des Zweckes dem die Gebäude zuzuführen sind.}
  \wert{Text}
  % \beispiel{}
  \rechtsmaterie{\href{https://www.ris.bka.gv.at/NormDokument.wxe?Abfrage=LrW&Gesetzesnummer=20000006&FassungVom=2020-11-05&Artikel=&Paragraf=5&Anlage=&Uebergangsrecht=}{WBO \S 5/4/z}: \\
(4) Über die Festsetzungen nach Abs. 2 und 3 hinaus können die
Bebauungspläne zusätzlich enthalten:\\
\etc\\
z)
besondere Bestimmungen für Hochhäuser und Großbauvorhaben hinsichtlich der Zweckbestimmungen innerhalb der Widmungskategorie, denen die Gebäude zuzuführen sind, sowie hinsichtlich der baulichen und volumsbezogenen Gliederung und Gestaltung, ferner Zweckbestimmungen über die Verwendung von Gebäuden in Parkschutzgebieten sowie in Wohngebieten und gemischten Baugebieten innerhalb der jeweils festgesetzten Widmungskategorie.
}
\quelle{Aus \href{run:./files/200526_BRISE_RIM-IDM_Auszug
    AI-BB.xlsx}{200626\_BRISE\_RIM-IDM\_Auszug\_AI-BB}
}
}

% \merkmal{EinkaufszentrumMaxFlaeche}{
%   \erlaeuterung{Maximal Fläche des Einkaufzentrums je Widmungsfläche.}
%   \wert{Eine positive Zahl [$m^2$]}
%   \beispiel{"`Auf den mit EKZ BB4 bezeichneten Flächen ist die Errichtung eines Einkaufszentrums zulässig, wobei die von Räumen gemäß § 7c Abs. 1 BO für Wien in Anspruch genommene Gesamtfläche 4.300 $m^2$ nicht überschreiten darf."'\\(7443\_12)}
%   \rechtsmaterie{\href{https://www.ris.bka.gv.at/NormDokument.wxe?Abfrage=LrW&Gesetzesnummer=20000006&FassungVom=2020-11-05&Artikel=&Paragraf=7c&Anlage=&Uebergangsrecht=}{WBO
%       \S 7c/3}:\\
%   (3) Für Einkaufszentren kann im Bebauungsplan eine höchstens zulässige Fläche (Abs. 1), bezogen auf eine durch Fluchtlinien bestimmte Grundfläche, festgelegt werden; ferner kann zur Sicherung der räumlich funktionellen Nahebeziehungen, der zeitgemäßen Bedürfnisse und der sozialen Struktur der Bevölkerung festgelegt werden, dass nur Fachmärkte, aber keine Einkaufszentren für Lebens- und Genussmittel der Grundversorgung errichtet werden dürfen.}
% \quelle{Aus \href{run:./files/200526_BRISE_RIM-IDM_Auszug
%     AI-BB.xlsx}{200626\_BRISE\_RIM-IDM\_Auszug\_AI-BB}
% }
% \notiz{Jetzt absorbiert durch \merkmalref{Flaechen}.}
% }

% \merkmal{EinkaufszentrumZweck}{
%   \erlaeuterung{Wenn die Bestimmung für die maximal Fläche des Einkaufzentrums für eine bestimmten Verwendungszwecks beschränkt ist.}
%   \wert{Text}
%   % \beispiel{}
%   \rechtsmaterie{\href{https://www.ris.bka.gv.at/NormDokument.wxe?Abfrage=LrW&Gesetzesnummer=20000006&FassungVom=2020-11-05&Artikel=&Paragraf=7c&Anlage=&Uebergangsrecht=}{WBO \S 7c/3}:\\
%   (3) Für Einkaufszentren kann im Bebauungsplan eine höchstens zulässige Fläche (Abs. 1), bezogen auf eine durch Fluchtlinien bestimmte Grundfläche, festgelegt werden; ferner kann zur Sicherung der räumlich funktionellen Nahebeziehungen, der zeitgemäßen Bedürfnisse und der sozialen Struktur der Bevölkerung festgelegt werden, dass nur Fachmärkte, aber keine Einkaufszentren für Lebens- und Genussmittel der Grundversorgung errichtet werden dürfen.}
% % \notiz{Siehe Merkmal \merkmalref{} für die Beschränkung der Fläche.}
% \quelle{Aus \href{run:./files/200526_BRISE_RIM-IDM_Auszug
%     AI-BB.xlsx}{200626\_BRISE\_RIM-IDM\_Auszug\_AI-BB}
% }
% % \frage{Klärung Erklärung? NUTZUNGSART}
% \notiz{Jetzt absorbiert durch \merkmalref{WidmungUndZweckbestimmung}.}
% }

\merkmal{Geschaeftsstrassen}{
  \erlaeuterung{Legt fest ob eine Verkehrsfläche die an einen Bauplatz grenzt eine Geschäftsstraße ist.}
  \wert{Text}
  \beispiel{
    "`Auf Liegenschaften, die unmittelbar an der Geschäftsstraße liegen, sind Einkaufszentren innerhalb des im DETAILPLAN (Beilage 2) dargestellten Bereichs zulässig."'\\(8038\_14)
  }
  \rechtsmaterie{\href{https://www.ris.bka.gv.at/NormDokument.wxe?Abfrage=LrW&Gesetzesnummer=20000006&FassungVom=2020-11-05&Artikel=&Paragraf=5&Anlage=&Uebergangsrecht=}{WBO \S 5/4/c}: \\
(4) Über die Festsetzungen nach Abs. 2 und 3 hinaus können die
Bebauungspläne zusätzlich enthalten:\\
\etc\\
c)
Geschäftsstraßen und Einkaufszentren; ferner an Geschäftsstraßen Bestimmungen über die höchstens zulässige Zahl der Geschosse, in denen die Nutzung für ein Einkaufszentrum zulässig ist;
}
\quelle{Aus \href{run:./files/200526_BRISE_RIM-IDM_Auszug
    AI-BB.xlsx}{200626\_BRISE\_RIM-IDM\_Auszug\_AI-BB}
}
}

% \merkmal{GrossbauvorhabenMaxFlaeche}{
%   \erlaeuterung{Maximal Fläche des Großbauvorhabens je Widmungsfläche.}
%   \wert{Eine positive Zahl [$m^2$]}
%   % \beispiel{}
%   \rechtsmaterie{\href{https://www.ris.bka.gv.at/NormDokument.wxe?Abfrage=LrW&Gesetzesnummer=20000006&FassungVom=2020-11-05&Artikel=&Paragraf=7b&Anlage=&Uebergangsrecht=}{WBO
%       \S 7b/6}:\\
%   (6) Für Großbauvorhaben kann im Bebauungsplan eine höchstens zulässige Fläche, auch für einzelne festgesetzte Zwecke, bezogen auf eine durch Fluchtlinien bestimmte Grundfläche, festgelegt werden.}
% \quelle{Aus \href{run:./files/200526_BRISE_RIM-IDM_Auszug
%     AI-BB.xlsx}{200626\_BRISE\_RIM-IDM\_Auszug\_AI-BB}
% }
% }

% \merkmal{GrossbauvorhabenZweck1}{
%   \erlaeuterung{Wenn die Bestimmung für die maximal Fläche des Großbauvorhabens für eine bestimmten Verwendungszwecks beschränkt ist.}
%   \wert{Text}
%   % \beispiel{}
%   \rechtsmaterie{\href{https://www.ris.bka.gv.at/NormDokument.wxe?Abfrage=LrW&Gesetzesnummer=20000006&FassungVom=2020-11-05&Artikel=&Paragraf=7b&Anlage=&Uebergangsrecht=}{WBO \S 7b/6}:\\
%   (6) Für Großbauvorhaben kann im Bebauungsplan eine höchstens zulässige Fläche, auch für einzelne festgesetzte Zwecke, bezogen auf eine durch Fluchtlinien bestimmte Grundfläche, festgelegt werden.}
% \quelle{Aus \href{run:./files/200526_BRISE_RIM-IDM_Auszug
%     AI-BB.xlsx}{200626\_BRISE\_RIM-IDM\_Auszug\_AI-BB}
% }
% % \frage{Klärung Erklärung?}
% }

\merkmal{HochhausZulaessigGemaessBB}{
  \erlaeuterung{Wahr, wenn die Errichtung von Hochhäusern erlaubt ist.}
  \wert{Wahrheitswert}
  \beispiel{"`Die Errichtung von Hochhäusern gemäß § 7f der BO Wien ist untersagt."'\\(8250	\_11\_0)}
  \rechtsmaterie{\href{https://www.ris.bka.gv.at/NormDokument.wxe?Abfrage=LrW&Gesetzesnummer=20000006&FassungVom=2020-11-05&Artikel=&Paragraf=7f&Anlage=&Uebergangsrecht=}{WBO \S 7f}:\\
  § 7f. (1) Hochhäuser sind Gebäude, deren oberster Abschluss einschließlich aller Dachaufbauten gemäß § 81 Abs. 6 und 7 mehr als 35 m über dem tiefsten Punkt des anschließenden Geländes beziehungsweise der festgesetzten Höhenlage der anschließenden Verkehrsfläche liegt.\\
(2) Sofern der Bebauungsplan nicht anderes bestimmt, sind Hochhäuser nur im Wohngebiet und gemischten Baugebiet in der Bauklasse VI sowie im Industriegebiet, im Sondergebiet und in Strukturgebieten auf Grundflächen, für die im Bebauungsplan ein oberster Abschluss gemäß Abs. 1 in einer Höhe von mehr als 35 m festgesetzt ist, zulässig.}
\quelle{Aus \href{run:./files/200526_BRISE_RIM-IDM_Auszug
    AI-BB.xlsx}{200626\_BRISE\_RIM-IDM\_Auszug\_AI-BB}
}
}







\subsection{Kategorie: Höhe}
\label{sec:kategorie:hohe}

\renewcommand{\category}{Höhe}

Enthält Merkmale, welche die Höhe betreffen.\medskip


% \merkmal{Bauklasse}{
%   % \erlaeuterung{}
%   \wert{Text}
%   % \beispiel{}
%   % \rechtsmaterie{}
%   \notiz{aus WienBV\_Bebauungsbestimmungen}
%   \quelle{Aus \href{run:./files/200929_BRISE_IDM-REM-LOI.xlsx}{200929\_BRISE\_IDM REM LOI}
%   }
%   \frage{Erläuterung? Ist das relevant? Ist es das gleiche Merkmal
%   wie \merkmalref{BauklasseID}? RAUSWERFEN}
% }

\merkmal{Bauklasse}{
  \erlaeuterung{Spezifiziert die Bauklasse eines Gebäudes (Optionen “I” bis “VI”).}
  \wert{Text}
  \beispiel{"{}Die Dächer der auf den mit der Festsetzung [\dots] % Bauland/
    % Wohngebiet, Geschäftsviertel,
    Bauklasse I bezeichneten
    Grundflächen zur Errichtung gelangenden Gebäude sind bis zu einer
    Gebäudehöhe von 6 m entsprechend dem Stand der Technik als
    begrünte Flachdächer auszubilden [\dots].% , sofern es sich nicht um
    % Glasdächer handelt.
    "\\(7531\_8)% 7408\_11\_0
  }
  \quelle{Aus den Beispielen sowie
    \href{run:./files/Merkmale_neu_liste-WP4-2020-11-06.xlsx}{Merkmale\_neu\_liste
    WP4 (Version von 2020-11-06)}. Note: War früher \mml{BauklasseID}.
}
% \frage{Ist das das gleiche Merkmal wir \merkmalref{Bauklasse}? JA,
%   ABER AUS DEN TEXTUELLEN BESTIMMUNGEN, ALSO BEHALTEN}
}

% \merkmal{Bauklasse6Maximum}{
%   \erlaeuterung{Aus VMP: Grenzmaße der Bauklasse 6, z.B. Gebäudehöhe min. 32m, max. 38m > Minimal-Grenzwert}
%   \wert{Eine positive Zahl [m]}
%   % \beispiel{}
%   \rechtsmaterie{\href{https://www.ris.bka.gv.at/NormDokument.wxe?Abfrage=LrW&Gesetzesnummer=20000006&FassungVom=2020-11-05&Artikel=&Paragraf=75&Anlage=&Uebergangsrecht=}{WBO \S 75/3}:\\
%   (3) In der Bauklasse VI beträgt die Gebäudehöhe mindestens 21 m; der Bebauungsplan hat die einzuhaltenden Gebäudehöhen innerhalb zweier Grenzmaße festzusetzen.
% }
%   \notiz{aus WienBV\_Bebauungsbestimmungen}
%   \quelle{Aus \href{run:./files/200929_BRISE_IDM-REM-LOI.xlsx}{200929\_BRISE\_IDM REM LOI}
%   }
%   \frage{Erläuterung? Ist das relevant? NEIN}
% }

% \merkmal{Bauklasse6Minimum}{
%   \erlaeuterung{Aus VMP: Grenzmaße der Bauklasse 6, z.B. Gebäudehöhe min. 32m, max. 38m > Maximal-Grenzwert}
%   \wert{Eine positive Zahl [m]}
%   % \beispiel{}
%   \rechtsmaterie{\href{https://www.ris.bka.gv.at/NormDokument.wxe?Abfrage=LrW&Gesetzesnummer=20000006&FassungVom=2020-11-05&Artikel=&Paragraf=75&Anlage=&Uebergangsrecht=}{WBO \S 75/3}:\\
%   (3) In der Bauklasse VI beträgt die Gebäudehöhe mindestens 21 m; der Bebauungsplan hat die einzuhaltenden Gebäudehöhen innerhalb zweier Grenzmaße festzusetzen.
% }
%   \notiz{aus WienBV\_Bebauungsbestimmungen}
%   \quelle{Aus \href{run:./files/200929_BRISE_IDM-REM-LOI.xlsx}{200929\_BRISE\_IDM REM LOI}
%   }
%   \frage{Erläuterung? Ist das relevant? NEIN}
% }

% \merkmal{BauklasseMaximum}{
%   \erlaeuterung{Aus VMP (Angabe in m)}
%   \wert{Eine positive Zahl [m]}
%   % \beispiel{}
%   \rechtsmaterie{\href{https://www.ris.bka.gv.at/NormDokument.wxe?Abfrage=LrW&Gesetzesnummer=20000006&FassungVom=2020-11-05&Artikel=&Paragraf=75&Anlage=&Uebergangsrecht=}{WBO \S 75/3}:\\
%   (3) In der Bauklasse VI beträgt die Gebäudehöhe mindestens 21 m; der Bebauungsplan hat die einzuhaltenden Gebäudehöhen innerhalb zweier Grenzmaße festzusetzen.
% }
%   \notiz{aus WienBV\_Bebauungsbestimmungen}
%   \quelle{Aus \href{run:./files/200929_BRISE_IDM-REM-LOI.xlsx}{200929\_BRISE\_IDM REM LOI}
%   }
%   \frage{Erläuterung? Ist das relevant? Ist das das gleiche wie
%     \merkmalref{Bauklasse6Maximum}? RAUSWERFEN}
% }

% \merkmal{BauklasseMinimum}{
%   \erlaeuterung{Aus VMP (Angabe in m)}
%   \wert{Eine positive Zahl [m]}
%   % \beispiel{}
%   \rechtsmaterie{\href{https://www.ris.bka.gv.at/NormDokument.wxe?Abfrage=LrW&Gesetzesnummer=20000006&FassungVom=2020-11-05&Artikel=&Paragraf=75&Anlage=&Uebergangsrecht=}{WBO \S 75/3}:\\
%   (3) In der Bauklasse VI beträgt die Gebäudehöhe mindestens 21 m; der Bebauungsplan hat die einzuhaltenden Gebäudehöhen innerhalb zweier Grenzmaße festzusetzen.
% }
%   \notiz{aus WienBV\_Bebauungsbestimmungen}
%   \quelle{Aus \href{run:./files/200929_BRISE_IDM-REM-LOI.xlsx}{200929\_BRISE\_IDM REM LOI}
%   }
%   \frage{Erläuterung? Ist das relevant? Ist das das gleiche wie \merkmalref{Bauklasse6Minimum}?  RAUSWERFEN}
% }

\merkmal{BauklasseVIHoeheMax}{ %BBBauklasseMaximum
  \erlaeuterung{Begrenzung der maximalen Gebäudehöhe in der Bauklasse
    VI % durch die Gebäudeklasse
    .}
  \wert{Eine positive Zahl [m]}
  % \beispiel{}
  \rechtsmaterie{\href{https://www.ris.bka.gv.at/NormDokument.wxe?Abfrage=LrW&Gesetzesnummer=20000006&FassungVom=2020-11-05&Artikel=&Paragraf=75&Anlage=&Uebergangsrecht=}{WBO \S 75/3}:\\
  (3) In der Bauklasse VI beträgt die Gebäudehöhe mindestens 21 m; der Bebauungsplan hat die einzuhaltenden Gebäudehöhen innerhalb zweier Grenzmaße festzusetzen.
}
% \notiz{}
\quelle{Für das ursprüngliche \mml{BBBauklasseMaximum}: Aus \href{run:./files/200526_BRISE_RIM-IDM_Auszug
    AI-BB.xlsx}{200626\_BRISE\_RIM-IDM\_Auszug\_AI-BB}
}
% \frage{Gilt das nur für Bauklasse 6? Ist es das gleiche wie
%   \merkmalref{BauklasseMaximum} oder \merkmalref{Bauklasse6Maximum}?
%   NUR DIESES BEHALTEN; BEZIEHT SICH NUR AUF BAUKLASSE 6}
}

\merkmal{BauklasseVIHoeheMin}{ %BBBauklasseMinimum
  \erlaeuterung{Einschränkung der mind. Gebäudehöhe der Bauklasse VI.}
  \wert{Eine positive Zahl [m]}
  % \beispiel{}
  \rechtsmaterie{\href{https://www.ris.bka.gv.at/NormDokument.wxe?Abfrage=LrW&Gesetzesnummer=20000006&FassungVom=2020-11-05&Artikel=&Paragraf=75&Anlage=&Uebergangsrecht=}{WBO
      \S 75/3}}:\\
  (3) In der Bauklasse VI beträgt die Gebäudehöhe mindestens 21 m; der Bebauungsplan hat die einzuhaltenden Gebäudehöhen innerhalb zweier Grenzmaße festzusetzen.
\quelle{Für das ursprüngliche \mml{BBBauklasseMinimum}: Aus \href{run:./files/200526_BRISE_RIM-IDM_Auszug
    AI-BB.xlsx}{200626\_BRISE\_RIM-IDM\_Auszug\_AI-BB}
}
% \frage{Gilt das nur für Bauklasse 6? Ist es das gleiche wie
%   \merkmalref{BauklasseMinimum} oder \merkmalref{Bauklasse6Minimum}?
%   NUR FUER BAUKLASSE VI}
}

\merkmal{FBOKMinimumWohnungen}{
  \erlaeuterung{Festsetzung der Fußbodenoberkante vom ersten Wohnungsgeschoß.}
  \wert{Eine positive Zahl [m]}
  \beispiel{
    "`Auf der mit BB5 bezeichneten Fläche dürfen Wohnungen erst ab einer Fußbodenoberkante von 5,0 m über dem anschließenden Gelände errichtet werden."'\\(7736\_12)
  }
  \rechtsmaterie{\href{https://www.ris.bka.gv.at/NormDokument.wxe?Abfrage=LrW&Gesetzesnummer=20000006&FassungVom=2020-11-05&Artikel=&Paragraf=5&Anlage=&Uebergangsrecht=}{WBO \S 5/4/x}:\\
  (4) Über die Festsetzungen nach Abs. 2 und 3 hinaus können die
  Bebauungspläne zusätzlich enthalten:\\
  \etc\\
  x)
Bestimmungen über Mindestraumhöhen in Erdgeschoßen; in Geschäftsvierteln einen gegenüber der Bestimmung des § 6 Abs. 10 größeren Abstand der Fußbodenoberkante vom anschließenden Gelände oder der anschließenden Verkehrsfläche oder das Verbot der Errichtung von Wohnungen überhaupt;}
\quelle{Aus \href{run:./files/200526_BRISE_RIM-IDM_Auszug
    AI-BB.xlsx}{200626\_BRISE\_RIM-IDM\_Auszug\_AI-BB}
}
}


\merkmal{GebaeudeHoeheArt}{
  \erlaeuterung{Beschreibt die Art der
    Gebäudehöhe, auf welche Bezug genommen wird. Zur
    Unterscheidung zwischen z.B. "`tatsächlich errichteter"', 
    "`ausgeführter"', "`festgesetzter"' und "`im Bebauungsplan ausgewiesener"' Gebäudehöhe.}
  \wert{Text [Optionen z.B.:  "`tatsächlich errichtet"',
    "`ausgeführt"' und "`festgesetzt"']}
  \beispiel{"`Im gesamten Plangebiet darf bei den zur Errichtung
    gelangenden Gebäuden der höchste Punkt des Daches maximal 4,5 m
    über der tatsächlich errichteten Gebäudehöhe
    liegen."'\\(6963\_17)\smallskip\\
  "`Der höchste Punkt der Dächer darf hierbei nicht höher als 3,5 m
  über der tatsächlich ausgeführten Gebäudehöhe
  liegen."'\\(7774\_23)\smallskip\\
  "`Der höchste Punkt der zur Errichtung gelangenden Dächer darf die
  festgesetzte Gebäudehöhe um höchstens 4,5 m
  überragen."'\\(8159\_10)\smallskip\\
  "`Der höchste Punkt des Daches von Gebäuden darf höchstens 4,5 m über der im Bebauungsplan ausgewiesenen Gebäudehöhe liegen, wobei die Errichtung von nur einem Dachgeschoß je Gebäude zulässig ist."'\\(7327\_7)} 
  \quelle{Aus Kleinworkshop \# 1}
}


% \merkmal{GebaeudehoeheBeschraenkung}{
%   % \erlaeuterung{}
%   \wert{Eine positive Zahl [m]}
%   % \beispiel{}
%   \rechtsmaterie{\href{https://www.ris.bka.gv.at/NormDokument.wxe?Abfrage=LrW&Gesetzesnummer=20000006&FassungVom=2020-11-05&Artikel=&Paragraf=5&Anlage=&Uebergangsrecht=}{WBO \S 5/4/h}:\\
%   (4) Über die Festsetzungen nach Abs. 2 und 3 hinaus können die
%   Bebauungspläne zusätzlich enthalten:\\
%   \etc\\
%   h)
% Bestimmungen über die Gebäudehöhe, im Bauland bei Festsetzung einer Bauklasse nur bis zu deren Grenzen, ferner über die Höhe von sonstigen Bauwerken, sowie über die höchstens zulässige Zahl der Geschosse, die zur Gänze oder zu einem Teil über dem anschließenden Gelände liegen;
% }
%   % \notiz{}
%   \quelle{Aus WienBV\_Bebauungsbestimmungen in \href{run:./files/200929_BRISE_IDM-REM-LOI.xlsx}{200929\_BRISE\_IDM REM LOI}
%   }
% \frage{Ist das noch relevant oder nun durch
%   \merkmalref{GebaeudeHoeheMax} abgedeckt? GIBT MIR NOCH BESCHEID aber
% wahrscheinlich nur dieses}
% }

\merkmal{GebaeudeHoeheMax}{
  \erlaeuterung{Spezifiziert eine maximale Gebäudehöhe.}
  \wert{Eine positive Zahl [m]}
  \beispiel{"`Die Gebäudehöhe darf 4,5 m nicht
    überschreiten."'\\(7181\_12)\smallskip\\
    "`Auf den mit BB5 bezeichneten Grundflächen darf kein Bauteil der
    zur Errichtung gelangenden Gebäude die Höhe von 33,0 m über
    Wr. Null überschreiten."'\\(7857\_19)\smallskip\\
    "`Für alle Gebäude bis zu einer Gebäudehöhe von 12 m wird
    bestimmt: Flachdächer bis zu einer Dachneigung von 5 Grad sind
    entsprechend dem Stand der technischen Wissenschaften zu
    begrünen."'\\(7245\_6) 
  }
\rechtsmaterie{\href{https://www.ris.bka.gv.at/NormDokument.wxe?Abfrage=LrW&Gesetzesnummer=20000006&FassungVom=2020-11-05&Artikel=&Paragraf=5&Anlage=&Uebergangsrecht=}{WBO \S 5/4/h}:\\
  (4) Über die Festsetzungen nach Abs. 2 und 3 hinaus können die
  Bebauungspläne zusätzlich enthalten:\\
  \etc\\
  h)
Bestimmungen über die Gebäudehöhe, im Bauland bei Festsetzung einer Bauklasse nur bis zu deren Grenzen, ferner über die Höhe von sonstigen Bauwerken, sowie über die höchstens zulässige Zahl der Geschosse, die zur Gänze oder zu einem Teil über dem anschließenden Gelände liegen;
}
  \quelle{Aus den Beispielen sowie
    \href{run:./files/Merkmale_neu_liste-WP4-2020-11-06.xlsx}{Merkmale\_neu\_liste
      WP4 (Version von 2020-11-06)}\\
    Notiz: war vorher \merkmalref{GebaeudehoeheBeschraenkung}...
    }
}


\merkmal{GebaeudeHoeheMin}{
  \erlaeuterung{Spezifiziert die minimale Gebäudehöhe.}
  \wert{Eine positive Zahl [m]}
  \beispiel{"`Bei den zur Errichtung gelangenden Gebäuden ist, ab einer Gebäudehöhe von 16,0m, der oberste Abschluss des Gebäudedaches entsprechend dem Stand der Technik als begrüntes Dach auszubilden, soweit dieses nicht solarenergetisch genutzt wird."'\\(7990\_15)
  }
% \rechtsmaterie{
% }
  \quelle{Aus den Beispielen.
    }
}


\merkmal{HoehenlageGrundflaeche}{
  \erlaeuterung{Bestimmung zur Herstellung einer bestimmten
    Höhenlage der Grundflächen.}
  \beispiel{"`Die mit G/BB 5 bezeichnete GrundFläche ist in einer
    Höhenlage von 74,5 m über Wr. Null
    herzustellen."'\\(7412\_39)\smallskip\\
    "`Für alle an die Baulinie grenzenden, als unbebaute Flächen im
    Bauland ausgewiesenen Bereiche wird die Herstellung einer
    Höhenlage im Niveau der anschließenden Verkehrsfläche
    angeordnet."'\\(7990\_16)
  }
  \rechtsmaterie{\href{https://www.ris.bka.gv.at/NormDokument.wxe?Abfrage=LrW&Gesetzesnummer=20000006&FassungVom=2020-11-05&Artikel=&Paragraf=5&Anlage=&Uebergangsrecht=}{WBO \S 5/4/o}:\\
  (4) Über die Festsetzungen nach Abs. 2 und 3 hinaus können die
  Bebauungspläne zusätzlich enthalten:\\
  \etc\\
  o)
die Anordnung der Herstellung bestimmter Höhenlagen der Grundflächen;
}
\quelle{War früher \mml{AnschlussGebaeudeAnGelaende}. Aus \href{run:./files/200526_BRISE_RIM-IDM_Auszug
    AI-BB.xlsx}{200626\_BRISE\_RIM-IDM\_Auszug\_AI-BB} (für das
  ursprüngliche \mml{AnschlussGebaeudeAnGelaende}).
}
% \frage{Wie genau ist die Erklärung gemeint? Grundstueck muss
%   gewisse hoehenlage haben -> geklaert}
}



\merkmal{MaxHoeheWohngebaeude}{
  \erlaeuterung{Maximale Gebäudehöhe von Kleinhäusern.}
  \wert{Eine positive Zahl [m]}
  % \beispiel{}
  \rechtsmaterie{\href{https://www.ris.bka.gv.at/NormDokument.wxe?Abfrage=LrW&Gesetzesnummer=20000006&FassungVom=2020-11-05&Artikel=&Paragraf=5&Anlage=&Uebergangsrecht=}{WBO
      \S 5/4/u}:\\
  (4) Über die Festsetzungen nach Abs. 2 und 3 hinaus können die
  Bebauungspläne zusätzlich enthalten:\\
  \etc\\
  u)
Gebiete, die der Errichtung von Wohngebäuden mit einer Gebäudehöhe von höchstens 7,50 m, die nicht mehr als zwei Wohnungen enthalten und bei denen für Betriebs- oder Geschäftszwecke höchstens ein Geschoß in Anspruch genommen wird (Kleinhäuser) und Reihenhäusern vorbehalten bleiben;
}
\notiz{Siehe auch Merkmal \merkmalref{Kleinhaeuser}.}
\quelle{Aus \href{run:./files/200526_BRISE_RIM-IDM_Auszug
    AI-BB.xlsx}{200626\_BRISE\_RIM-IDM\_Auszug\_AI-BB}
}
% \frage{Gilt das Merkmal nur für Kleinhäuser oder für
%   Wohnhäuser allgemein? NUR FUER kleinhaeuser, also ref \merkmalref{Kleinhaus}}
}

\merkmal{MindestraumhoeheEG}{
  \erlaeuterung{Festlegung einer Mindestraumhöhe im Erdgeschoß in Geschäftsviertel.}
  \wert{Eine positive Zahl [m]}
  % \beispiel{}
  \rechtsmaterie{\href{https://www.ris.bka.gv.at/NormDokument.wxe?Abfrage=LrW&Gesetzesnummer=20000006&FassungVom=2020-11-05&Artikel=&Paragraf=5&Anlage=&Uebergangsrecht=}{WBO \S 5/4/x}:\\
  (4) Über die Festsetzungen nach Abs. 2 und 3 hinaus können die
  Bebauungspläne zusätzlich enthalten:\\
  \etc\\
  x)
Bestimmungen über Mindestraumhöhen in Erdgeschoßen; in Geschäftsvierteln einen gegenüber der Bestimmung des § 6 Abs. 10 größeren Abstand der Fußbodenoberkante vom anschließenden Gelände oder der anschließenden Verkehrsfläche oder das Verbot der Errichtung von Wohnungen überhaupt;
}
\notiz{Bezieht sich nur auf Geschäftsviertel. % Siehe auch \merkmalref{Geschaeftsviertel}
}
\quelle{Aus \href{run:./files/200526_BRISE_RIM-IDM_Auszug
    AI-BB.xlsx}{200626\_BRISE\_RIM-IDM\_Auszug\_AI-BB}
}
% \frage{Gilt das nur in Geschäftsvierteln? JA  referenz von geschaeftsviertel}
}

% \merkmal{GebaeudehoeheNachPar81}{
%   \erlaeuterung{}
%   \wert{}
%   \beispiel{}
%   \rechtsmaterie{}
% }

% \merkmal{HoeheUeberNiveauVerkehrsflaecheMin}{
%   \erlaeuterung{}
%   \wert{}
%   \beispiel{}
%   \rechtsmaterie{}
% }


\subsection{Kategorie: Lage, Gelände und Planzeichen}
\label{sec:kateg-lage-gelande}

\renewcommand{\category}{Lage, Gelände und Planzeichen}

Enthält Merkmale, welche die geometrische Lage, das Gelände
oder die Planzeichen betreffen.

\merkmal{AnFluchtlinie}{
  \erlaeuterung{Wahr, wenn eine Regelung auf eine Fluchtlinie bezogen ist. Die Fluchtlinie kann unter anderem eine Baulinie, eine Baufluchtlinie oder eine Straßenfluchtlinie sein.}
  \wert{Wahrheitswert}
  \beispiel{"`Für die mit BB10 bezeichnete Fluchtlinie wird bestimmt:
    Die Errichtung einer Lärmschutzeinrichtung mit einer Höhe von bis
    zu 16 m gemessen vom Niveau der angrenzenden Grundfläche ist
    zulässig;"'\\(8159\_27)\smallskip\\
  "`Für die Querschnitte der Verkehrsflächen gemäß § 5 Abs. 2 lit. c der BO für Wien wird bestimmt, dass bei einer Straßenbreite ab 11,0 m entlang der Fluchtlinien Gehsteige mit mindestens 2,0 m Breite herzustellen sind."'\\(7774\_4)}\smallskip\\
	"`Entlang der dem Epk zugeordneten Straßenfluchtlinien ist die Errichtung von Einfriedungen untersagt."'\\(8063\_16)\smallskip\\
	"`Die Errichtung von Staffelgeschoßen an den zu den
	Baulinien orientierten Schauseiten der Gebäude ist untersagt.`"\\(7101\_13)
  % \rechtsmaterie{}
  \quelle{Aus Kleinworkshop \# 1, sowie Annotations Kleinworkshop 2021-03-05 (Gruppe 3)}
}

\merkmal{AnOeffentlichenVerkehrsflaechen}{
  \erlaeuterung{Wahr, wenn eine Regelung auf die Lage an öffentlichen Verkehrsflächen bezogen ist.}
  \wert{Wahrheitswert}
  \beispiel{"`Im gesamten Plangebiet ist an allen öffentlichen Verkehrsflächen die Errichtung von Erkern, Balkonen und vorragenden Loggien untersagt."'\\(6963\_10)\smallskip\\
  "`Auf den mit StrG bezeichneten Grundflächen ist die Errichtung von Einfriedungen an den zu Verkehrsflächen gewandten Bauplatzgrenzen untersagt.`"\\(7857\_24\_0)}
  \quelle{Aus Kleinworkshop \# 1}
}

\merkmal{GelaendeneigungMin}{
  \erlaeuterung{Spezifiziert eine minimale Geländeneigung.}
  \wert{Eine positive Zahl [\%]}
  \beispiel{"`Bei einer Geländeneigung von mehr als 10 \% darf die Gebäudehöhe 3,5 m überschreiten, jedoch an keiner Stelle mehr als 5,0 m betragen."'\\(7443\_14)}
  % \rechtsmaterie{}
  \quelle{Aus den Beispielen sowie
    \href{run:./files/Merkmale_neu_liste-WP4-2020-11-06.xlsx}{Merkmale\_neu\_liste
      WP4 (Version von 2020-11-06)}
    }
}

\merkmal{InSchutzzone}{
  \erlaeuterung{Wahr wenn eine Regelung auf eine Schutzzone bezogen ist.}
  \wert{Wahrheitswert}
  \beispiel{"`Die Dachneigung der innerhalb der Schutzzone zur Errichtung gelangenden Gebäuden hat nicht weniger als 25 Grad und nicht mehr als 55 Grad zu betragen."'\\(7181\_7)}
  % \rechtsmaterie{}
  % \notiz{Wahrscheinlich durch Planzeichen oder Widmung schon abgedeckt}
  \quelle{Aus den Beispielen sowie
    \href{run:./files/Merkmale_neu_liste-WP4-2020-11-06.xlsx}{Merkmale\_neu\_liste
      WP4 (Version von 2020-11-06)}
    }
}

\merkmal{PlangebietAllgemein}{
  \erlaeuterung{Spezifiziert, daß eine Regelung auf dem gesamten
    Plangebiet gilt.}
  \wert{Wahrheitswert}
  \beispiel{"`auf dem gesamten Plangebiet"';\\ (7181\_3)}
  % \rechtsmaterie{}
  \quelle{Aus den Beispielen sowie
    \href{run:./files/Merkmale_neu_liste-WP4-2020-11-06.xlsx}{Merkmale\_neu\_liste
      WP4 (Version von 2020-11-06)}
    }
}

\merkmal{Planzeichen}{
  \erlaeuterung{Spezifiziert ein Zeichen auf dem Plan, beispielsweise "`BB1"', "`Esp"', "`gk"', "`GBGV"', "`öDg"', "`StrE 1"' oder "`durch die Punkte a und b definierten Strecke"''.}
  \wert{Text}
  \beispiel{"`Auf den mit BB1 bezeichneten Flächen ist die
    Unterbrechung der geschlossenen Bauweise
    zulässig."'\\(7181\_9) \\
    "`Der höchste Punkt der zur Errichtung gelangenden Dächer darf mit
    Ausnahme der mit BB2 bezeichneten Bereiche nicht mehr als 3,5 m
    über der festgesetzten Gebäudehöhe liegen."'\\
    (7792\_10)\smallskip\\
    "`Auf den mit L BB3 bezeichneten Flächen ist die Errichtung von
    Wohngebäuden untersagt."'\\(7181\_11)\smallskip\\
    "`Die Dächer der auf den mit der Festsetzung WGV I 5 m g bzw. GBGV
    I 5 m g bezeichneten Grundflächen zur Errichtung gelangenden
    Gebäude sind entsprechend dem Stand der Technik als begrünte
    Flachdächer auszubilden, sofern es sich nicht um Glasdächer
    handelt."'\\(7408\_11)\smallskip\\
    "`Auf der mit öDg bezeichneten \etc\ % und als Grünland/ Erholungsgebiet/
    % Sportund Spielplätze gewidmeten
    Grundfläche ist im Niveau des
    anschließenden Geländes ein öffentlicher Durchgang mit einer
    Breite von 3,0 m freizuhalten und zu dulden."'\\(7531\_19) \\
    "`Die mit StrE 1 bezeichneten Flächen bilden zusammen eine
    Struktureinheit und dürfen unmittelbar bebaut bzw. überbaut
    werden."'\\(7545\_20)\smallskip\\
    "`Von der durch die Punkte a und b definierten Strecke ist bis zu der durch die Punkte c und d definierten Strecke im Niveau der anschließenden Verkehrsfläche ein Raum mit einer lichten Höhe von mindestens 4,0 m und einer lichten Breite von mindestens 6,0 m zur Errichtung und Duldung eines öffentlichen Durchganges von jeder Bebauung freizuhalten."'\\(7990\_32\_0)\smallskip\\
    "`Für alle Flächen, für die gärtnerische Ausgestaltung (G) vorgeschrieben ist, sind bei unterirdischen Bauten Vorkehrungen zu treffen, dass für das Pflanzen von Bäumen ausreichende Erdkerne vorhanden bleiben."'\\(6492\_10\_0)
  }
  % \rechtsmaterie{}
  \notiz{Für Widmungen, welche in natürlicher Sprache gegeben
    sind, siehe \merkmalref{WidmungUndZweckbestimmung}.
    Für mehr Details siehe auch die Flächenwidmungsseite von Stadt Wien \footnote{\url{https://www.wien.gv.at/stadtentwicklung/flaechenwidmung/planzeigen/zeichen-flaewid.html}.}
  }
  \quelle{Aus den Beispielen sowie
    \href{run:./files/Merkmale_neu_liste-WP4-2020-11-06.xlsx}{Merkmale\_neu\_liste
      WP4 (Version von 2020-11-06)}. Note: War früher \mml{PlanzeichenBBID}.
    }
}

\merkmal{Struktureinheit}{
  \erlaeuterung{Gibt an, ob eine Fläche einer Struktureinheit
    zugehörig ist.}
  \wert{Wahrheitswert}
  \beispiel{"`Die mit StrE 1 bezeichneten Flächen bilden zusammen eine
    Struktureinheit und dürfen unmittelbar bebaut bzw. überbaut
    werden"'\\(7545\_20)}
  \quelle{Aus den Beispielen sowie
    \href{run:./files/Merkmale_neu_liste-WP4-2020-11-06.xlsx}{Merkmale\_neu\_liste
      WP4 (Version von 2020-11-06)}
    }
}

\merkmal{VerkehrsflaecheID}{
  \erlaeuterung{Spezifiziert eine bestimmte Verkehrsfläche, zum Beispiel eine Straße, einen Weg oder einen Platz.}
  \wert{Text}
  \beispiel{"`In den Verkehrsflächen Unter-Laaer Straße und
    Klederinger Straße zwischen ONr. 109 - ONr. 149 sind Vorkehrungen
    zu treffen, dass die Pflanzung einer Baumreihe möglich
    ist."'\\(7181\_3)\smallskip\\
    "`An den zu den Verkehrsflächen der Gablenzgasse, Koppstraße und
    Panikengasse gerichteten Gebäudefronten sind Hauptfenster von
    Wohnungen im Erdgeschoß nicht zulässig."'\\(7408\_7)\smallskip\\
    "`Entlang der Verkehrsfläche parallel zur Bahntrasse (§ 53) ist
    Vorsorge zur Pflanzung bzw. Erhaltung von einer Baumreihe zu
    treffen."'\\(7990\_5)
  }
  % \rechtsmaterie{}
  \quelle{Aus den Beispielen sowie
    \href{run:./files/Merkmale_neu_liste-WP4-2020-11-06.xlsx}{Merkmale\_neu\_liste
      WP4 (Version von 2020-11-06)}
    }
}

% \merkmal{anBaulinie}{
%   \erlaeuterung{}
%   \wert{}
%   \beispiel{}
%   \rechtsmaterie{}
% }

% \merkmal{zwischenPunkten}{
%   \erlaeuterung{}
%   \wert{}
%   \beispiel{}
%   \rechtsmaterie{}
%   \notiz{Ersetzen durch "in Verbindung der Punkte"?}
% }

% \merkmal{inVerbindungFluchtlininen}{
%   \erlaeuterung{}
%   \wert{}
%   \beispiel{}
%   \rechtsmaterie{}
% }








\subsection{Kategorie: Laubengänge, Durchfahrten, Arkaden}
\label{sec:kateg-laub-durchf}

\renewcommand{\category}{Laubengänge, Durchfahrten, Arkaden}

Enthält Merkmale, welche Laubengänge, Durchfahrten,
Durchgänge oder Arkaden betreffen.

\merkmal{ArkadeHoehe}{
  \erlaeuterung{Höhe der Arkade die von Bebauung freizuhalten ist.}
  \wert{Eine positive Zahl [m]}
  \beispiel{"`Auf der mit Ak öDg BB13 bezeichneten Grundfläche wird
    eine Arkade mit einer lichten Höhe von 3,0 m
    angeordnet."'\\(7813\_15)\smallskip\\
    "`Für die mit Ak bezeichnete Grundfläche wird bestimmt: Die Anlage
    einer 4,7 m hohen Arkade im Niveau der angrenzenden öffentlichen
    Verkehrsfläche wird angeordnet."'\\(8038\_15)
  }
  \rechtsmaterie{\href{https://www.ris.bka.gv.at/NormDokument.wxe?Abfrage=LrW&Gesetzesnummer=20000006&FassungVom=2020-11-05&Artikel=&Paragraf=5&Anlage=&Uebergangsrecht=}{WBO \S 5/4/f}: \\
(4) Über die Festsetzungen nach Abs. 2 und 3 hinaus können die
Bebauungspläne zusätzlich enthalten:\\
\etc\\
f)
die Anordnung von Laubengängen, Durchfahrten, Durchgängen oder Arkaden;
}
\quelle{Aus \href{run:./files/200526_BRISE_RIM-IDM_Auszug
    AI-BB.xlsx}{200626\_BRISE\_RIM-IDM\_Auszug\_AI-BB}
}
}

\merkmal{ArkadeLaenge}{
  \erlaeuterung{Länge der Arkade zwischen Punkt A u. B.}
  \wert{Eine positive Zahl}
  % \beispiel{}
  \rechtsmaterie{\href{https://www.ris.bka.gv.at/NormDokument.wxe?Abfrage=LrW&Gesetzesnummer=20000006&FassungVom=2020-11-05&Artikel=&Paragraf=5&Anlage=&Uebergangsrecht=}{WBO \S 5/4/f}: \\
(4) Über die Festsetzungen nach Abs. 2 und 3 hinaus können die
Bebauungspläne zusätzlich enthalten:\\
\etc\\
f)
die Anordnung von Laubengängen, Durchfahrten, Durchgängen oder Arkaden;
}
\quelle{Aus \href{run:./files/200526_BRISE_RIM-IDM_Auszug
    AI-BB.xlsx}{200626\_BRISE\_RIM-IDM\_Auszug\_AI-BB}
}
}

\merkmal{DurchfahrtBreite}{
  \erlaeuterung{Breite der Durchfahrt die von Bebauung freizuhalten ist.}
  \wert{Eine positive Zahl [m]}
  % \beispiel{}
  % \rechtsmaterie{}
  \quelle{Analog zu \merkmalref{DurchgangBreite}.}
}

\merkmal{DurchfahrtHoehe}{
  \erlaeuterung{Höhe der Durchfahrt die von Bebauung freizuhalten ist wenn die Überbauung zulässig ist.}
  \wert{Eine positive Zahl [m]}
  \beispiel{"`Die mit „öDf“ bezeichneten Bereiche sind zur Errichtung und Duldung von öffentlichen Durchfahrten mit einer lichten Höhe von mindestens 3,8 m von jeder Bebauung freizuhalten und die Herstellung und die jederzeitige Benutzung derselben zu dulden."'\\(7356\_27)}
  \rechtsmaterie{\href{https://www.ris.bka.gv.at/NormDokument.wxe?Abfrage=LrW&Gesetzesnummer=20000006&FassungVom=2020-11-05&Artikel=&Paragraf=5&Anlage=&Uebergangsrecht=}{WBO \S 5/4/f}: \\
(4) Über die Festsetzungen nach Abs. 2 und 3 hinaus können die
Bebauungspläne zusätzlich enthalten:\\
\etc\\
f)
die Anordnung von Laubengängen, Durchfahrten, Durchgängen oder Arkaden;
}
\quelle{Aus \href{run:./files/200526_BRISE_RIM-IDM_Auszug
    AI-BB.xlsx}{200626\_BRISE\_RIM-IDM\_Auszug\_AI-BB}
}
}

\merkmal{DurchgangBreite}{
  \erlaeuterung{Breite des Durchganges der von Bebauung freizuhalten ist.}
  \wert{Eine positive Zahl [m]}
  \beispiel{In Verbindung der Punkte A und B ist ein Bereich für die
    Errichtung eines öffentlichen Durchgangs mit einer lichten Breite
    von 3,0 m und einer lichten Höhe von 3,0 m freizuhalten und die
    Benutzung desselben zu dulden.\\ (7272\_14)}
  % \rechtsmaterie{}
  \quelle{Aus den Beispielen (von WP4 als "`nicht relevant"' eingestuft).}
  % \frage{Warum ist das nicht relevant? DRIN LASSEN (normalerweise sind
  % durchgaenge / fahrten als flaeche ausgewiesen -> brauchen nur hoehe}
}

\merkmal{DurchgangHoehe}{
  \erlaeuterung{Höhe des Durchganges der von Bebauung freizuhalten ist falls die Überbauung zulässig ist.}
  \wert{Eine positive Zahl [m]}
  \beispiel{"`Auf der mit BB 5 bezeichneten und als Bauland/
  Wohngebiet, Bauklasse III gewidmeten Grundfläche ist im Niveau der
  anschließenden Verkehrsfläche ein öffentlicher Durchgang mit einer
  lichten Höhe von 3,5 m freizuhalten und zu dulden."'\\
  (7020\_27)
}
  \rechtsmaterie{\href{https://www.ris.bka.gv.at/NormDokument.wxe?Abfrage=LrW&Gesetzesnummer=20000006&FassungVom=2020-11-05&Artikel=&Paragraf=5&Anlage=&Uebergangsrecht=}{WBO \S 5/4/f}: \\
(4) Über die Festsetzungen nach Abs. 2 und 3 hinaus können die
Bebauungspläne zusätzlich enthalten:\\
\etc\\
f)
die Anordnung von Laubengängen, Durchfahrten, Durchgängen oder Arkaden;
}
\notiz{Ausgehend vom Niveau der angrenzenden Verkehrsfläche.
}
\quelle{Aus \href{run:./files/200526_BRISE_RIM-IDM_Auszug
    AI-BB.xlsx}{200626\_BRISE\_RIM-IDM\_Auszug\_AI-BB}
}
\frage{DONE 2021-01-22: Ist das immer im Niveau der anschliessenden
  Verkehrslflaeche, oder kommt es auch anders vor? (Hatten mal Merkmal
"imNiveauDerAnschliessendenVerkehrsflaeche" oder so) => immer vom
niveau der anschliessenden Verkehrsflaeche => Beispiel hinzufuegen}
}

\merkmal{LaubengangHoehe}{
  \erlaeuterung{Höhe des Laubenganges.}
  \wert{Eine positive Zahl [m]}
  % \beispiel{}
  \rechtsmaterie{\href{https://www.ris.bka.gv.at/NormDokument.wxe?Abfrage=LrW&Gesetzesnummer=20000006&FassungVom=2020-11-05&Artikel=&Paragraf=5&Anlage=&Uebergangsrecht=}{WBO \S 5/4/f}: \\
(4) Über die Festsetzungen nach Abs. 2 und 3 hinaus können die
Bebauungspläne zusätzlich enthalten:\\
\etc\\
f)
die Anordnung von Laubengängen, Durchfahrten, Durchgängen oder Arkaden;
}
\quelle{Aus \href{run:./files/200526_BRISE_RIM-IDM_Auszug
    AI-BB.xlsx}{200626\_BRISE\_RIM-IDM\_Auszug\_AI-BB}
}
}

\merkmal{LaubengangLaenge}{
  \erlaeuterung{Länge des Laubenganges zwischen Punkt A u. B.}
  \wert{Eine positive Zahl [m]}
  % \beispiel{}
  \rechtsmaterie{\href{https://www.ris.bka.gv.at/NormDokument.wxe?Abfrage=LrW&Gesetzesnummer=20000006&FassungVom=2020-11-05&Artikel=&Paragraf=5&Anlage=&Uebergangsrecht=}{WBO \S 5/4/f}: \\
(4) Über die Festsetzungen nach Abs. 2 und 3 hinaus können die
Bebauungspläne zusätzlich enthalten:\\
\etc\\
f)
die Anordnung von Laubengängen, Durchfahrten, Durchgängen oder Arkaden;
}
\quelle{Aus \href{run:./files/200526_BRISE_RIM-IDM_Auszug
    AI-BB.xlsx}{200626\_BRISE\_RIM-IDM\_Auszug\_AI-BB}
}
}

\subsection{Kategorie: Nutzung und Widmung}
\label{sec:kateg-nutz-und}

\renewcommand{\category}{Nutzung und Widmung}

Enthält Merkmale, welche die Nutzung oder Widmung betreffen.

\merkmal{AusnahmeVonWohnungenUnzulaessig}{
  \erlaeuterung{Spezifiziert Ausnahmen eines Verbots von Wohnungen,
    z.B. "Wohnungen für den Bedarf der Betriebsleitung"}
  \wert{Text}
  \beispiel{"`Die Errichtung von Wohnungen ist nicht
    zulässig. Ausgenommen ist die Errichtung von Wohnungen für den
    Bedarf der Betriebsleitung und der Betriebsaufsicht."'\\(7545\_23)}
  \quelle{Aus 
    \href{run:./files/Merkmale_neu_liste-WP4-2020-11-06.xlsx}{Merkmale\_neu\_liste
      WP4 (Version von 2020-11-06)}
    }
}

\merkmal{AusnuetzbarkeitWidmungskategorieGefoerderterWohnbau}{
  \erlaeuterung{Flächenmäßige Ausnützbarkeit der Widmungskategorie "Geförderter Wohnbau" bezogen auf die Grundfläche (relativ/prozentual).}
  \wert{Eine positive Zahl [\%]}
  % \beispiel{}
  \rechtsmaterie{\href{https://www.ris.bka.gv.at/NormDokument.wxe?Abfrage=LrW&Gesetzesnummer=20000006&FassungVom=2020-11-05&Artikel=&Paragraf=5&Anlage=&Uebergangsrecht=}{WBO \S 5/4/d,e}: \\
(4) Über die Festsetzungen nach Abs. 2 und 3 hinaus können die
Bebauungspläne zusätzlich enthalten:\\
\etc\\
d)
Bestimmungen über die flächenmäßige beziehungsweise volumenbezogene Ausnützbarkeit der Bauplätze und der Baulose oder von Teilen davon; in Gebieten für geförderten Wohnbau Bestimmungen über den Anteil der Wohnnutzfläche der auf einem Bauplatz geschaffenen Wohnungen und Wohneinheiten in Heimen, die hinsichtlich der Grundkostenangemessenheit dem Wiener Wohnbauförderungs- und Wohnhaussanierungsgesetz – WWFSG 1989 entsprechen müssen;\\
e)
Bestimmungen über die bauliche Ausnützbarkeit von ländlichen Gebieten,
Parkanlagen, Freibädern, Parkschutzgebieten und Grundflächen für
Badehütten, bei Gewässern auch die Ausweisung der von jeder Bebauung
freizuhaltenden Uferzonen; Bestimmungen über die bauliche
Ausnützbarkeit von Sport- und Spielplätzen, bei Sportplätzen auch in
bezug auf Sporthallen, sowie eine höchstens zulässige bebaubare
Fläche, bezogen auf eine durch Grenzlinien bestimmte Grundfläche;
Bestimmungen über die Ausnützbarkeit der Sondernutzungsgebiete
hinsichtlich der Art, des Zweckes, ihres Umfanges und ihrer Abgrenzung
zu Nutzungen anderer Art sowie hinsichtlich der endgültigen Gestaltung
ihrer Oberflächen unter Festsetzung der beabsichtigten Wirkung auf das
örtliche Stadt- bzw. Landschaftsbild nach der endgültigen Widmung der
Widmungskategorie Grünland für die endgültige Nutzung der Grundflächen
durch Bestimmung von Geländehöhen (Überhöhungen und Vertiefungen),
Böschungswinkeln, Bepflanzungen der endgültigen baulichen
Ausnützbarkeit und ähnlichem; die Festsetzung eines Zeitpunktes für
die Herstellung der endgültigen Widmung ist zulässig;
}
\quelle{Aus \href{run:./files/200526_BRISE_RIM-IDM_Auszug
    AI-BB.xlsx}{200626\_BRISE\_RIM-IDM\_Auszug\_AI-BB}
}
}

\merkmal{VerbotAufenthaltsraum}{
  \erlaeuterung{Verbot von Aufenthaltsräumen.}
  \wert{Wahrheitswert}
  % \beispiel{}
  \rechtsmaterie{\href{https://www.ris.bka.gv.at/NormDokument.wxe?Abfrage=LrW&Gesetzesnummer=20000006&FassungVom=2020-11-05&Artikel=&Paragraf=5&Anlage=&Uebergangsrecht=}{WBO \S 5/4/y}:\\
  (4) Über die Festsetzungen nach Abs. 2 und 3 hinaus können die
  Bebauungspläne zusätzlich enthalten:\\
  \etc\\
  y)
das Verbot der Errichtung von Aufenthaltsräumen oberhalb der für die Beurteilung der zulässigen Gebäudehöhe maßgebenden Ebene;
}
\notiz{Vergleiche \merkmalref{VerbotFensterZuOeffentlichenVerkehrsflaechen}
  für das Verbot von Fenstern von Aufenthaltsräumen zu
  öffentlichen Verkehrsflächen.}
\quelle{Aus \href{run:./files/200526_BRISE_RIM-IDM_Auszug
    AI-BB.xlsx}{200626\_BRISE\_RIM-IDM\_Auszug\_AI-BB}
}
% \frage{Gilt das nur in Geschäftsvierteln? NEIN}
}


\merkmal{VerbotBueroGeschaeftsgebaeude}{
  \erlaeuterung{Die Errichtung von Büro oder Geschäftsgebäuden ist unzulässig.}
  \wert{Wahrheitswert}
  % \beispiel{}
  \rechtsmaterie{\href{https://www.ris.bka.gv.at/NormDokument.wxe?Abfrage=LrW&Gesetzesnummer=20000006&FassungVom=2020-11-05&Artikel=&Paragraf=5&Anlage=&Uebergangsrecht=}{WBO \S 5/4/w}:\\
  (4) Über die Festsetzungen nach Abs. 2 und 3 hinaus können die
  Bebauungspläne zusätzlich enthalten:\\
  \etc\\
  w)
die Unzulässigkeit der Errichtung von Büro- und Geschäftsgebäuden, die Beschränkung des Rechtes, Fenster von Aufenthaltsräumen von Wohnungen zu öffentlichen Verkehrsflächen herzustellen, sowie in Wohnzonen die Verpflichtung, nicht weniger als 80 vH der Summe der Nutzflächen der Hauptgeschosse eines Gebäudes, jedoch unter Ausschluss des Erdgeschosses oder jener höchstens zulässigen Zahl von Geschossen, in denen die Nutzung für ein Einkaufszentrum zulässig ist, Wohnzwecken vorzubehalten;
}
\quelle{Für das ursprüngliche
  \mml{UnzulaessigBueroGeschaeftsgebaeude} aus \href{run:./files/200526_BRISE_RIM-IDM_Auszug
    AI-BB.xlsx}{200626\_BRISE\_RIM-IDM\_Auszug\_AI-BB}
}
}


\merkmal{VerbotWohnung}{
  \erlaeuterung{Verbot der Errichtung von Wohnungen.  % in Geschäftsviertel.
  }
  \wert{Wahrheitswert [wahr, wenn die Errichtung von Wohnungen
    verboten ist]}
  \beispiel{"`Die Errichtung von Wohnungen ist nicht
    zulässig."'\\(7545\_23)\smallskip\\
    "`Auf den mit L BB3 bezeichneten Flächen ist die Errichtung von
    Wohngebäuden untersagt."'\\(7181\_11)\smallskip\\
  "`Auf den als Geschäftsviertel ausgewiesenen Flächen ist die Errichtung von Wohnungen unzulässig."'\\(7272\_7)}
  \rechtsmaterie{\href{https://www.ris.bka.gv.at/NormDokument.wxe?Abfrage=LrW&Gesetzesnummer=20000006&FassungVom=2020-11-05&Artikel=&Paragraf=5&Anlage=&Uebergangsrecht=}{WBO \S 5/4/x}:\\
  (4) Über die Festsetzungen nach Abs. 2 und 3 hinaus können die
  Bebauungspläne zusätzlich enthalten:\\
  \etc\\
  x)
Bestimmungen über Mindestraumhöhen in Erdgeschoßen; in Geschäftsvierteln einen gegenüber der Bestimmung des § 6 Abs. 10 größeren Abstand der Fußbodenoberkante vom anschließenden Gelände oder der anschließenden Verkehrsfläche oder das Verbot der Errichtung von Wohnungen überhaupt;
}
\quelle{Aus \href{run:./files/200526_BRISE_RIM-IDM_Auszug
    AI-BB.xlsx}{200626\_BRISE\_RIM-IDM\_Auszug\_AI-BB}
}
}


\merkmal{WidmungInMehrerenEbenen}{
  \erlaeuterung{Sammelmerkmal für Widmungen in mehreren Ebenen.}
  \beispiel{
    "`Für die mit BB 14 bezeichnete Grundfläche werden
    gesonderte Bebauungsbestimmungen für zwei übereinanderliegende
    Räume derart getroffen, daß der Raum ab einer Tiefe von 1,0 m
    unter dem bestehenden Niveau dem Einstellen von Kraftfahrzeugen
    vorbehalten bleibt und der Raum darüber dem Parkschutzgebiet
    zugeordnet wird."'\\
    (7020\_45)\smallskip\\
    "`Für die mit BB 3 bezeichneten Flächen werden gesonderte
    Widmungen für zwei übereinander liegende Räume derart getroffen,
    daß der bis zur Deckenoberkante der unterirdischen Objekte
    reichende Raum dem Bauland/Wohngebiet und der Raum darüber dem
    Grünland/Erholungsgebiet Parkanlage, Grundfläche für öffentliche
    Zwecke zugeordnet wird."'\\(7020\_8)\smallskip\\
    "`Für den mit BB12 bezeichneten Bereich werden gesonderte
    Bestimmungen für zwei übereinanderliegende Räume derart
    festgesetzt, dass der Raum bis zur Brückenkonstruktionsunterkante
    als Schutzgebiet Wald- und Wiesengürtel und der Raum darüber als
    Verkehrsband ausgewiesen wird."'\\(7443\_22)
  }
  \notiz{Für Widmungen in nur einer Ebene, siehe
    \merkmalref{WidmungUndZweckbestimmung}. Für durch Planzeichen
    gegebene Widmungen, siehe \merkmalref{Planzeichen}.
    Fasst etliche vorherige Merkmale bezüglich Widmungen in
    mehreren Ebenen zusammen, siehe Anhang~\ref{changes:2021-04-12}
    für die Details.
  }
  \quelle{Diskussion 2021-04-12}
}


\merkmal{WidmungUndZweckbestimmung}{
  \erlaeuterung{Sammelmerkmal für die Widmung und die Zweckbestimmung
    innerhalb einer Widmungskategorie}
  \beispiel{
    "`Auf den als ländliches Gebiet (L) gewidmeten
    Flächen ist die Errichtung von Glashäusern
    unzulässig."'\\(7181\_8)\smallskip\\
    "`Auf den mit BB2 bezeichneten Teilen des Wohn- oder gemischten
    Baugebietes darf die bebaute Fläche maximal 20\% der Bauplatzgröße
    betragen."'\\(7408\_16)\smallskip\\
    "`In den mit BB5 und als Bauland/Gartensiedlungsgebiet gewidmeten
    Flächen ist eine Dachneigung von maximal 45 Grad
    zulässig."'\\(7443\_12) \\
    "`Auf der mit BB3 bezeichneten Fläche ist die Errichtung
    von Gebäuden für gastronomische Zwecke und für Zwecke der
    Bootsvermietung \etc\ % mit einer maximal bebauten Grundfläche von 550 m² und einer Gebäudehöhe von maximal 4,0 m
    zulässig."'\\(7545\_12)\smallskip\\
    "`Auf den mit BB1 bezeichneten Grundflächen sind die Gebäude der
    Zweckbestimmung Bildungseinrichtung zuzuführen."'\\(8008\_22)\smallskip\\
    "`Auf der mit G BB4 bezeichneten Fläche ist die Errichtung von
    Spielplätzen und eine damit verbundene Befestigung der Grundfläche
    im erforderlichen Ausmaß zulässig [\dots]."'\\(7356\_11)\smallskip\\
    "`Die mit G BB5 bezeichneten Flächen sind der Errichtung von
    baulichen Anlagen unter Niveau zum Einstellen von Kraftfahrzeugen
    vorbehalten."'\\(7356\_12)\smallskip\\
    "`Auf den mit BB2 bezeichneten Flächen ist die Errichtung von
    Baulichkeiten, die der Erholung und Gesundheit der Bevölkerung
    dienen, mit einer Gebäudehöhe von maximal 4,5 m
    zulässig."'\\(7548\_8)\smallskip\\
    "`Für die mit BB5 bezeichneten Grundflächen wird bestimmt: Die
    Gebäude dürfen nur für Einrichtungen der öffentlichen Versorgung
    und Sicherheit verwendet werden."'\\(8038\_23)\smallskip\\
    "`Auf den mit StrG bezeichneten Grundflächen sind Flachdächer auf Gebäuden mit einer Gebäudehöhe von bis zu 12 m zu begrünen oder für Terrassen oder zur Energiegewinnung zu nutzen."' \\ (7857\_23\_0)
  }
  \notiz{Für Widmungen in mehreren Ebenen, siehe
    \merkmalref{WidmungInMehrerenEbenen}. Für durch Planzeichen
    gegebene Widmungen, siehe \merkmalref{Planzeichen}.
    Fasst vorherige Merkmale bezüglich Widmung und Zweckbestimmung
    zusammen, siehe Anhang~\ref{changes:2021-04-12}
    für die Details.
  }
  \quelle{Diskussion 2021-04-12}
}


% \merkmal{WidmungID}{
%   \erlaeuterung{Spezifiziert die Widmung, beispielsweise "L", "GB",
%     "Betriebsbaugebiet", ``Sondergebiet", etc}
%   \wert{Text}
%   \beispiel{"`Auf den als ländliches Gebiet (L) gewidmeten
%     Flächen ist die Errichtung von Glashäusern
%     unzulässig."'\\(7181\_8)\\
%   "`Auf den mit BB2 bezeichneten Teilen des Wohn- oder gemischten
%   Baugebietes darf die bebaute Fläche maximal 20\% der Bauplatzgröße
%   betragen."'\\(7408\_16)\\
%   "`In den mit BB5 und als Bauland/Gartensiedlungsgebiet gewidmeten Flächen ist eine Dachneigung von maximal 45 Grad zulässig."'\\(7443\_12)}
%   \notiz{Für Widmungen in mehreren Ebenen, siehe Merkmale
%     \merkmalref{WidmungErsteEbene}, \merkmalref{WidmungZweiteEbene},
%     etc. Für Zweckbestimmungen innerhalb der Widmung siehe das
%     Merkmal \merkmalref{ZweckbestimmungWidmungskategorie}.}
%   \quelle{Aus den Beispielen sowie
%     \href{run:./files/Merkmale_neu_liste-WP4-2020-11-06.xlsx}{Merkmale\_neu\_liste
%       WP4 (Version von 2020-11-06)}
%     }
%     % \frage{Ist das das gleiche wie
%     %   \merkmalref{ZweckbestimmungWidmungskategorie1}? NEIN}
% }




% \merkmal{WidmungErsteEbene}{
%   \erlaeuterung{Widmungsart der ersten Ebene wenn es mehrere Ebenen gibt.}
%   \wert{Text}
%   % \beispiel{}
%   \rechtsmaterie{\href{https://www.ris.bka.gv.at/NormDokument.wxe?Abfrage=LrW&Gesetzesnummer=20000006&FassungVom=2020-11-05&Artikel=&Paragraf=4&Anlage=&Uebergangsrecht=}{WBO
%       \S 4/3}:\\
%   (3) Die Flächenwidmungspläne können für verschiedene übereinanderliegende Räume desselben Plangebietes gesonderte Widmungen ausweisen.}
% \notiz{Nur für Widmungen in mehreren Ebenen. Für Widmungen mit
%   nur einer Ebene insgesamt, siehe Merkmal \merkmalref{WidmungID}.} 
% \quelle{Aus \href{run:./files/200526_BRISE_RIM-IDM_Auszug
%     AI-BB.xlsx}{200626\_BRISE\_RIM-IDM\_Auszug\_AI-BB}
% }
% % \frage{Schließt diese Merkmal das Merkmal \merkmalref{WidmungID}
% %   bzw. \merkmalref{ZweckbestimmungWidmungskategorie1} aus? JA, HIER
% %   NUR FUER MEHRERE EBENEN}
% }

% \merkmal{WidmungErsteEbeneBezugHoehe}{
%   \erlaeuterung{Höhe ab welcher die erste Widmung erlaubt ist (Untergrenze).}
%   \wert{Eine positive Zahl [m]}
%   % \beispiel{}
%   \rechtsmaterie{\href{https://www.ris.bka.gv.at/NormDokument.wxe?Abfrage=LrW&Gesetzesnummer=20000006&FassungVom=2020-11-05&Artikel=&Paragraf=4&Anlage=&Uebergangsrecht=}{WBO
%       \S 4/3}:\\
%   (3) Die Flächenwidmungspläne können für verschiedene übereinanderliegende Räume desselben Plangebietes gesonderte Widmungen ausweisen.}
% \notiz{Für die Obergrenze, also die Grenze zwischen erster und
%   zweiter Ebene siehe das Merkmal \merkmalref{WidmungZweiteEbeneBezugHoehe}.
% }
% \quelle{Aus \href{run:./files/200526_BRISE_RIM-IDM_Auszug
%     AI-BB.xlsx}{200626\_BRISE\_RIM-IDM\_Auszug\_AI-BB}
% }
% \frage{DONE Neu: Ist das die Ober- oder Untergrenze? (Haben immer nur die
%   Obergrenze gesehen) => ist die Untergrenze => in Bemerkungen
%   reinschreiben mit verweis auf zweite ebene}
% }

% \merkmal{WidmungErsteEbeneBezugObjekt}{
%   \erlaeuterung{Gibt ein Objekt an, ab welchem die erste Widmung
%     erlaubt ist (Untergrenze).}
%   \wert{Text}
%   % \beispiel{"`Für die mit BB 3 bezeichneten Flächen werden gesonderte Widmungen für zwei übereinander liegende Räume derart getroffen, daß der bis zur Deckenoberkante der unterirdischen Objekte reichende Raum dem Bauland/Wohngebiet und der Raum darüber dem Grünland/Erholungsgebiet Parkanlage, Grundfläche für öffentliche Zwecke zugeordnet wird."'\\(7020\_8)}
%   % \rechtsmaterie{\href{https://www.ris.bka.gv.at/NormDokument.wxe?Abfrage=LrW&Gesetzesnummer=20000006&FassungVom=2020-11-05&Artikel=&Paragraf=4&Anlage=&Uebergangsrecht=}{WBO
%   %     \S 4/3}:\\
%   % (3) Die Flächenwidmungspläne können für verschiedene übereinanderliegende Räume desselben Plangebietes gesonderte Widmungen ausweisen.}
%   \notiz{Für die Obergrenze, also die Grenze zwischen erster und
%   zweiter Ebene siehe das Merkmal \merkmalref{WidmungZweiteEbeneBezugObjekt}.
%   }
% \quelle{Aus Annotation Kleinworkshop 2020-11-27
% }
% \frage{DONE Neu: Vorschlag Merkmal => passt}
% }

% \merkmal{WidmungZweiteEbene}{
%   \erlaeuterung{Widmungsart der zweiten Ebene.}
%   \wert{Text}
%   % \beispiel{}
%   \rechtsmaterie{\href{https://www.ris.bka.gv.at/NormDokument.wxe?Abfrage=LrW&Gesetzesnummer=20000006&FassungVom=2020-11-05&Artikel=&Paragraf=4&Anlage=&Uebergangsrecht=}{WBO
%       \S 4/3}:\\
%   (3) Die Flächenwidmungspläne können für verschiedene übereinanderliegende Räume desselben Plangebietes gesonderte Widmungen ausweisen.}
% \notiz{Nur für Widmungen in mehreren Ebenen. Für Widmungen mit
%   nur einer Ebene insgesamt, siehe Merkmal \merkmalref{WidmungID}.} 
% \quelle{Aus \href{run:./files/200526_BRISE_RIM-IDM_Auszug
%     AI-BB.xlsx}{200626\_BRISE\_RIM-IDM\_Auszug\_AI-BB}
% }
% }

% \merkmal{WidmungZweiteEbeneBezugHoehe}{
%   \erlaeuterung{Höhe ab welcher die zweite Widmung erlaubt ist (Untergrenze).}
%   \wert{Eine positive Zahl [m]}
%   \beispiel{"`Für die mit BB 14 bezeichnete Grundfläche werden
%   gesonderte Bebauungsbestimmungen für zwei übereinanderliegende
%   Räume derart getroffen, daß der Raum ab einer Tiefe von 1,0 m
%   unter dem bestehenden Niveau dem Einstellen von Kraftfahrzeugen
%   vorbehalten bleibt und der Raum darüber dem Parkschutzgebiet
%   zugeordnet wird."'\\
%   (7020\_45)
% }
%   \rechtsmaterie{\href{https://www.ris.bka.gv.at/NormDokument.wxe?Abfrage=LrW&Gesetzesnummer=20000006&FassungVom=2020-11-05&Artikel=&Paragraf=4&Anlage=&Uebergangsrecht=}{WBO
%       \S 4/3}:\\
%   (3) Die Flächenwidmungspläne können für verschiedene übereinanderliegende Räume desselben Plangebietes gesonderte Widmungen ausweisen.}
% \quelle{Aus \href{run:./files/200526_BRISE_RIM-IDM_Auszug
%     AI-BB.xlsx}{200626\_BRISE\_RIM-IDM\_Auszug\_AI-BB}
% }
% }

% \merkmal{WidmungZweiteEbeneBezugObjekt}{
%   \erlaeuterung{Gibt ein Objekt an, ab welchem die zweite Widmung
%     erlaubt ist (Untergrenze).}
%   \wert{Text}
%   \beispiel{"`Für die mit BB 3 bezeichneten Flächen werden gesonderte Widmungen für zwei übereinander liegende Räume derart getroffen, daß der bis zur Deckenoberkante der unterirdischen Objekte reichende Raum dem Bauland/Wohngebiet und der Raum darüber dem Grünland/Erholungsgebiet Parkanlage, Grundfläche für öffentliche Zwecke zugeordnet wird."'\\(7020\_8)}
%   % \rechtsmaterie{\href{https://www.ris.bka.gv.at/NormDokument.wxe?Abfrage=LrW&Gesetzesnummer=20000006&FassungVom=2020-11-05&Artikel=&Paragraf=4&Anlage=&Uebergangsrecht=}{WBO
%   %     \S 4/3}:\\
%   % (3) Die Flächenwidmungspläne können für verschiedene übereinanderliegende Räume desselben Plangebietes gesonderte Widmungen ausweisen.}
% \quelle{Aus Annotation Kleinworkshop 2020-11-27
% }
% }

% \merkmal{WidmungDritteEbene}{
%   \erlaeuterung{Widmungsart der dritten Ebene}
%   \wert{Text}
%   % \beispiel{}
%   \rechtsmaterie{\href{https://www.ris.bka.gv.at/NormDokument.wxe?Abfrage=LrW&Gesetzesnummer=20000006&FassungVom=2020-11-05&Artikel=&Paragraf=4&Anlage=&Uebergangsrecht=}{WBO
%       \S 4/3}:\\
%   (3) Die Flächenwidmungspläne können für verschiedene übereinanderliegende Räume desselben Plangebietes gesonderte Widmungen ausweisen.}
% \notiz{Nur für Widmungen in mehreren Ebenen. Für Widmungen mit
%   nur einer Ebene insgesamt, siehe Merkmal \merkmalref{WidmungID}.} 
% \quelle{Analog zu \merkmalref{WidmungErsteEbene}.}
% }

% \merkmal{WidmungDritteEbeneBezugHoehe}{
%   \erlaeuterung{Höhe ab welcher die dritte Widmung erlaubt ist.}
%   \wert{Eine positive Zahl [m]}
%   % \beispiel{}
%   \rechtsmaterie{\href{https://www.ris.bka.gv.at/NormDokument.wxe?Abfrage=LrW&Gesetzesnummer=20000006&FassungVom=2020-11-05&Artikel=&Paragraf=4&Anlage=&Uebergangsrecht=}{WBO
%       \S 4/3}:\\
%   (3) Die Flächenwidmungspläne können für verschiedene übereinanderliegende Räume desselben Plangebietes gesonderte Widmungen ausweisen.}
% \quelle{Analog zu \merkmalref{WidmungErsteEbeneBezugHoehe}.}
% }

% \merkmal{WidmungDritteEbeneBezugObjekt}{
%   \erlaeuterung{Gibt ein Objekt an, ab welchem die dritte Widmung erlaubt ist.}
%   \wert{Text}
%   % \beispiel{"`Für die mit BB 3 bezeichneten Flächen werden gesonderte Widmungen für zwei übereinander liegende Räume derart getroffen, daß der bis zur Deckenoberkante der unterirdischen Objekte reichende Raum dem Bauland/Wohngebiet und der Raum darüber dem Grünland/Erholungsgebiet Parkanlage, Grundfläche für öffentliche Zwecke zugeordnet wird."'\\(7020\_8)}
%   % \rechtsmaterie{\href{https://www.ris.bka.gv.at/NormDokument.wxe?Abfrage=LrW&Gesetzesnummer=20000006&FassungVom=2020-11-05&Artikel=&Paragraf=4&Anlage=&Uebergangsrecht=}{WBO
%   %     \S 4/3}:\\
%   % (3) Die Flächenwidmungspläne können für verschiedene übereinanderliegende Räume desselben Plangebietes gesonderte Widmungen ausweisen.}
% \quelle{Analog zu \merkmalref{WidmungErsteEbeneBezugObjekt}.
% }
% }

% \merkmal{ZweckbestimmungWidmungskategorie}{
%   \erlaeuterung{Festlegung des Zweckes der Gebäude oder Baulichkeiten.}
%   \wert{Text}
%   \beispiel{"`Auf der mit BB3 bezeichneten Fläche ist die Errichtung
%     von Gebäuden für gastronomische Zwecke und für Zwecke der
%     Bootsvermietung \etc\ % mit einer maximal bebauten Grundfläche von 550 m² und einer Gebäudehöhe von maximal 4,0 m
%     zulässig."'\\(7545\_12)\\
%   "`Auf den mit BB1 bezeichneten Grundflächen sind die Gebäude der
%   Zweckbestimmung Bildungseinrichtung zuzuführen."'\\(8008\_22)\\
%   "`Auf der mit G BB4 bezeichneten Fläche ist die Errichtung von
%   Spielplätzen und eine damit verbundene Befestigung der Grundfläche
%   im erforderlichen Ausmaß zulässig [\dots]."'\\(7356\_11)\\
%   "`Die mit G BB5 bezeichneten Flächen sind der Errichtung von
%   baulichen Anlagen unter Niveau zum Einstellen von Kraftfahrzeugen
%   vorbehalten."'\\(7356\_12)\\
%   "`Auf den mit BB2 bezeichneten Flächen ist die Errichtung von
%   Baulichkeiten, die der Erholung und Gesundheit der Bevölkerung
%   dienen, mit einer Gebäudehöhe von maximal 4,5 m
%   zulässig."'\\(7548\_8)\\
%   "`Für die mit BB5 bezeichneten Grundflächen wird bestimmt: Die Gebäude dürfen nur für Einrichtungen der öffentlichen Versorgung und Sicherheit verwendet werden."'\\(8038\_23)
% }
%   \rechtsmaterie{\href{https://www.ris.bka.gv.at/NormDokument.wxe?Abfrage=LrW&Gesetzesnummer=20000006&FassungVom=2020-11-05&Artikel=&Paragraf=77&Anlage=&Uebergangsrecht=}{WBO
%       \S 77/4/c}:\\
%   (4) Über die Festsetzung nach Abs. 2 und 3 hinaus können die Bebauungspläne für Strukturen zusätzlich enthalten:\\
% a)
% weitere Bestimmungen über die Gebäudehöhe und den obersten Abschluss des Daches;\\
% b)
% verschiedene Widmungen der Grundflächen auf dem Bauplatz;\\
% c)
% die Zweckbestimmungen innerhalb der Widmungskategorie, denen die Gebäude zuzuführen sind.
% }
% \notiz{Es können auch mehrere Zweckbestimmungen angegeben
%   sein. Für die Widmung allgemein, siehe \merkmalref{WidmungID},
%   für Widmung in mehreren Ebenen siehe \merkmalref{WidmungErsteEbene}.}
% \quelle{Aus \href{run:./files/200526_BRISE_RIM-IDM_Auszug
%     AI-BB.xlsx}{200626\_BRISE\_RIM-IDM\_Auszug\_AI-BB}
% }
% \frage{DONE NOTE: Haben es geändert von
%   \mml{ZweckbestimmungWidmungskategorie1}, da das leichter zu
%   trainieren ist. Werden dann aber bei den Werten mehrere Vorkommnisse
%   annotieren.}
% }


% \merkmal{Flaechenwidmung}{
%   \erlaeuterung{}
%   \wert{}
%   \beispiel{}
%   \rechtsmaterie{}
% }

% \merkmal{Betriebsbaugebiet}{
%   \erlaeuterung{}
%   \wert{}
%   \beispiel{}
%   \rechtsmaterie{}
% }

% \merkmal{}{
%   \erlaeuterung{}
%   \wert{}
%   \beispiel{}
%   \rechtsmaterie{}
% }






\subsection{Kategorie: Stellplätze, Garagen, Parkgebäude}
\label{sec:kateg-stellpl-garag}

\renewcommand{\category}{Stellplätze, Garagen, Parkgebäude}

Enthält Merkmale, welche Aspekte des Abstellens von
Kraftfahrzeugen betreffen.\medskip


\merkmal{AnlageZumEinstellenVorhanden}{
  \erlaeuterung{Gibt an, ob eine Anlage zum Einstellen von
    Kraftfahrzeugen auf einer Fläche vorhanden ist}
  \wert{Wahrheitswert}
  % \rechtsmaterie{}
  \beispiel{"`Nicht bebaute, aber bebaubare Flächen im Bauland sind,
    soweit Anlagen zum Einstellen von Kraftfahrzeugen darauf nicht
    errichtet werden, gärtnerisch auszugestalten, wobei die Errichtung
    von betrieblich benötigten Rangier- und Zufahrtsflächen zulässig
    ist."'\\ (7792\_12)}
  \quelle{Aus Kleinworkshop 2020-12-04}
  % \frage{DONE Ist das abgedeckt durch
  %   \merkmalref{AusnahmeGaertnerischAuszugestaltende}? => drinlassen}
}


\merkmal{GaragengebaeudeAusfuehrung}{
  \erlaeuterung{Legt fest ob Garagengebäude zulässig sind und in welchem Ausmaß.}
  \wert{Text}
  \beispiel{"`Auf den mit BB4 bezeichneten Flächen ist das Errichten von Anlagen zum Einstellen von Kraftfahrzeugen mit einer maximalen Gebäudehöhe von 5,5 m zulässig, die Dachflächen der Garagen sind dabei als begrünte Flachdächer auszubilden."'\\(7272\_12)}
  \rechtsmaterie{\href{https://www.ris.bka.gv.at/NormDokument.wxe?Abfrage=LrW&Gesetzesnummer=20000052&Artikel=&Paragraf=48&Anlage=&Uebergangsrecht=}{WGG \S 48/2}:\\
  (2) Für räumlich begrenzte Teile des Stadtgebietes kann der Bebauungsplan besondere Anordnungen über das zulässige Ausmaß der Herstellung von Stellplätzen festlegen und dabei den Umfang der Stellplatzverpflichtung gemäß § 50 bis zu 90\% verringern sowie Anordnungen über die Art, in der die Stellplatzverpflichtung zu erfüllen ist, und die Zulässigkeit und das Ausmaß von Garagengebäuden sowie von Stellplätzen im Freien treffen (Stellplatzregulativ).}
\quelle{Aus \href{run:./files/200526_BRISE_RIM-IDM_Auszug
    AI-BB.xlsx}{200626\_BRISE\_RIM-IDM\_Auszug\_AI-BB}
}
}

\merkmal{GebaeudeEinschraenkungP}{
  \erlaeuterung{Einschränkung der Errichtung von Gebäuden für Flächen zum Abstellen von Kraftfahrzeugen.}
  \wert{Text}
  \beispiel{"`Die Errichtung von unterirdischen Anlagen zum Einstellen von Kraftfahrzeugen ist zulässig."'\\(7545\_13)}
  \rechtsmaterie{\href{https://www.ris.bka.gv.at/NormDokument.wxe?Abfrage=LrW&Gesetzesnummer=20000006&FassungVom=2020-11-05&Artikel=&Paragraf=5&Anlage=&Uebergangsrecht=}{WBO
      \S 5/4/t}:\\
  (4) Über die Festsetzungen nach Abs. 2 und 3 hinaus können die
  Bebauungspläne zusätzlich enthalten:\\
  \etc\\
  t)
Grundflächen, die für die Errichtung von Gemeinschaftsanlagen zum Einstellen von Kraftfahrzeugen freizuhalten sind oder der Errichtung von Bauwerken zum Einstellen von Kraftfahrzeugen vorbehalten bleiben; Festsetzungen über die Ausgestaltung der Oberfläche von Anlagen zur Einstellung von Kraftfahrzeugen;
}
\quelle{Aus \href{run:./files/200526_BRISE_RIM-IDM_Auszug
    AI-BB.xlsx}{200626\_BRISE\_RIM-IDM\_Auszug\_AI-BB}
}
}

\merkmal{VerbotStellplaetzeUndParkgebaeude}{
  \erlaeuterung{Verbot der Errichtung von Gebäuden oder Flächen zum Abstellen von Kraftfahrzeugen.}
  % \erlaeuterung{Verbot der Errichtung von Gebäuden für Flächen zum Abstellen von Kraftfahrzeugen.}
  \wert{Text}
  \beispiel{
    "`Die Errichtung von Wohnungen und Stellplätzen ist nicht zulässig."'\\(7857\_13)
  }
  \rechtsmaterie{\href{https://www.ris.bka.gv.at/NormDokument.wxe?Abfrage=LrW&Gesetzesnummer=20000006&FassungVom=2020-11-05&Artikel=&Paragraf=5&Anlage=&Uebergangsrecht=}{WBO
      \S 5/4/t}:\\
  (4) Über die Festsetzungen nach Abs. 2 und 3 hinaus können die
  Bebauungspläne zusätzlich enthalten:\\
  \etc\\
  t)
Grundflächen, die für die Errichtung von Gemeinschaftsanlagen zum Einstellen von Kraftfahrzeugen freizuhalten sind oder der Errichtung von Bauwerken zum Einstellen von Kraftfahrzeugen vorbehalten bleiben; Festsetzungen über die Ausgestaltung der Oberfläche von Anlagen zur Einstellung von Kraftfahrzeugen;}
\quelle{Aus \href{run:./files/200526_BRISE_RIM-IDM_Auszug
    AI-BB.xlsx}{200626\_BRISE\_RIM-IDM\_Auszug\_AI-BB}. Note: War
  früher \mml{GebaeudeVerbotP} (allerdings nur für Gebaeude,
  nicht für Stellplätze allgemein).
}
}

\merkmal{OberflaecheBestimmungP}{
  \erlaeuterung{Bestimmung über die Ausgestaltung der Oberfläche von Flächen zum Abstellen von Kraftfahrzeugen.}
  \wert{Text}
  \beispiel{"`Auf den mit P BB5 bezeichneten und für Anlagen zum Einstellen von Kraftfahrzeugen vorbehaltenen Grundflächen dürfen mindestens 20 v.H. dieser Grundflächen nicht versiegelt werden."'\\(7531\_18)}
  \rechtsmaterie{\href{https://www.ris.bka.gv.at/NormDokument.wxe?Abfrage=LrW&Gesetzesnummer=20000006&FassungVom=2020-11-05&Artikel=&Paragraf=5&Anlage=&Uebergangsrecht=}{WBO
      \S 5/4/t}:\\
  (4) Über die Festsetzungen nach Abs. 2 und 3 hinaus können die
  Bebauungspläne zusätzlich enthalten:\\
  \etc\\
  t)
Grundflächen, die für die Errichtung von Gemeinschaftsanlagen zum Einstellen von Kraftfahrzeugen freizuhalten sind oder der Errichtung von Bauwerken zum Einstellen von Kraftfahrzeugen vorbehalten bleiben; Festsetzungen über die Ausgestaltung der Oberfläche von Anlagen zur Einstellung von Kraftfahrzeugen;}
\quelle{Aus \href{run:./files/200526_BRISE_RIM-IDM_Auszug
    AI-BB.xlsx}{200626\_BRISE\_RIM-IDM\_Auszug\_AI-BB}
}
}

% \merkmal{StellplatzArt}{
%   \erlaeuterung{}
%   \wert{Text}
%   \beispiel{}
%   \rechtsmaterie{}
%   \quelle{BLARGH: Evtl nicht relevant?\\Aus \href{run:./files/200929_BRISE_IDM-REM-LOI.xlsx}{200929\_BRISE\_IDM REM LOI}
%   }
% }

\merkmal{StellplatzImNiveauZulaessig}{
  \erlaeuterung{Wahr wenn das Abstellen von Fahrzeugen im Niveau zulässig ist.}
  \wert{Wahrheitswert}
  \beispiel{"`Auf der mit P BB5 bezeichneten Fläche ist das Abstellen von Kraftfahrzeugen im Niveau zulässig."'\\(7545\_14)}
  % \rechtsmaterie{}
  \quelle{Aus den Beispielen sowie
    \href{run:./files/Merkmale_neu_liste-WP4-2020-11-06.xlsx}{Merkmale\_neu\_liste
      WP4 (Version von 2020-11-06)}
    }
}

\merkmal{StellplatzMax}{
  \erlaeuterung{Festlegung für Großbauvorhaben maximal zulässige herzustellenden Stellplätze je Bauplatz. }
  \wert{Eine positive Zahl [Anzahl]}
  % \beispiel{}
  \rechtsmaterie{\href{https://www.ris.bka.gv.at/NormDokument.wxe?Abfrage=LrW&Gesetzesnummer=20000006&FassungVom=2020-11-05&Artikel=&Paragraf=7b&Anlage=&Uebergangsrecht=}{WBO
      \S 7b/5}:\\
  (5) Im Bebauungsplan kann festgelegt werden, wie groß ein Bauplatz beziehungsweise Trennstück für ein Großbauvorhaben mindestens sein muss und wie viele Stellplätze auf diesem Bauplatz beziehungsweise Trennstück höchstens tatsächlich hergestellt werden dürfen.}
\quelle{Aus \href{run:./files/200526_BRISE_RIM-IDM_Auszug
    AI-BB.xlsx}{200626\_BRISE\_RIM-IDM\_Auszug\_AI-BB}
}
}

\merkmal{StellplatzregulativUmfangMaximumAbsolut}{
  \erlaeuterung{Legt einen Absolut Wert für das Stellplatzregulativ fest.}
  \wert{Eine positive Zahl [Anzahl]}
  % \beispiel{}
  \rechtsmaterie{\href{https://www.ris.bka.gv.at/NormDokument.wxe?Abfrage=LrW&Gesetzesnummer=20000052&Artikel=&Paragraf=48&Anlage=&Uebergangsrecht=}{WGG \S 48/2}:\\
  (2) Für räumlich begrenzte Teile des Stadtgebietes kann der Bebauungsplan besondere Anordnungen über das zulässige Ausmaß der Herstellung von Stellplätzen festlegen und dabei den Umfang der Stellplatzverpflichtung gemäß § 50 bis zu 90\% verringern sowie Anordnungen über die Art, in der die Stellplatzverpflichtung zu erfüllen ist, und die Zulässigkeit und das Ausmaß von Garagengebäuden sowie von Stellplätzen im Freien treffen (Stellplatzregulativ).}
\quelle{Aus \href{run:./files/200526_BRISE_RIM-IDM_Auszug
    AI-BB.xlsx}{200626\_BRISE\_RIM-IDM\_Auszug\_AI-BB}
}
% \frage{}
}

\merkmal{StellplatzregulativUmfangMaximumRelativ}{
  \erlaeuterung{Legt fest in welchen maximalen Ausmaß von der Stellplatzverpflichtung abgewichen werden darf (relativ/prozentual).}
  \wert{Eine positive Zahl [\%]}
  \beispiel{"`Es dürfen insgesamt höchstens 80 v.H. der nach dem Wiener Garagengesetz 2008 erforderlichen Stellplätze hergestellt werden."'\\(8286\_21)}
  \rechtsmaterie{\href{https://www.ris.bka.gv.at/NormDokument.wxe?Abfrage=LrW&Gesetzesnummer=20000052&Artikel=&Paragraf=48&Anlage=&Uebergangsrecht=}{WGG \S 48/2}:\\
  (2) Für räumlich begrenzte Teile des Stadtgebietes kann der Bebauungsplan besondere Anordnungen über das zulässige Ausmaß der Herstellung von Stellplätzen festlegen und dabei den Umfang der Stellplatzverpflichtung gemäß § 50 bis zu 90\% verringern sowie Anordnungen über die Art, in der die Stellplatzverpflichtung zu erfüllen ist, und die Zulässigkeit und das Ausmaß von Garagengebäuden sowie von Stellplätzen im Freien treffen (Stellplatzregulativ).}
\quelle{Aus \href{run:./files/200526_BRISE_RIM-IDM_Auszug
    AI-BB.xlsx}{200626\_BRISE\_RIM-IDM\_Auszug\_AI-BB}
}
}

\merkmal{StellplatzregulativUmfangMinimumRelativ}{
  \erlaeuterung{Legt fest in welchen minimalen Ausmaß von der Stellplatzverpflichtung abgewichen werden darf (relativ/prozentual).}
  \wert{Eine positive Zahl [\%]}
  \beispiel{"`Gemäß § 48 des Wiener Garagengesetzes 2008 wird bestimmt: Auf den mit BB4 bezeichneten Flächen beträgt die Stellplatzverpflichtung 70 v.H. der nach dem Wiener Garagengesetz 2008 erforderlichen Stellplatzanzahl."'\\(8286\_21)}
  \rechtsmaterie{\href{https://www.ris.bka.gv.at/NormDokument.wxe?Abfrage=LrW&Gesetzesnummer=20000052&Artikel=&Paragraf=48&Anlage=&Uebergangsrecht=}{WGG \S 48/2}:\\
  (2) Für räumlich begrenzte Teile des Stadtgebietes kann der
  Bebauungsplan besondere Anordnungen über das zulässige Ausmaß der
  Herstellung von Stellplätzen festlegen und dabei den Umfang der
  Stellplatzverpflichtung gemäß § 50 bis zu 90\% verringern sowie
  Anordnungen über die Art, in der die Stellplatzverpflichtung zu
  erfüllen ist, und die Zulässigkeit und das Ausmaß von
  Garagengebäuden sowie von Stellplätzen im Freien treffen
  (Stellplatzregulativ).}
\quelle{Aus \href{run:./files/200526_BRISE_RIM-IDM_Auszug
    AI-BB.xlsx}{200626\_BRISE\_RIM-IDM\_Auszug\_AI-BB}
}
\frage{Ist das noch relevant oder veraltet? Semantik ist nicht ganz
  klar. ER FRAGT NOCHMAL NACH}
}

\merkmal{StellplatzregulativVorhanden}{
  \erlaeuterung{Ist ein Stellplatzregulativ angegeben}
  \wert{Wahrheitswert}
  % \beispiel{}
  \rechtsmaterie{\href{https://www.ris.bka.gv.at/NormDokument.wxe?Abfrage=LrW&Gesetzesnummer=20000052&Artikel=&Paragraf=48&Anlage=&Uebergangsrecht=}{WGG
      \S 48/2}:\\
  (2) Für räumlich begrenzte Teile des Stadtgebietes kann der Bebauungsplan besondere Anordnungen über das zulässige Ausmaß der Herstellung von Stellplätzen festlegen und dabei den Umfang der Stellplatzverpflichtung gemäß § 50 bis zu 90\% verringern sowie Anordnungen über die Art, in der die Stellplatzverpflichtung zu erfüllen ist, und die Zulässigkeit und das Ausmaß von Garagengebäuden sowie von Stellplätzen im Freien treffen (Stellplatzregulativ).}
\quelle{Aus \href{run:./files/200526_BRISE_RIM-IDM_Auszug
    AI-BB.xlsx}{200626\_BRISE\_RIM-IDM\_Auszug\_AI-BB}
}
}

\merkmal{StellplatzverpflichtungArt}{
  \erlaeuterung{Legt die Art fest wie die Stellplatzverpflichtung zu erfüllen ist.}
  \wert{Text}
  % \beispiel{}
  \rechtsmaterie{\href{https://www.ris.bka.gv.at/NormDokument.wxe?Abfrage=LrW&Gesetzesnummer=20000052&Artikel=&Paragraf=48&Anlage=&Uebergangsrecht=}{WGG \S 48/2}:\\
  (2) Für räumlich begrenzte Teile des Stadtgebietes kann der Bebauungsplan besondere Anordnungen über das zulässige Ausmaß der Herstellung von Stellplätzen festlegen und dabei den Umfang der Stellplatzverpflichtung gemäß § 50 bis zu 90\% verringern sowie Anordnungen über die Art, in der die Stellplatzverpflichtung zu erfüllen ist, und die Zulässigkeit und das Ausmaß von Garagengebäuden sowie von Stellplätzen im Freien treffen (Stellplatzregulativ).}
\quelle{Aus \href{run:./files/200526_BRISE_RIM-IDM_Auszug
    AI-BB.xlsx}{200626\_BRISE\_RIM-IDM\_Auszug\_AI-BB}
}
}

% \merkmal{FlaecheVorbehaltP}{
%   \erlaeuterung{}
%   \wert{}
%   \beispiel{}
%   \rechtsmaterie{}
% }

% \merkmal{TiefgarageZulaessig}{
%   \erlaeuterung{}
%   \wert{}
%   \beispiel{}
%   \rechtsmaterie{}
% }

% \merkmal{}{
%   \erlaeuterung{}
%   \wert{}
%   \beispiel{}
%   \rechtsmaterie{}
% }









\subsection{Kategorie: Strasse und Gehsteige}
\label{sec:kateg-strasse-und}

\renewcommand{\category}{Strasse und Gehsteige}

Enthält Merkmale, welche Strassen und Gehsteige betreffen.

\merkmal{GehsteigbreiteMin}{
  \erlaeuterung{Spezifiziert die Mindestbreite für Gehsteige.}
  \wert{Eine positive Zahl [m]}
  \beispiel{"`Für die Querschnitte der Verkehrsflächen gemäß \S 5 (2) lit. c der BO für Wien wird bestimmt, dass bei einer Straßenbreite ab 10 m entlang der Fluchtlinien Gehsteige mit einer Breite von mindestens 2,0 m herzustellen sind."'\\ (7181\_3)}
  % \rechtsmaterie{}
  \quelle{Aus den Beispielen sowie
    \href{run:./files/Merkmale_neu_liste-WP4-2020-11-06.xlsx}{Merkmale\_neu\_liste
      WP4 (Version von 2020-11-06)}
    }
}

\merkmal{OeffentlicheVerkehrsflaecheBreiteMin}{
  \erlaeuterung{Spezifiziert eine Mindestbreite für
    öffentliche Verkehrsflächen.}
  \wert{Eine positive Zahl [m]}
  \beispiel{"`Für öffentliche Verkehrsflächen ab einer Breite von 9,0 m ist innerhalb des Plangebietes jeweils Vorsorge für die Pflanzung von mindestens einer Baumreihe zu treffen."'\\(7374\_6)}
  % \rechtsmaterie{}
  \notiz{Vergeiche Merkmal \merkmalref{StrassenbreiteMin} % 
    für
    Mindestbreiten von Straßen.}
  \quelle{Aus den Beispielen.
    }
}

\merkmal{StrassenbreiteMax}{
  \erlaeuterung{Spezifiziert eine Höchstbreite für Strassen.}
  \wert{Eine positive Zahl [m]}
  \beispiel{"`Für die Querschnitte der Verkehrsflächen gemäß § 5 (2)
    lit. c der Bauordnung für Wien wird bestimmt, dass bei einer
    Straßenbreite unter 10,0 m insgesamt mindestens 2,0 m Breite
    Gehsteig(e) \etc  % , Seite - 2 - bei einer Straßenbreite von 10,0 m bis unter 16,0 m Gehsteige mit insgesamt mindestens 3,0 m Breite und bei einer Straßenbreite ab 16,0 m Gehsteige mit jeweils mindestens 2,0 m Breite,
    herzustellen sind."'\\(7272\_6)}
  % \rechtsmaterie{}
  \notiz{Vergeiche Merkmal \merkmalref{StrassenbreiteVonBis} % 
    für
    Mindest- und Höchstbreite gleichzeitig.}
  \quelle{Analog zu \merkmalref{StrassenbreiteMin}.}
}


\merkmal{StrassenbreiteMin}{
  \erlaeuterung{Spezifiziert eine Mindestbreite für Strassen.}
  \wert{Eine positive Zahl [m]}
  \beispiel{"`Für die Querschnitte der Verkehrsflächen gemäß \S 5 (2) lit. c der BO für Wien wird bestimmt, dass bei einer Straßenbreite ab 10 m entlang der Fluchtlinien Gehsteige mit einer Breite von mindestens 2,0 m herzustellen sind."'\\(7181\_3)}
  % \rechtsmaterie{}
  \notiz{Vergeiche Merkmal \merkmalref{StrassenbreiteVonBis} % 
    für
    Mindest- und Höchstbreite gleichzeitig. Siehe auch
    \merkmalref{OeffentlicheVerkehrsflaecheBreiteMin} für die
    Mindestbreite von öffentlichen Verkehrsflächen.}
  \quelle{Aus den Beispielen sowie
    \href{run:./files/Merkmale_neu_liste-WP4-2020-11-06.xlsx}{Merkmale\_neu\_liste
      WP4 (Version von 2020-11-06)}
    }
}


\merkmal{StrassenbreiteVonBis}{
  \erlaeuterung{Spezifiziert untere und obere Grenzen für die
    Strassenbreite gleichzeitig.}
  \wert{Text}
  \beispiel{"`Für die Querschnitte der Verkehrsflächen gemäß § 5 (2) lit
    c der BO für Wien wird bestimmt, daß \etc\ bei einer Straßenbreite
    von 10,0 m bis unter 16,0 m entlang der Fluchtlinien Gehsteige mit
    mindestens 1,5 m Breite \etc\ herzustellen sind."'\\(6963\_6)}
  \notiz{Vergleiche die Merkmale \merkmalref{StrassenbreiteMin} und
    \merkmalref{StrassenbreiteMax} für Bestimmungen, welche nur
    die Ober- oder Untergrenze betreffen.}
  \quelle{Aus Kleinworkshop \# 1.}
}


% \merkmal{VorkehrungPflanzungBaumreihe}{
%   \erlaeuterung{}
%   \wert{}
%   \beispiel{}
%   \rechtsmaterie{}
% }

% \merkmal{VorkehrungPflanzungZweiBaumreihen}{
%   \erlaeuterung{}
%   \wert{}
%   \beispiel{}
%   \rechtsmaterie{}
% }

% \merkmal{GehtsteigNiveaugleichMitFahrbahn}{
%   \erlaeuterung{}
%   \wert{}
%   \beispiel{}
%   \rechtsmaterie{}
% }

% \merkmal{VorsorgeErrichtungRadverkehrsanlage}{
%   \erlaeuterung{}
%   \wert{}
%   \beispiel{}
%   \rechtsmaterie{}
% }

% \merkmal{VorsorgeErrichtungBusRadverkehrsanlageMinBreite}{
%   \erlaeuterung{}
%   \wert{}
%   \beispiel{}
%   \rechtsmaterie{}
% }

% \merkmal{min\_measure(Verkehrsflaeche,BreiteGesamt}{
%   \erlaeuterung{}
%   \wert{}
%   \beispiel{}
%   \rechtsmaterie{}
% }

% \merkmal{}{
%   \erlaeuterung{}
%   \wert{}
%   \beispiel{}
%   \rechtsmaterie{}
% }

% \merkmal{}{
%   \erlaeuterung{}
%   \wert{}
%   \beispiel{}
%   \rechtsmaterie{}
% }








\subsection{Kategorie: Volumen}
\label{sec:kategorie:volumen}

\renewcommand{\category}{Volumen}

Enthält Merkmale, welche das Volumen betreffen.


\merkmal{VolumenUndUmbaubarerRaum}{
  \erlaeuterung{Sammelmerkmal für alle Bestimmungen, welche ein Volumen oder den
    umbaubaren Raum betreffen.}
  % \wert{Text}
  \beispiel{
      "`Auf den mit BB7 bezeichneten Flächen darf das Ausmaß des
      umbauten Raumes insgesamt 65.000 $\mathrm{m}^3$ nicht
      überschreiten."'\\(6986e\_9) 
    }
  \notiz{
    Fasst etliche vorherige Merkmale bezüglich Volumen und
    umbaubarem Raum zusammen, siehe Anhang~\ref{changes:2021-04-12}
    für die Details..
  }
  \quelle{Diskussion 2021-04-12}
}


% \merkmal{AusnuetzbarkeitVolumenBaumasse}{
%   \erlaeuterung{Volumenbezogene Ausnützbarkeit bezogen auf die Baumasse (m3).}
%   \wert{Eine positive Zahl [$m^3$]}
%   % \beispiel{}
%   \rechtsmaterie{\href{https://www.ris.bka.gv.at/NormDokument.wxe?Abfrage=LrW&Gesetzesnummer=20000006&FassungVom=2020-11-05&Artikel=&Paragraf=5&Anlage=&Uebergangsrecht=}{WBO
%       \S 5/4/d,e}:\\siehe Merkmal \merkmalref{AusnuetzbarkeitVolumenRelativ} % 
%   }
% \quelle{Aus \href{run:./files/200526_BRISE_RIM-IDM_Auszug
%     AI-BB.xlsx}{200626\_BRISE\_RIM-IDM\_Auszug\_AI-BB}
% }
% }


% \merkmal{AusnuetzbarkeitVolumenBaumasseRelativ}{
%   \erlaeuterung{Volumenbezogene Ausnützbarkeit bezogen auf die Baumasse (relativ/prozentual).}
%   \wert{Eine positive Zahl [\%]}
%   % \beispiel{}
%   \rechtsmaterie{\href{https://www.ris.bka.gv.at/NormDokument.wxe?Abfrage=LrW&Gesetzesnummer=20000006&FassungVom=2020-11-05&Artikel=&Paragraf=5&Anlage=&Uebergangsrecht=}{WBO
%       \S 5/4/d,e}:\\siehe Merkmal \merkmalref{AusnuetzbarkeitVolumenRelativ}% 
%   }
% \quelle{Aus \href{run:./files/200526_BRISE_RIM-IDM_Auszug
%     AI-BB.xlsx}{200626\_BRISE\_RIM-IDM\_Auszug\_AI-BB}
% }
% }


% \merkmal{AusnuetzbarkeitVolumenRelativ}{
%   \erlaeuterung{Volumenbezogene Ausnützbarkeit bezogen auf die Grundfläche (relativ/prozentual).}
%   \wert{Eine positive Zahl [\%]}
%   % \beispiel{}
%   \rechtsmaterie{\href{https://www.ris.bka.gv.at/NormDokument.wxe?Abfrage=LrW&Gesetzesnummer=20000006&FassungVom=2020-11-05&Artikel=&Paragraf=5&Anlage=&Uebergangsrecht=}{WBO
%       \S 5/4/d,e}: \\
% (4) Über die Festsetzungen nach Abs. 2 und 3 hinaus können die
% Bebauungspläne zusätzlich enthalten:\\
% \etc\\
% d)
% Bestimmungen über die flächenmäßige beziehungsweise volumenbezogene Ausnützbarkeit der Bauplätze und der Baulose oder von Teilen davon; in Gebieten für geförderten Wohnbau Bestimmungen über den Anteil der Wohnnutzfläche der auf einem Bauplatz geschaffenen Wohnungen und Wohneinheiten in Heimen, die hinsichtlich der Grundkostenangemessenheit dem Wiener Wohnbauförderungs- und Wohnhaussanierungsgesetz – WWFSG 1989 entsprechen müssen;\\
% e)
% Bestimmungen über die bauliche Ausnützbarkeit von ländlichen Gebieten,
% Parkanlagen, Freibädern, Parkschutzgebieten und Grundflächen für
% Badehütten, bei Gewässern auch die Ausweisung der von jeder Bebauung
% freizuhaltenden Uferzonen; Bestimmungen über die bauliche
% Ausnützbarkeit von Sport- und Spielplätzen, bei Sportplätzen auch in
% bezug auf Sporthallen, sowie eine höchstens zulässige bebaubare
% Fläche, bezogen auf eine durch Grenzlinien bestimmte Grundfläche;
% Bestimmungen über die Ausnützbarkeit der Sondernutzungsgebiete
% hinsichtlich der Art, des Zweckes, ihres Umfanges und ihrer Abgrenzung
% zu Nutzungen anderer Art sowie hinsichtlich der endgültigen Gestaltung
% ihrer Oberflächen unter Festsetzung der beabsichtigten Wirkung auf das
% örtliche Stadt- bzw. Landschaftsbild nach der endgültigen Widmung der
% Widmungskategorie Grünland für die endgültige Nutzung der Grundflächen
% durch Bestimmung von Geländehöhen (Überhöhungen und Vertiefungen),
% Böschungswinkeln, Bepflanzungen der endgültigen baulichen
% Ausnützbarkeit und ähnlichem; die Festsetzung eines Zeitpunktes für
% die Herstellung der endgültigen Widmung ist zulässig;}
% \quelle{Aus \href{run:./files/200526_BRISE_RIM-IDM_Auszug
%     AI-BB.xlsx}{200626\_BRISE\_RIM-IDM\_Auszug\_AI-BB}
% }
% }


% \merkmal{UmbaubarerRaumBauplatzMax}{
%   % \erlaeuterung{Beschränkung des maximal umbaubaren Raumes auf
%   %   einem Bauplatz in Kubikmetern}
%   \erlaeuterung{Beschränkt  das höchste zulässige Ausmaß des umbaubaren Raumes der Bauwerke auf dem Bauplatz.}
%   % \erlaeuterung{Beschränkt im Strukturgebiet das höchste zulässige Ausmaß des umbaubaren Raumes der Bauwerke auf dem Bauplatz.}
%   \wert{Eine positive Zahl [$m^3$].}
%   \beispiel{
%     "`Auf den mit BB7 bezeichneten Flächen darf das Ausmaß des umbauten Raumes insgesamt 65.000 $\mathrm{m}^3$ nicht überschreiten."'\\(6986e\_9)
%   }
%   \rechtsmaterie{
%     \href{https://www.ris.bka.gv.at/NormDokument.wxe?Abfrage=LrW&Gesetzesnummer=20000006&FassungVom=2020-11-05&Artikel=&Paragraf=77&Anlage=&Uebergangsrecht=}{WBO
%       \S 77/3/b}:\\
%   (3) Über jede Struktureinheit hat der Bebauungsplan folgende Festsetzung zu enthalten:\\
%   % a)
%   % welche Teile des Bauplatzes unmittelbar bebaut werden dürfen;
%   \etc \\
%   b)
%   das höchste zulässige Ausmaß des umbaubaren Raumes der
%   Bauwerke auf dem Bauplatz;
%   }
%   \quelle{Aus
%     \href{run:./files/200526_BRISE_RIM-IDM_Auszug_AI-BB.xlsx}{200626\_BRISE\_RIM-IDM\_Auszug\_AI-BB}
%   }
% }


% % \subsubsection{UmbaubarerRaumBauplatzMax}
% % \label{sec:umbaRBM}


% % \paragraph{Erläuterung}
% % \paragraph{Wert} Positive Zahl [$m^3$]
% % \paragraph{Beispiele}
% % \paragraph{Rechtsmaterie}


% \merkmal{UmbaubarerRaumGebaeudeMax}{
%   \erlaeuterung{Ausmaß der Nutzflächen der einzelnen Räume und das Gesamtausmaß der Nutzfläche.}
%   \wert{Eine positive Zahl [$m^2$]}
%   % \beispiel{}
%   \rechtsmaterie{\href{https://www.ris.bka.gv.at/NormDokument.wxe?Abfrage=LrW&Gesetzesnummer=20000006&FassungVom=2020-11-05&Artikel=&Paragraf=64&Anlage=&Uebergangsrecht=}{WBO
%       \S 64/1/c}}:\\
%   \S 64. (1) Die Baupläne haben zu enthalten:\\
%   \etc\\
%   c)
% bei Bauführungen oder Widmungsänderungen, durch die Räume neu geschaffen, aufgelassen, geändert oder umgewidmet werden, das Ausmaß der Nutzflächen der einzelnen Räume und das Gesamtausmaß der Nutzfläche der einzelnen Benützungseinheiten (Wohnungen, Betriebe u. ä.); bei Neu- und Zubauten überdies das Ausmaß des umbauten Raumes der betroffenen Gebäude oder Gebäudeteile;
% \quelle{Aus \href{run:./files/200526_BRISE_RIM-IDM_Auszug
%     AI-BB.xlsx}{200626\_BRISE\_RIM-IDM\_Auszug\_AI-BB}
% }
% \frage{Zwei Charakterisierungen, aber nur ein Wert? Ist das Volumen
%   oder Fläche? Wenn Fläche, sollten wir es
%   \texttt{UmbaubareFlaecheGebaeudeMax} nennen? ER SCHAUT NOCHMAL}
% }


% \merkmal{UmbaubarerRaumGebaeudeteilMax}{
%   \erlaeuterung{Ausmaß der Nutzflächen der einzelnen Räume und das Gesamtausmaß der einzelnen Benützungseinheiten.}
%   \wert{Eine positive Zahl [$m^2$]}
%   % \beispiel{}
%   \rechtsmaterie{\href{https://www.ris.bka.gv.at/NormDokument.wxe?Abfrage=LrW&Gesetzesnummer=20000006&FassungVom=2020-11-05&Artikel=&Paragraf=64&Anlage=&Uebergangsrecht=}{WBO \S 64/1/c}}:\\
%   \S 64. (1) Die Baupläne haben zu enthalten:\\
%   \etc\\
%   c)
% bei Bauführungen oder Widmungsänderungen, durch die Räume neu geschaffen, aufgelassen, geändert oder umgewidmet werden, das Ausmaß der Nutzflächen der einzelnen Räume und das Gesamtausmaß der Nutzfläche der einzelnen Benützungseinheiten (Wohnungen, Betriebe u. ä.); bei Neu- und Zubauten überdies das Ausmaß des umbauten Raumes der betroffenen Gebäude oder Gebäudeteile;
% \quelle{Aus \href{run:./files/200526_BRISE_RIM-IDM_Auszug
%     AI-BB.xlsx}{200626\_BRISE\_RIM-IDM\_Auszug\_AI-BB}
% }
% \frage{Zwei Charakterisierungen, aber nur ein Wert? Ist das Volumen
%   oder Fläche? Wenn Fläche, sollten wir es
%   \texttt{UmbaubareFlaecheGebaeudeteilMax} nennen? ER SCHAUT NOCHMAL}
% }


% \merkmal{WidmungAusnuetzbarkeitVolumen}{
%   % \erlaeuterung{}
%   \wert{Eine positive Zahl [$m^3$]}
%   % \beispiel{}
%   \rechtsmaterie{\href{https://www.ris.bka.gv.at/NormDokument.wxe?Abfrage=LrW&Gesetzesnummer=20000006&FassungVom=2020-11-05&Artikel=&Paragraf=5&Anlage=&Uebergangsrecht=}{WBO \S 5/4/d}: \\
% (4) Über die Festsetzungen nach Abs. 2 und 3 hinaus können die
% Bebauungspläne zusätzlich enthalten:\\
% \etc\\
% d)
% Bestimmungen über die flächenmäßige beziehungsweise volumenbezogene Ausnützbarkeit der Bauplätze und der Baulose oder von Teilen davon; in Gebieten für geförderten Wohnbau Bestimmungen über den Anteil der Wohnnutzfläche der auf einem Bauplatz geschaffenen Wohnungen und Wohneinheiten in Heimen, die hinsichtlich der Grundkostenangemessenheit dem Wiener Wohnbauförderungs- und Wohnhaussanierungsgesetz – WWFSG 1989 entsprechen müssen;}
%   \notiz{Aus WienBV\_Bebauungsbestimmungen}
%   \quelle{Aus \href{run:./files/200929_BRISE_IDM-REM-LOI.xlsx}{200929\_BRISE\_IDM REM LOI}
%   }
% \frage{Erläuterung? Ist das relevant? NEIN}
% }










\subsection{Kategorie: Vorbauten}
\label{sec:kategorie:vorbauten}

\renewcommand{\category}{Vorbauten}

Enthält Merkmale welche Vorbauten betreffen. \medskip


\merkmal{VorbautenBeschraenkung}{
  \erlaeuterung{Auf dem Bauplatz ist das Errichten von Vorbauten eingeschränkt möglich.}
  \wert{Text [Lage des zulässigen Vorbaus]}
  \beispiel{
    "`Erker dürfen die Baulinien nicht überragen."'\\(7327\_5)\smallskip\\
    "`Über die Baufluchtlinien in Vorgärten ragende Erker, Balkone und
    vorragende Loggien müssen einen Abstand von mindestens 3,0 m von
    den Baulinien einhalten."'\\(7327\_5)
  }
  \rechtsmaterie{\href{https://www.ris.bka.gv.at/NormDokument.wxe?Abfrage=LrW&Gesetzesnummer=20000006&FassungVom=2020-11-05&Artikel=&Paragraf=5&Anlage=&Uebergangsrecht=}{WBO \S 5/4/i}:\\
  (4) Über die Festsetzungen nach Abs. 2 und 3 hinaus können die
  Bebauungspläne zusätzlich enthalten:\\
  \etc\\
  i)
die Massengliederung, die Anordnungen oder das Verbot der Staffelung der Baumassen und die Beschränkung oder das Verbot der Herstellung von Vorbauten;}
\notiz{Für allgemeine Beschränkungen zu Vorbauten. Für
  spezifische Beschränkungen zur Ausladung vorstehender
  Bauelemente etc.\ siehe \merkmalref{VorstehendeBauelementeAusladungMax}.
}
\quelle{Aus \href{run:./files/200526_BRISE_RIM-IDM_Auszug
    AI-BB.xlsx}{200626\_BRISE\_RIM-IDM\_Auszug\_AI-BB}
}
  % \frage{DONE Neu: Ist das abgedeckt durch
  %   \merkmalref{VorstehendeBauelementeAusladungMax}? (Aus
  %   Kleinworkshop 2020-12-04) => bezieht sich beides auf die
  %   Massengliederung -> sie waren sich nicht sicher wie die
  %   bescrhaenkungen aussehen koennen -> damals haben sie das
  %   allgemeine merkmal genommen => drinnenlassen und verweis auf
  %   anderes merkmal}
}


\merkmal{VorbautenVerbot}{
  \erlaeuterung{Auf dem Bauplatz ist das Errichten von bestimmten Vorbauten verboten.}
  \wert{Text [Art des verbotenen Vorbaus]}
  \beispiel{"`An allen Baulinien ist die Errichtung von Erkern
    untersagt."'\\(8008\_12)\smallskip\\
    "`An allen Baulinien ist die Errichtung von Erkern, Balkonen und
    vorragenden Loggien untersagt."'\\(7408\_6)\smallskip\\
    "`Über Baufluchtlinien, deren Abstand zur gegenüberliegenden
    Fluchtlinie geringer als die zulässige Gebäudehöhe ist, dürfen
    keine Erker und Loggien vorragen."'\\(8250\_15)
}
  \rechtsmaterie{\href{https://www.ris.bka.gv.at/NormDokument.wxe?Abfrage=LrW&Gesetzesnummer=20000006&FassungVom=2020-11-05&Artikel=&Paragraf=5&Anlage=&Uebergangsrecht=}{WBO \S 5/4/i}:\\
  (4) Über die Festsetzungen nach Abs. 2 und 3 hinaus können die
  Bebauungspläne zusätzlich enthalten:\\
  \etc\\
  i)
die Massengliederung, die Anordnungen oder das Verbot der Staffelung der Baumassen und die Beschränkung oder das Verbot der Herstellung von Vorbauten;
}
\quelle{Aus \href{run:./files/200526_BRISE_RIM-IDM_Auszug
    AI-BB.xlsx}{200626\_BRISE\_RIM-IDM\_Auszug\_AI-BB}
  \notiz{Nicht notwendigerweise ein generelles Verbot von Vorbauten
    (siehe Beispiel).}
}
}


\merkmal{VorstehendeBauelementeAusladungMax}{
  \erlaeuterung{Spezifiziert die maximal erlaubte Ausladung
    vorstehender Bauelemente.}
  \wert{Eine positive Zahl [m]}
  \beispiel{"`Vorstehende Bauelemente, die der Gliederung oder der architektonischen Ausgestaltung dienen, sind bei Straßenbreiten bis 16,0 m bis zu einer Ausladung von 0,6 m, bei Straßenbreiten ab 16,0 m bis zu einer Ausladung von 0,8 m zulässig."'\\(6963\_11)}
  % \rechtsmaterie{}
  \notiz{Für allgemeinere Beschränkungen zu Vorbauten wie
    Erkern etc.\ siehe \merkmalref{VorbautenBeschraenkung}.
  }
  \quelle{Aus Kleinworkshop \# 1.}
  % \frage{DONE Neu: Wie ist die genaue Beziehung zu
  %   \merkmalref{VorbautenBeschraenkung}? Bezieht sich das nur auf
  %   vorstehende Bauelemente, die der architektonischen Gliederung
  %   dienen, oder auch auf Erker, Loggien, etc? Sollen wir das für
  %   WP4 differenzieren? => vorbautenbeschraenkung ist allgemeiner;
  %   referenz auf das; erker etc zu vorbautenbeschraenkung.}
}




% \merkmal{}{
%   \erlaeuterung{}
%   \wert{}
%   \beispiel{}
%   \rechtsmaterie{}
% }





% \section{Beispiele}
% \label{sec:beispiele}


\appendix
\printindex



\section{Ausgemusterte Merkmale und sonstige große Änderungen}
\label{sec:changes}


\subsection{2021-08-30}
\label{changes:2021-08-30}
\begin{itemize}
\item Die Kategorie \mml{Meta} wurde aufgelöst.
\item Das Merkmal \mml{VorkehrungBepflanzungOeffentlicheVerkehrsflaeche} wurde absorbiert in \\
\merkmalref{VorkehrungBepflanzung}.
\item Die folgenden Merkmale wurden zusammengefasst in \merkmalref{AnFluchtlinie}:
\begin{itemize}
	\item \mml{AnBaulinie}
	\item \mml{AnStrassenfluchtlinie}
\end{itemize}
\item Das Merkmal \mml{HochhausUnzulaessigGemaessBB} wird jetzt als falscher Wert von \\ \merkmalref{HochhausZulaessigGemaessBB} represeintiert.
\end{itemize}

\subsection{2021-04-14}
\label{changes:2021-04-14}
\begin{itemize}
\item Die folgenden Merkmale wurden zusammengefasst in \merkmalref{AufbautenZulaessig}:
\begin{itemize}
	\item \mml{TechnischeAufbautenZulaessig}
	\item \mml{TechnischeUndBelichtungsaufbautenZulaessig}
\end{itemize}
\item Das Merkmal \mml{StruktureinheitBebaubar} wurde umbenannt in \merkmalref{Struktureinheit}.
\item Das Merkmal \mml{AnschlussGebaeudeAnGelaende} wurde umbenannt in
\merkmalref{HoehenlageGrundflaeche}.
\item Das Merkmal \mml{VonBebauungFreizuhaltenAusnahme} wurde
absorbiert in\\ \merkmalref{WeitereBestimmungPruefungErforderlich}.
\end{itemize}

\subsection{2021-04-12}
\label{changes:2021-04-12}
\begin{itemize}
  \item Merkmal \mml{GebaeudeVerbotP} ist jetzt
    \merkmalref{VerbotStellplaetzeUndParkgebaeude} mit erweiterter
    Bedeutung (umfasst auch Stellplaetze).
  \item Die folgenden Merkmale wurden zusammengefasst in
    \merkmalref{WidmungUndZweckbestimmung}:
    \begin{itemize}
    \item \mml{WidmungID}
    \item \mml{ZweckbestimmungWidmungskategorie}
    \item \mml{EinkaufszentrumZweck}
    \item \mml{GrossbauvorhabenZweck1}
    \end{itemize}
  \item Die folgenden Merkmale wurden zusammengefasst in
    \merkmalref{WidmungInMehrerenEbenen}:
    \begin{itemize}
    \item \mml{WidmungErsteEbene}
    \item \mml{WidmungErsteEbeneBezugHoehe}
    \item \mml{WidmungErsteEbeneBezugObjekt}
    \item \mml{WidmungZweiteEbene}
    \item \mml{WidmungZweiteEbeneBezugHoehe}
    \item \mml{WidmungZweiteEbeneBezugObjekt}
    \item \mml{WidmungDritteEbene}
    \item \mml{WidmungDritteEbeneBezugHoehe}
    \item \mml{WidmungDritteEbeneBezugObjekt}
    \end{itemize}
  \item Die folgenden Merkmale wurden zusammengefasst in \merkmalref{Flaechen}:
    \begin{itemize}
    \item \mml{BauplatzUnterirdischeBebauungMax}
    \item \mml{BauplatzGroesseMin}
    \item \mml{AusnuetzbarkeitFlaecheBGF}
    \item \mml{AusnuetzbarkeitFlaecheBGFRelativ}
    \item \mml{AusnuetzbarkeitFlaecheFluchtlinienbezugRelativ}
    \item \mml{AusnuetzbarkeitFlaecheGrundflaechenbezug}
    \item \mml{AusnuetzbarkeitFlaecheGrundflaechenbezugRelativ}
    \item \mml{AusnuetzbarkeitFlaecheNutzflaeche}
    \item \mml{AusnuetzbarkeitFlaecheNutzflaecheRelativ}
    \item \mml{AusnuetzbarkeitFlaecheWohnnutzflaeche}
    \item \mml{AusnuetzbarkeitFlaecheWohnnutzflaecheRelativMax}
    \item \mml{AusnuetzbarkeitFlaecheWohnnutzflaecheRelativMin}
    \item \mml{BebaubareFlaecheAbgegrenzt}
    \item \mml{BebaubareFlaecheGesamterBauplatz}
    \item \mml{BebaubareFlaecheJeBauplatz}
    \item \mml{BebaubareFlaecheJeBauplatzteil}
    \item \mml{BebaubareFlaecheJeGebaeude}
    \item \mml{BebauteFlaechefuerNebengebaeudeJeBauplatzMax}
    \item \mml{BebauteFlaechefuerNebengebaeudeJeBaulosMax}
    \item \mml{FlaecheBebaubar}
    \item \mml{FlaecheBebaut}
    \item \mml{EinkaufszentrumMaxFlaeche}
    \item \mml{GrossbauvorhabenMaxFlaeche}
    \end{itemize}
  \item Die folgenden Merkmale wurden zusammengefasst in
    \merkmalref{VolumenUndUmbaubarerRaum}: 
    \begin{itemize}
    \item \mml{AusnuetzbarkeitVolumenBaumasse}
    \item \mml{AusnuetzbarkeitVolumenBaumasseRelativ}
    \item \mml{AusnuetzbarkeitVolumenRelativ}
    \item \mml{UmbaubarerRaumBauplatzMax}
    \item \mml{UmbaubarerRaumGebaeudeMax}
    \item \mml{UmbaubarerRaumGebaeudeteilMax}
    \end{itemize}

\end{itemize}

\subsection{2021-04-07}
\begin{itemize}
\item Merkmal \mml{BBAusnuetzbarkeitFlaecheWohnnutzflaecheRelativ}
  wurde aufgeteilt in
  \begin{itemize}
  \item \merkmalref{AusnuetzbarkeitFlaecheWohnnutzflaecheRelativMax}
    und 
  \item \merkmalref{AusnuetzbarkeitFlaecheWohnnutzflaecheRelativMin}.
  \end{itemize}
\item Merkmal \mml{GebaeudeHoeheBeschraenkung} wurde absorbiert in \merkmalref{GebaeudeHoeheMax}.
\item Merkmale \mml{BBDachneigungMax} und \mml{DachneigungMax} wurden
  zusammengefasst in das Merkmal \merkmalref{DachneigungMax}.
\item Merkmal \mml{BBDachneigungMin} wurde absorbiert in \merkmalref{DachneigungMin}.
\item Merkmal \mml{ZulaessigeGeschossanzahl} wurde umbenannt in\\ \merkmalref{ZulaessigeGeschossanzahlEinkaufszentrum}.
\item Merkmal \mml{MaxAnzahlGeschosseOberirdischDachgeschoss} wurde umbenannt in\\ \merkmalref{MaxAnzahlDachgeschosse}.
\item Merkmal \mml{UnzulaessigkeitUnterirdischeBauwerke} wurde umbenannt in\\ \merkmalref{VerbotUnterirdischeBauwerkeUeberBaufluchtlinie}.
\item Merkmal
  \mml{UnzulaessigkeitFensterZuOeffentlichenVerkehrsflaechen} wurde
  umbenannt in \merkmalref{VerbotFensterZuOeffentlichenVerkehrsflaechen}
\item Merkmal \mml{UnzulaessigBueroGeschaeftsgebaeude} wurde umbenannt
  in\\ \merkmalref{VerbotBueroGeschaeftsgebaeude}
\item Merkmal \mml{BBBauklasseMaximum} wurde umbenannt in \merkmalref{BauklasseVIHoeheMax}
\item Merkmal \mml{BBBauklasseMinimum} wurde umbenannt in
  \merkmalref{BauklasseVIHoeheMin}
\item Merkmale \mml{BBAusnuetzbarkeit\dots} wurden umbenannt in \mml{Ausnuetzbarkeit\dots}
\item Merkmale \mml{BBBebaubareFlaeche\dots} wurden umbenannt in \mml{BebaubareFlaeche\dots}
\end{itemize}

\subsection{2021-03-02}
\begin{itemize}
\item Merkmal \mml{AbschlussDachMax} wurde aufgeteilt in
  \begin{itemize}
  \item \merkmalref{AbschlussDachMaxBezugGebaeude} und
  \item \merkmalref{AbschlussDachMaxBezugGelaende}.
  \end{itemize}
\end{itemize}

\inVersion{long}{

\section{Ausgemusterte Merkmale im Detail}


\subsection{Ausgestaltung und Sonstiges}

\merkmal{TechnischeAufbautenZulaessig}{
  \erlaeuterung{Wahr wenn technisch erforderliche Aufbauten zulässig sind.}
  \wert{Wahrheitswert}
  \beispiel{"`Für 10 \% der mit BB4 bezeichneten Fläche sind
    technische Vorkehrungen für Baumpflanzungen (Erdkerne) zu
    treffen. Technisch erforderliche Aufbauten sind zulässig."'\\(7545\_13)}
  % \rechtsmaterie{}
  \notiz{Vergleiche auch
    \merkmalref{TechnischeUndBelichtungsAufbautenZulaessig} für
    technische und
    Belichtungsaufbauten auf Dächern.}
  \notiz{Absorbiert in \merkmalref{AufbautenZulaessig}}
  \quelle{Aus den Beispielen sowie
    \href{run:./files/Merkmale_neu_liste-WP4-2020-11-06.xlsx}{Merkmale\_neu\_liste
    WP4 (Version von 2020-11-06)}
    }
}

\merkmal{VonBebauungFreizuhaltenAusnahme}{
  \erlaeuterung{Ausnahmen zu: von Bebauung freizuhaltenden Bereichen (öffentlichen Durchfahrten und Durchgängen, Verkehrsbauwerken und öffentlichen Aufschließungsleitungen).}
  \wert{Text}
  \beispiel{
    "`Auf der mit BB6 bezeichneten Fläche wird im Niveau der
    anschließenden Verkehrsfläche ein Durchgang mit einer lichten Höhe
    von 9,0 m angeordnet. Die Anordnung statisch erforderlicher
    Konstruktionselemente ist zulässig."'\\(7990\_22)
  }
  \rechtsmaterie{\href{https://www.ris.bka.gv.at/NormDokument.wxe?Abfrage=LrW&Gesetzesnummer=20000006&FassungVom=2020-11-05&Artikel=&Paragraf=5&Anlage=&Uebergangsrecht=}{WBO \S 5/4/g}: \\
(4) Über die Festsetzungen nach Abs. 2 und 3 hinaus können die
Bebauungspläne zusätzlich enthalten:\\
\etc\\
g)
Grundflächen und Räume, die zur Errichtung und Duldung von öffentlichen Durchfahrten und Durchgängen, Verkehrsbauwerken und öffentlichen Aufschließungsleitungen durch die Gemeinde von jeder Bebauung freizuhalten sind und Bestimmungen über die sich daraus ergebenden Einschränkungen der Bebaubarkeit und Nutzung;
}
\notiz{Siehe Merkmal \merkmalref{VonBebauungFreizuhalten} für das
  Verbot der Bebauung.}
\notiz{Wurde entfernt, da Kontextsensitiv. Stattdessen
  \merkmalref{WeitereBestimmungPruefungErforderlich} oder
  \merkmalref{AusnahmePruefungErforderlich} verwenden.}
\quelle{Aus \href{run:./files/200526_BRISE_RIM-IDM_Auszug
    AI-BB.xlsx}{200626\_BRISE\_RIM-IDM\_Auszug\_AI-BB}
}
% \frage{Ist der Wert ein Text?}
}



\subsection{Dach}

\merkmal{TechnischeUndBelichtungsAufbautenZulaessig}{
  \erlaeuterung{Wahr, wenn technische bzw.\ der Belichtung dienende
    Dachaufbauten im erforderlichen Ausmaß zulässig sind.}
  \wert{Wahrheitswert}
  \beispiel{"`Flachdächer bis zu einer Dachneigung von fünf
    Grad sind entsprechend dem Stand der technischen Wissenschaften zu
    begrünen. Technische bzw. der Belichtung dienende Aufbauten sind im
    erforderlichen Ausmaß zulässig"'\\(7181\_6)}
  % \rechtsmaterie{}
  \notiz{Vergleiche auch Merkmal
    \merkmalref{TechnischeAufbautenZulaessig} für technische
    Aufbauten auf der Grundfläche. Siehe auch Merkmal
    \merkmalref{TechnischeAufbautenHoeheMax} für die
    Beschränkung der Höhe technischer Aufbauten.}
  \notiz{Absorbiert in \merkmalref{AufbautenZulaessig}.}
  \quelle{Aus den Beispielen sowie
    \href{run:./files/Merkmale_neu_liste-WP4-2020-11-06.xlsx}{Merkmale\_neu\_liste
    WP4 (Version von 2020-11-06)}
    }
}




\subsection{Fläche}

\merkmal{BauplatzUnterirdischeBebauungMax}{
  \erlaeuterung{Spezifiziert das maximale Ausmaß der Fläche unterirdischer Bauten/Bauteile relativ zur Fläche des Bauplatzes}
  \wert{Eine positive Zahl [\%]}
  \beispiel{"`Innerhalb der als Bauland gewidmeten und mit G
    bezeichneten Flächen dürfen unterirdische Bauten oder Bauteile nur
    in einem Ausmaß von maximal 20 v. H. des Bauplatzes errichtet
    werden."' \\(7408\_13)}
  % \rechtsmaterie{}
  \notiz{Jetzt absorbiert in \merkmalref{Flaechen}.}
  \quelle{Aus den Beispielen sowie
    \href{run:./files/Merkmale_neu_liste-WP4-2020-11-06.xlsx}{Merkmale\_neu\_liste
    WP4 (Version von 2020-11-06)}
    }
}

\merkmal{BauplatzGroesseMin}{
  \erlaeuterung{Spezifiziert eine Mindestgröße für einen Bauplatz.}
  \wert{Eine positive Zahl [$\mathrm{m}^2$]}
  \beispiel{"`Auf den mit BB 1 bezeichneten Flächen des Wohngebietes
    dürfen bei einer Bauplatzgröße über 300 $m^2$ max. 25\% des Bauplatzes bebaut werden."'\\(7412\_14)}
  \notiz{Jetzt absorbiert in \merkmalref{Flaechen}.}
  \quelle{Annotations Kleinworkshop 2021-03-01.}
  % \notiz{}
}

\merkmal{AusnuetzbarkeitFlaecheBGF}{
  \label{merkmal:BBAusnuetzbarkeitFlaecheBGF}
  \erlaeuterung{Maximale BGF-bezogene Ausnützbarkeit (m2).}
  \wert{Eine positive Zahl [$m^2$]}
  \beispiel{"`Im Bauland/Wohngebiet darf die Bruttogeschoßfläche aller Geschoße, die ganz oder teilweise über dem anschließenden Gelände liegen, insgesamt höchstens 11.500 $\mathrm{m}^2$ betragen."'\\(8025\_15)}
  \rechtsmaterie{\href{https://www.ris.bka.gv.at/NormDokument.wxe?Abfrage=LrW&Gesetzesnummer=20000006&FassungVom=2020-11-05&Artikel=&Paragraf=5&Anlage=&Uebergangsrecht=}{WBO \S 5/4/d,e}: \\
(4) Über die Festsetzungen nach Abs. 2 und 3 hinaus können die
Bebauungspläne zusätzlich enthalten:\\
\etc\\
d)
Bestimmungen über die flächenmäßige beziehungsweise volumenbezogene Ausnützbarkeit der Bauplätze und der Baulose oder von Teilen davon; in Gebieten für geförderten Wohnbau Bestimmungen über den Anteil der Wohnnutzfläche der auf einem Bauplatz geschaffenen Wohnungen und Wohneinheiten in Heimen, die hinsichtlich der Grundkostenangemessenheit dem Wiener Wohnbauförderungs- und Wohnhaussanierungsgesetz – WWFSG 1989 entsprechen müssen;\\
e)
Bestimmungen über die bauliche Ausnützbarkeit von ländlichen Gebieten,
Parkanlagen, Freibädern, Parkschutzgebieten und Grundflächen für
Badehütten, bei Gewässern auch die Ausweisung der von jeder Bebauung
freizuhaltenden Uferzonen; Bestimmungen über die bauliche
Ausnützbarkeit von Sport- und Spielplätzen, bei Sportplätzen auch in
bezug auf Sporthallen, sowie eine höchstens zulässige bebaubare
Fläche, bezogen auf eine durch Grenzlinien bestimmte Grundfläche;
Bestimmungen über die Ausnützbarkeit der Sondernutzungsgebiete
hinsichtlich der Art, des Zweckes, ihres Umfanges und ihrer Abgrenzung
zu Nutzungen anderer Art sowie hinsichtlich der endgültigen Gestaltung
ihrer Oberflächen unter Festsetzung der beabsichtigten Wirkung auf das
örtliche Stadt- bzw. Landschaftsbild nach der endgültigen Widmung der
Widmungskategorie Grünland für die endgültige Nutzung der Grundflächen
durch Bestimmung von Geländehöhen (Überhöhungen und Vertiefungen),
Böschungswinkeln, Bepflanzungen der endgültigen baulichen
Ausnützbarkeit und ähnlichem; die Festsetzung eines Zeitpunktes für
die Herstellung der endgültigen Widmung ist zulässig;
}
  \notiz{Jetzt absorbiert in \merkmalref{Flaechen}.}
\quelle{Aus \href{run:./files/200526_BRISE_RIM-IDM_Auszug
    AI-BB.xlsx}{200626\_BRISE\_RIM-IDM\_Auszug\_AI-BB}
}
}

\merkmal{AusnuetzbarkeitFlaecheBGFRelativ}{
  \erlaeuterung{Maximale BGF-bezogene Ausnützbarkeit (relativ/prozentual).}
  \wert{Eine positive Zahl [\%]}
  % \beispiel{}
  \rechtsmaterie{\href{https://www.ris.bka.gv.at/NormDokument.wxe?Abfrage=LrW&Gesetzesnummer=20000006&FassungVom=2020-11-05&Artikel=&Paragraf=5&Anlage=&Uebergangsrecht=}{WBO
      \S 5/4/d,e}\\siehe Merkmal \merkmalref{AusnuetzbarkeitFlaecheBGF} % 
  }
  \notiz{Jetzt absorbiert in \merkmalref{Flaechen}.}
\quelle{Aus \href{run:./files/200526_BRISE_RIM-IDM_Auszug
    AI-BB.xlsx}{200626\_BRISE\_RIM-IDM\_Auszug\_AI-BB}
}
}


\merkmal{AusnuetzbarkeitFlaecheFluchtlinienbezugRelativ}{
  \erlaeuterung{Maximale flächenmäßige Ausnützbarkeit
    bezogen auf eine von Fluchtlinien bestimmten Fläche.}
  \wert{Eine positive Zahl [\%]}
  \beispiel{"`Die mit BB 15 bezeichnete und als Grünland/
    Erholungsgebiet Sport- und Spielplätze gewidmete Grundfläche darf
    bis zu einem Ausmaß von höchstens 20 v. H. dieser durch
    Fluchtlinien bestimmten Fläche bebaut werden."'\\
    (7020\_38)
  }
  % \rechtsmaterie{}
  \notiz{Jetzt absorbiert in \merkmalref{Flaechen}.}
  \quelle{Kleinworkshop 2021-01-22}
}


\merkmal{AusnuetzbarkeitFlaecheGrundflaechenbezug}{
  \erlaeuterung{Maximale flächenmäßige Ausnützbarkeit bezogen auf die Grundfläche (m2).}
  \wert{Eine positive Zahl [$m^2$]}
  \beispiel{"`Innerhalb der mit BB 13 bezeichneten und als Grünland/
    Parkschutzgebiet gewidmeten Grundfläche dürfen zwei Gebäude mit
    einer gesamten bebauten Fläche von max. 400 $m^2$ und einer
    Gebäudehöhe von max. 9 m errichtet werden."'\\(7020\_35)\smallskip\\
  "`Auf den mit BB 6 bezeichneten und als Erholungsgebiet/ Kleingartengebiet gewidmeten Flächen darf das Ausmaß der bebauten Fläche 25 $m^2$ je Kleingarten nicht überschreiten."'\\(7020\_47)}
  \rechtsmaterie{\href{https://www.ris.bka.gv.at/NormDokument.wxe?Abfrage=LrW&Gesetzesnummer=20000006&FassungVom=2020-11-05&Artikel=&Paragraf=5&Anlage=&Uebergangsrecht=}{WBO
      \S 5/4/d,e}:\\siehe Merkmal \merkmalref{AusnuetzbarkeitFlaecheBGF}% 
  }
  \notiz{Siehe auch \merkmalref{AnordnungGaertnerischeAusgestaltung}
    für den Bezug auf die gärtnerische Ausgestaltung.
  }
  \notiz{Jetzt absorbiert in \merkmalref{Flaechen}.}
\quelle{Aus \href{run:./files/200526_BRISE_RIM-IDM_Auszug
    AI-BB.xlsx}{200626\_BRISE\_RIM-IDM\_Auszug\_AI-BB}
}
\frage{DONE Neu (workshop 2020-12-11): Bezieht sich das auch auf
  gaertnerische Ausgestaltung etc? (siehe 6963\_26: "mindestens X
  prozent gaertnerisch auszugestalten"). Ist das maximal oder minimal?
=> GaertnerischeAusgestaltungProzentual (neues Merkmal) => Verweis auf
das Merkmal; ist die maximale ausnuetzbarkeit! => Ueberall bei
Ausnuetzbarkeit reinschreiben}
}

\merkmal{AusnuetzbarkeitFlaecheGrundflaechenbezugRelativ}{
  \erlaeuterung{Maximale flächenmäßige Ausnützbarkeit bezogen auf die Grundfläche (relativ/prozentual).}
  \wert{Eine positive Zahl [\%]}
  \beispiel{"`Auf der mit Esp BB2 bezeichneten Fläche dürfen nur
    Gebäude und bauliche Anlagen in einem Ausmaß von insgesamt maximal
    10 \% der Grundfläche \etc\ % und einer Gebäudehöhe von maximal 7,5 m
    errichtet werden."'\\(7545\_11)}
  \rechtsmaterie{\href{https://www.ris.bka.gv.at/NormDokument.wxe?Abfrage=LrW&Gesetzesnummer=20000006&FassungVom=2020-11-05&Artikel=&Paragraf=5&Anlage=&Uebergangsrecht=}{WBO
      \S 5/4/d,e}}:\\siehe Merkmal \merkmalref{AusnuetzbarkeitFlaecheBGF}% 
  \notiz{Jetzt absorbiert in \merkmalref{Flaechen}.}
\quelle{Aus \href{run:./files/200526_BRISE_RIM-IDM_Auszug
    AI-BB.xlsx}{200626\_BRISE\_RIM-IDM\_Auszug\_AI-BB}
}
}

\merkmal{AusnuetzbarkeitFlaecheNutzflaeche}{
  \erlaeuterung{Maximale nutzflächenmäßige Ausnützbarkeit bezogen auf die Grundfläche (m2).}
  \wert{Eine positive Zahl [$m^2$]}
  % \beispiel{}
  \rechtsmaterie{\href{https://www.ris.bka.gv.at/NormDokument.wxe?Abfrage=LrW&Gesetzesnummer=20000006&FassungVom=2020-11-05&Artikel=&Paragraf=5&Anlage=&Uebergangsrecht=}{WBO
      \S 5/4/d,e}\\siehe Merkmal
    \merkmalref{AusnuetzbarkeitFlaecheBGF} 
  }
  \notiz{Jetzt absorbiert in \merkmalref{Flaechen}.}
\quelle{Aus \href{run:./files/200526_BRISE_RIM-IDM_Auszug
    AI-BB.xlsx}{200626\_BRISE\_RIM-IDM\_Auszug\_AI-BB}
}
}

\merkmal{AusnuetzbarkeitFlaecheNutzflaecheRelativ}{
  \erlaeuterung{Maximale nutzflächenmäßige Ausnützbarkeit bezogen auf die Grundfläche (relativ/prozentual).}
  \wert{Eine positive Zahl [\%]}
  % \beispiel{}
  \rechtsmaterie{\href{https://www.ris.bka.gv.at/NormDokument.wxe?Abfrage=LrW&Gesetzesnummer=20000006&FassungVom=2020-11-05&Artikel=&Paragraf=5&Anlage=&Uebergangsrecht=}{WBO
      \S 5/4/d,e}\\siehe Merkmal \merkmalref{AusnuetzbarkeitFlaecheBGF}
  }
  \notiz{Jetzt absorbiert in \merkmalref{Flaechen}.}
\quelle{Aus \href{run:./files/200526_BRISE_RIM-IDM_Auszug
    AI-BB.xlsx}{200626\_BRISE\_RIM-IDM\_Auszug\_AI-BB}
}
}

\merkmal{AusnuetzbarkeitFlaecheWohnnutzflaeche}{
  \erlaeuterung{Maximale wohnnutzflächenmäßige Ausnützbarkeit bezogen auf die Grundfläche (m2).}
  \wert{Eine positive Zahl [$m^2$]}
  % \beispiel{}
  \rechtsmaterie{\href{https://www.ris.bka.gv.at/NormDokument.wxe?Abfrage=LrW&Gesetzesnummer=20000006&FassungVom=2020-11-05&Artikel=&Paragraf=5&Anlage=&Uebergangsrecht=}{WBO
      \S 5/4/d,e}\\siehe Merkmal \merkmalref{AusnuetzbarkeitFlaecheBGF}% 
  }
  \notiz{Jetzt absorbiert in \merkmalref{Flaechen}.}
\quelle{Aus \href{run:./files/200526_BRISE_RIM-IDM_Auszug
    AI-BB.xlsx}{200626\_BRISE\_RIM-IDM\_Auszug\_AI-BB}
}
}

\merkmal{AusnuetzbarkeitFlaecheWohnnutzflaecheRelativMax}{
  \erlaeuterung{Maximale wohnnutzflächenmäßige Ausnützbarkeit bezogen auf die Grundfläche (relativ/prozentual).}
  \wert{Eine positive Zahl [\%]}
  % \beispiel{}
  \rechtsmaterie{\href{https://www.ris.bka.gv.at/NormDokument.wxe?Abfrage=LrW&Gesetzesnummer=20000006&FassungVom=2020-11-05&Artikel=&Paragraf=5&Anlage=&Uebergangsrecht=}{WBO
      \S 5/4/d,e}\\siehe Merkmal \merkmalref{AusnuetzbarkeitFlaecheBGF}% 
  }
  \notiz{Jetzt absorbiert in \merkmalref{Flaechen}.}
\quelle{Aus \href{run:./files/200526_BRISE_RIM-IDM_Auszug
    AI-BB.xlsx}{200626\_BRISE\_RIM-IDM\_Auszug\_AI-BB}
}
\frage{DONE Neu: (workshop 2020-12-11): Ist das minimal oder maximal? In
  6963\_29 ist es "minimal". => Maximal, aber koennte auch minimal
  sein => aufteilen in Max und Min (aber nur Wohnnutzflaeche)}
}

\merkmal{AusnuetzbarkeitFlaecheWohnnutzflaecheRelativMin}{
  \erlaeuterung{Minimale wohnnutzflächenmäßige Ausnützbarkeit bezogen auf die Grundfläche (relativ/prozentual).}
  \wert{Eine positive Zahl [\%]}
  \beispiel{"`In den als Wohnzone ausgewiesenen Bereichen ist \etc\ nur die Errichtung von Wohngebäuden zulässig, in denen nicht weniger als 80 v.H. der Summe der Nutzflächen der Hauptgeschoße, jedoch unter Ausschluß des Erdgeschoßes, Wohnzwecken vorbehalten sind."'\\(6963\_29)
  }
  \rechtsmaterie{\href{https://www.ris.bka.gv.at/NormDokument.wxe?Abfrage=LrW&Gesetzesnummer=20000006&FassungVom=2020-11-05&Artikel=&Paragraf=5&Anlage=&Uebergangsrecht=}{WBO
      \S 5/4/d,e}\\siehe Merkmal \merkmalref{AusnuetzbarkeitFlaecheBGF}% 
  }
  \notiz{Jetzt absorbiert in \merkmalref{Flaechen}.}
\quelle{Aus Workshop 2020-12-11
}
\frage{DONE Neu: (workshop 2020-12-11): Ist das minimal oder maximal? In
  6963\_29 ist es "minimal". => Maximal, aber koennte auch minimal
  sein => aufteilen in Max und Min (aber nur Wohnnutzflaeche)}
}

\merkmal{BebaubareFlaecheAbgegrenzt}{
  \erlaeuterung{Die nach § 5 Abs. 4 lit. d durch den Bebauungsplan beschränkte bebaubare Fläche des Bauplatzes (relativ/prozentual).}
  % \erlaeuterung{Beschränkt die bebaubare Fläche pro Fläche
  % (relativ).}
  \wert{Eine positive Zahl [\%]}
  \beispiel{"`Die mit Spk/BB3 bezeichnete Fläche darf im Ausmaß von höchstens 40 \% bebaut werden."'\\(7774\_11)}
  \rechtsmaterie{\href{https://www.ris.bka.gv.at/NormDokument.wxe?Abfrage=LrW&Gesetzesnummer=20000006&FassungVom=2020-11-05&Artikel=&Paragraf=82&Anlage=&Uebergangsrecht=}{WBO \S 82/5}:\\
  (5) Die durch Nebengebäude in Anspruch genommene Grundfläche ist auf die nach den gesetzlichen Ausnutzbarkeitsbestimmungen bebaubare Fläche und die die nach § 5 Abs. 4 lit. d durch den Bebauungsplan beschränkte bebaubare Fläche des Bauplatzes anzurechnen. Im Gartensiedlungsgebiet ist die mit einem Nebengebäude bebaute Grundfläche auf die Ausnutzbarkeitsbestimmungen eines Bauloses dann anzurechnen, wenn die bebaubare Fläche im Bebauungsplan mit mindestens 100 m2 festgesetzt ist.
}
  \notiz{Jetzt absorbiert in \merkmalref{Flaechen}.}
\quelle{Aus \href{run:./files/200526_BRISE_RIM-IDM_Auszug
    AI-BB.xlsx}{200626\_BRISE\_RIM-IDM\_Auszug\_AI-BB}
}
% \frage{Die Rechtsmaterie scheint sich nur auf Nebengebäude zu
%   beziehen. Das Merkmal auch? NEIN! -> geklaert}
}

\merkmal{BebaubareFlaecheGesamterBauplatz}{
  \erlaeuterung{Einschränkung der bebaubaren Fläche des gesamten Bauplatzes (relativ/prozentual).}
  % \erlaeuterung{Beschränkt die bebaubare Fläche auf dem
  %   Bauplatz relativ zur Fläche des gesamten Bauplatzes.}
  \wert{Eine positive Zahl [\%]}
  \beispiel{"`Auf den mit BB2 bezeichneten Teilen des Wohn- oder gemischten Baugebietes darf die bebaute Fläche maximal 20\% der Bauplatzgröße betragen."'\\(7408\_16)}
  \rechtsmaterie{\href{https://www.ris.bka.gv.at/NormDokument.wxe?Abfrage=LrW&Gesetzesnummer=20000006&FassungVom=2020-11-05&Artikel=&Paragraf=82&Anlage=&Uebergangsrecht=}{WBO
      \S 82/5}}:\\
  (5) Die durch Nebengebäude in Anspruch genommene Grundfläche ist auf die nach den gesetzlichen Ausnutzbarkeitsbestimmungen bebaubare Fläche und die die nach § 5 Abs. 4 lit. d durch den Bebauungsplan beschränkte bebaubare Fläche des Bauplatzes anzurechnen. Im Gartensiedlungsgebiet ist die mit einem Nebengebäude bebaute Grundfläche auf die Ausnutzbarkeitsbestimmungen eines Bauloses dann anzurechnen, wenn die bebaubare Fläche im Bebauungsplan mit mindestens 100 m2 festgesetzt ist.
  \notiz{Jetzt absorbiert in \merkmalref{Flaechen}.}
\quelle{Aus \href{run:./files/200526_BRISE_RIM-IDM_Auszug
    AI-BB.xlsx}{200626\_BRISE\_RIM-IDM\_Auszug\_AI-BB}
}
% \frage{Die Rechtsmaterie scheint sich nur auf Nebengebäude zu
%   beziehen. Das Merkmal auch? NEIN -> geklaert}
}

\merkmal{BebaubareFlaecheJeBauplatz}{
  \erlaeuterung{Einschränkung der bebaubaren Fläche je Bauplatz (m2)}
  % \erlaeuterung{Beschränkt die bebaubare Fläche auf dem
  %   Bauplatz absolut.}
  \wert{Eine positive Zahl [$m^2$]}
  \beispiel{"`Das Ausmaß der bebaubaren Fläche wird mit 100 $m^2$ limitiert."'\\(7443\_14)}
  \rechtsmaterie{\href{https://www.ris.bka.gv.at/NormDokument.wxe?Abfrage=LrW&Gesetzesnummer=20000006&FassungVom=2020-11-05&Artikel=&Paragraf=82&Anlage=&Uebergangsrecht=}{WBO \S 82/5}:\\
  (5) Die durch Nebengebäude in Anspruch genommene Grundfläche ist auf die nach den gesetzlichen Ausnutzbarkeitsbestimmungen bebaubare Fläche und die die nach § 5 Abs. 4 lit. d durch den Bebauungsplan beschränkte bebaubare Fläche des Bauplatzes anzurechnen. Im Gartensiedlungsgebiet ist die mit einem Nebengebäude bebaute Grundfläche auf die Ausnutzbarkeitsbestimmungen eines Bauloses dann anzurechnen, wenn die bebaubare Fläche im Bebauungsplan mit mindestens 100 m2 festgesetzt ist.
}
\notiz{Vergleiche auch
  \merkmalref{BebauteFlaechefuerNebengebaeudeJeBauplatzMax} für
  die Beschränkung der bebaubare Fläche für Nebengebäude.}
  \notiz{Jetzt absorbiert in \merkmalref{Flaechen}.}
\quelle{Aus \href{run:./files/200526_BRISE_RIM-IDM_Auszug
    AI-BB.xlsx}{200626\_BRISE\_RIM-IDM\_Auszug\_AI-BB}
}
% \frage{Die Rechtsmaterie scheint sich nur auf Nebengebäude zu
%   beziehen. Das Merkmal auch? NEIN -> geklaert}
\frage{DONE Sind die Beschränkungen "je Bauplatz" auch "je
  Bauplatzteil" relevant? Das kommt wohl in älteren Verordnungen
  vor. => Er fragt. Aber aufnehmen}
}

\merkmal{BebaubareFlaecheJeBauplatzteil}{
  \erlaeuterung{Einschränkung der bebaubaren Fläche je Bauplatzteil (m2)}
  % \erlaeuterung{Beschränkt die bebaubare Fläche auf dem
  %   Bauplatz absolut.}
  \wert{Eine positive Zahl [$m^2$]}
  % \beispiel{}
  % \rechtsmaterie{\href{https://www.ris.bka.gv.at/NormDokument.wxe?Abfrage=LrW&Gesetzesnummer=20000006&FassungVom=2020-11-05&Artikel=&Paragraf=82&Anlage=&Uebergangsrecht=}{WBO \S 82/5}:\\
%   (5) Die durch Nebengebäude in Anspruch genommene Grundfläche ist auf die nach den gesetzlichen Ausnutzbarkeitsbestimmungen bebaubare Fläche und die die nach § 5 Abs. 4 lit. d durch den Bebauungsplan beschränkte bebaubare Fläche des Bauplatzes anzurechnen. Im Gartensiedlungsgebiet ist die mit einem Nebengebäude bebaute Grundfläche auf die Ausnutzbarkeitsbestimmungen eines Bauloses dann anzurechnen, wenn die bebaubare Fläche im Bebauungsplan mit mindestens 100 m2 festgesetzt ist.
% }
  \notiz{Jetzt absorbiert in \merkmalref{Flaechen}.}
\quelle{Aus workshop 
}
% \frage{DONE Die Rechtsmaterie scheint sich nur auf Nebengebäude zu
%   beziehen. Das Merkmal auch? NEIN -> geklaert}
% \frage{Sind die Beschränkungen "je Bauplatz" auch "je
%   Bauplatzteil" relevant? Das kommt wohl in älteren Verordnungen
%   vor. => Er fragt. Aber aufnehmen}
}

\merkmal{BebaubareFlaecheJeGebaeude}{
  \erlaeuterung{Einschränkung der bebaubaren Fläche je Gebäude (m2).}
  \wert{Eine positive Zahl [$m^2$]}
  \beispiel{"`Die Gebäudehöhe darf 7,5 m nicht überschreiten, die bebaute Fläche hat maximal 250 $m^2$ zu betragen."'\\(7167\_15)}
  \rechtsmaterie{\href{https://www.ris.bka.gv.at/NormDokument.wxe?Abfrage=LrW&Gesetzesnummer=20000006&FassungVom=2020-11-05&Artikel=&Paragraf=82&Anlage=&Uebergangsrecht=}{WBO \S 82/5}:\\
  (5) Die durch Nebengebäude in Anspruch genommene Grundfläche ist auf die nach den gesetzlichen Ausnutzbarkeitsbestimmungen bebaubare Fläche und die die nach § 5 Abs. 4 lit. d durch den Bebauungsplan beschränkte bebaubare Fläche des Bauplatzes anzurechnen. Im Gartensiedlungsgebiet ist die mit einem Nebengebäude bebaute Grundfläche auf die Ausnutzbarkeitsbestimmungen eines Bauloses dann anzurechnen, wenn die bebaubare Fläche im Bebauungsplan mit mindestens 100 m2 festgesetzt ist.}
  \notiz{Jetzt absorbiert in \merkmalref{Flaechen}.}
\quelle{Aus \href{run:./files/200526_BRISE_RIM-IDM_Auszug
    AI-BB.xlsx}{200626\_BRISE\_RIM-IDM\_Auszug\_AI-BB}
}
% \frage{Die Rechtsmaterie scheint sich nur auf Nebengebäude zu
%   beziehen. Das Merkmal auch? NEIN -> geklaert}
}

% \merkmal{BefestigungFuerSchuleOderSpielplatz}{
%   \erlaeuterung{}
%   \wert{}
%   \beispiel{}
%   \rechtsmaterie{}
% }

% \merkmal{BebaubareFlaeche}{
%   % \erlaeuterung{}
%   \wert{Eine positive Zahl [$m^2$]}
%   % \beispiel{}
%   % \rechtsmaterie{}
%   \notiz{aus WienBV\_Bebauungsbestimmungen}
%   \quelle{Aus \href{run:./files/200929_BRISE_IDM-REM-LOI.xlsx}{200929\_BRISE\_IDM REM LOI}
%   }
%   \frage{Erläuterung? Ist das relevant? EHER NEIN}
% }

% \merkmal{BebaubareFlaecheBeschraenkung}{
%   % \erlaeuterung{}
%   \wert{Eine positive Zahl [$m^2$]}
%   % \beispiel{}
%   % \rechtsmaterie{}
%   \notiz{aus WienBV\_Bebauungsbestimmungen}
%   \quelle{Aus \href{run:./files/200929_BRISE_IDM-REM-LOI.xlsx}{200929\_BRISE\_IDM REM LOI}
%   }
%   \frage{Erläuterung? Ist das relevant? EHER NEIN}
%   \frage{Ist das das gleiche wie \merkmalref{BebaubareFlaeche}?}
% }

% \merkmal{BebauteFlaechefuerNebengebaeudeMax}{
%   \erlaeuterung{Spezifiziert das maximale Ausmaß der mit
%     Nebengebäuden bebaubaren Grundfläche.}
%   \wert{Eine positive Zahl [$m^2$]}
%   \beispiel{"`Die mit Nebengebäuden bebaute Grundfläche darf höchstens
%     30m2 je Bauplatz betragen"'\\(7531\_9)}
%   % \rechtsmaterie{}
%   \quelle{Aus den Beispielen sowie
%     \href{run:./files/Merkmale_neu_liste-WP4-2020-11-06.xlsx}{Merkmale\_neu\_liste
%     WP4 (Version von 2020-11-06)}
%     }
% }

\merkmal{BebauteFlaechefuerNebengebaeudeJeBauplatzMax}{
  \erlaeuterung{Spezifiziert das maximale Ausmaß der mit
    Nebengebäuden bebaubaren Grundfläche auf einem Bauplatz.}
  \wert{Eine positive Zahl [$m^2$]}
  \beispiel{"`Die mit Nebengebäuden bebaute Grundfläche darf höchstens
    30m2 je Bauplatz betragen"'\\(7531\_9)}
  % \rechtsmaterie{}
  \notiz{Jetzt absorbiert in \merkmalref{Flaechen}.}
  \quelle{Aus den Beispielen sowie
    \href{run:./files/Merkmale_neu_liste-WP4-2020-11-06.xlsx}{Merkmale\_neu\_liste
    WP4 (Version von 2020-11-06)}. Die Unterscheidung Bauplatz/Baulos
  aus Annotation Kleinworkshop 2020-11-27.
    }
}

\merkmal{BebauteFlaechefuerNebengebaeudeJeBaulosMax}{
  \erlaeuterung{Spezifiziert das maximale Ausmaß der mit
    Nebengebäuden bebaubaren Grundfläche auf einem Baulos.}
  \wert{Eine positive Zahl [$m^2$]}
  % \beispiel{"`Die mit Nebengebäuden bebaute Grundfläche darf höchstens
  %   30m2 je Bauplatz betragen"'\\(7531\_9)}
  % \rechtsmaterie{}
  \notiz{Jetzt absorbiert in \merkmalref{Flaechen}.}
  \quelle{Aus den Beispielen sowie
    \href{run:./files/Merkmale_neu_liste-WP4-2020-11-06.xlsx}{Merkmale\_neu\_liste
    WP4 (Version von 2020-11-06)}. Die Unterscheidung Bauplatz/Baulos
  aus Annotation Kleinworkshop 2020-11-27.
    }
    \frage{DONE Neu: Unterscheidung Bauplatz/Baulos -> neues Merkmal. Passt
    das? => ja.}
}

% \merkmal{ReferenzFlaecheFWBP}{
%   % \erlaeuterung{}
%   \wert{Wahrheitswert}
%   % \beispiel{}
%   % \rechtsmaterie{}
%   \notiz{aus WienBV\_Bebauungsbestimmungen}
%   \quelle{Aus \href{run:./files/200929_BRISE_IDM-REM-LOI.xlsx}{200929\_BRISE\_IDM REM LOI}
%   }
%   \frage{Erläuterung? Ist das relevant? NEIN}
% }

% \merkmal{WidmungAusnutzbarkeitFlaeche}{
%   % \erlaeuterung{}
%   \wert{Eine positive Zahl [$m^2$]}
%   % \beispiel{}
%   \rechtsmaterie{\href{https://www.ris.bka.gv.at/NormDokument.wxe?Abfrage=LrW&Gesetzesnummer=20000006&FassungVom=2020-11-05&Artikel=&Paragraf=5&Anlage=&Uebergangsrecht=}{WBO
%       \S 5/4/d}:\\
%     (4) Über die Festsetzungen nach Abs. 2 und 3 hinaus können die
% Bebauungspläne zusätzlich enthalten:\\
% \etc\\
% d)
% Bestimmungen über die flächenmäßige beziehungsweise volumenbezogene Ausnützbarkeit der Bauplätze und der Baulose oder von Teilen davon; in Gebieten für geförderten Wohnbau Bestimmungen über den Anteil der Wohnnutzfläche der auf einem Bauplatz geschaffenen Wohnungen und Wohneinheiten in Heimen, die hinsichtlich der Grundkostenangemessenheit dem Wiener Wohnbauförderungs- und Wohnhaussanierungsgesetz – WWFSG 1989 entsprechen müssen;
% }
%   \notiz{aus WienBV\_Bebauungsbestimmungen}
%   \quelle{Aus \href{run:./files/200929_BRISE_IDM-REM-LOI.xlsx}{200929\_BRISE\_IDM REM LOI}
%   }
%   \frage{Erläuterung? Ist das relevant? EHER NEIN}
% }

\merkmal{FlaecheBebaubar}{
  \erlaeuterung{Gibt an, ob eine Grundfläche bebaubar ist. % ("wahr",
    % wenn bebaubar).
  }
  \wert{Wahrheitswert ["`wahr"' wenn bebaubar]}
  \beispiel{"`Nicht bebaute, jedoch bebaubare Grundflächen sind gärtnerisch auszugestalten."'\\(7020\_16)}
  % \rechtsmaterie{}
  \notiz{Oft in Verbindung mit Merkmal \merkmalref{FlaecheBebaut}.}
  \notiz{Jetzt absorbiert in \merkmalref{Flaechen}.}
  \quelle{Aus Kleinworkshop 2020-11-27.}
  % \frage{DONE 2020-11-27: Neues Merkmal => drinnenlassen}
}


\merkmal{FlaecheBebaut}{
  \erlaeuterung{Gibt an, ob eine Grundfläche bebaut ist. %  ("wahr"
    % wenn bebaut).
  }
  \wert{Wahrheitswert ["`wahr wenn bebaut"']}
  \beispiel{"`Nicht bebaute, jedoch bebaubare Grundflächen sind
    gärtnerisch auszugestalten."'\\(7020\_16)\smallskip\\
  "`Auf den mit BB3 bezeichneten Grundflächen sind die nicht bebauten Grundflächen gärtnerisch auszugestalten."'\\(7702\_9)}
  % \rechtsmaterie{}
  \notiz{Oft in Verbindung mit Merkmal \merkmalref{FlaecheBebaubar}.}
  \notiz{Jetzt absorbiert in \merkmalref{Flaechen}.}
  \quelle{Aus Kleinworkshop 2020-11-27.}
  % \frage{DONE 2020-11-27: Neues Merkmal  => drinnenlassen}
}

% \merkmal{}{
%   \erlaeuterung{}
%   \wert{}
%   \beispiel{}
%   \rechtsmaterie{}
% }



\subsection{Grossbauvorhaben, Einkaufszentren und Geschäftsstrassen}

\merkmal{EinkaufszentrumMaxFlaeche}{
  \erlaeuterung{Maximal Fläche des Einkaufzentrums je Widmungsfläche.}
  \wert{Eine positive Zahl [$m^2$]}
  \beispiel{"`Auf den mit EKZ BB4 bezeichneten Flächen ist die Errichtung eines Einkaufszentrums zulässig, wobei die von Räumen gemäß § 7c Abs. 1 BO für Wien in Anspruch genommene Gesamtfläche 4.300 $m^2$ nicht überschreiten darf."'\\(7443\_12)}
  \rechtsmaterie{\href{https://www.ris.bka.gv.at/NormDokument.wxe?Abfrage=LrW&Gesetzesnummer=20000006&FassungVom=2020-11-05&Artikel=&Paragraf=7c&Anlage=&Uebergangsrecht=}{WBO
      \S 7c/3}:\\
  (3) Für Einkaufszentren kann im Bebauungsplan eine höchstens zulässige Fläche (Abs. 1), bezogen auf eine durch Fluchtlinien bestimmte Grundfläche, festgelegt werden; ferner kann zur Sicherung der räumlich funktionellen Nahebeziehungen, der zeitgemäßen Bedürfnisse und der sozialen Struktur der Bevölkerung festgelegt werden, dass nur Fachmärkte, aber keine Einkaufszentren für Lebens- und Genussmittel der Grundversorgung errichtet werden dürfen.}
\quelle{Aus \href{run:./files/200526_BRISE_RIM-IDM_Auszug
    AI-BB.xlsx}{200626\_BRISE\_RIM-IDM\_Auszug\_AI-BB}
}
\notiz{Jetzt absorbiert durch \merkmalref{Flaechen}.}
}

\merkmal{EinkaufszentrumZweck}{
  \erlaeuterung{Wenn die Bestimmung für die maximal Fläche des Einkaufzentrums für eine bestimmten Verwendungszwecks beschränkt ist.}
  \wert{Text}
  % \beispiel{}
  \rechtsmaterie{\href{https://www.ris.bka.gv.at/NormDokument.wxe?Abfrage=LrW&Gesetzesnummer=20000006&FassungVom=2020-11-05&Artikel=&Paragraf=7c&Anlage=&Uebergangsrecht=}{WBO \S 7c/3}:\\
  (3) Für Einkaufszentren kann im Bebauungsplan eine höchstens zulässige Fläche (Abs. 1), bezogen auf eine durch Fluchtlinien bestimmte Grundfläche, festgelegt werden; ferner kann zur Sicherung der räumlich funktionellen Nahebeziehungen, der zeitgemäßen Bedürfnisse und der sozialen Struktur der Bevölkerung festgelegt werden, dass nur Fachmärkte, aber keine Einkaufszentren für Lebens- und Genussmittel der Grundversorgung errichtet werden dürfen.}
% \notiz{Siehe Merkmal \merkmalref{} für die Beschränkung der Fläche.}
\quelle{Aus \href{run:./files/200526_BRISE_RIM-IDM_Auszug
    AI-BB.xlsx}{200626\_BRISE\_RIM-IDM\_Auszug\_AI-BB}
}
% \frage{Klärung Erklärung? NUTZUNGSART}
\notiz{Jetzt absorbiert durch \merkmalref{WidmungUndZweckbestimmung}.}
}
\merkmal{GrossbauvorhabenMaxFlaeche}{
  \erlaeuterung{Maximal Fläche des Großbauvorhabens je Widmungsfläche.}
  \wert{Eine positive Zahl [$m^2$]}
  % \beispiel{}
  \rechtsmaterie{\href{https://www.ris.bka.gv.at/NormDokument.wxe?Abfrage=LrW&Gesetzesnummer=20000006&FassungVom=2020-11-05&Artikel=&Paragraf=7b&Anlage=&Uebergangsrecht=}{WBO
      \S 7b/6}:\\
  (6) Für Großbauvorhaben kann im Bebauungsplan eine höchstens zulässige Fläche, auch für einzelne festgesetzte Zwecke, bezogen auf eine durch Fluchtlinien bestimmte Grundfläche, festgelegt werden.}
\quelle{Aus \href{run:./files/200526_BRISE_RIM-IDM_Auszug
    AI-BB.xlsx}{200626\_BRISE\_RIM-IDM\_Auszug\_AI-BB}
}
\notiz{Jetzt absorbiert durch \merkmalref{Flaechen}.}
}

\merkmal{GrossbauvorhabenZweck1}{
  \erlaeuterung{Wenn die Bestimmung für die maximal Fläche des Großbauvorhabens für eine bestimmten Verwendungszwecks beschränkt ist.}
  \wert{Text}
  % \beispiel{}
  \rechtsmaterie{\href{https://www.ris.bka.gv.at/NormDokument.wxe?Abfrage=LrW&Gesetzesnummer=20000006&FassungVom=2020-11-05&Artikel=&Paragraf=7b&Anlage=&Uebergangsrecht=}{WBO \S 7b/6}:\\
  (6) Für Großbauvorhaben kann im Bebauungsplan eine höchstens zulässige Fläche, auch für einzelne festgesetzte Zwecke, bezogen auf eine durch Fluchtlinien bestimmte Grundfläche, festgelegt werden.}
\quelle{Aus \href{run:./files/200526_BRISE_RIM-IDM_Auszug
    AI-BB.xlsx}{200626\_BRISE\_RIM-IDM\_Auszug\_AI-BB}
}
% \frage{Klärung Erklärung?}
\notiz{Jetzt absorbiert durch \merkmalref{WidmungUndZweckbestimmung}.}
}



\subsection{Nutzung und Widmung}
\merkmal{WidmungID}{
  \erlaeuterung{Spezifiziert die Widmung, beispielsweise "L", "GB",
    "Betriebsbaugebiet", ``Sondergebiet", etc}
  \wert{Text}
  \beispiel{"`Auf den als ländliches Gebiet (L) gewidmeten
    Flächen ist die Errichtung von Glashäusern
    unzulässig."'\\(7181\_8)\smallskip\\
  "`Auf den mit BB2 bezeichneten Teilen des Wohn- oder gemischten
  Baugebietes darf die bebaute Fläche maximal 20\% der Bauplatzgröße
  betragen."'\\(7408\_16)\smallskip\\
  "`In den mit BB5 und als Bauland/Gartensiedlungsgebiet gewidmeten Flächen ist eine Dachneigung von maximal 45 Grad zulässig."'\\(7443\_12)}
  \notiz{Für Widmungen in mehreren Ebenen, siehe Merkmale
    \merkmalref{WidmungErsteEbene}, \merkmalref{WidmungZweiteEbene},
    etc. Für Zweckbestimmungen innerhalb der Widmung siehe das
    Merkmal \merkmalref{ZweckbestimmungWidmungskategorie}.}
  \notiz{Jetzt absorbiert durch \merkmalref{WidmungUndZweckbestimmung}.}
  \quelle{Aus den Beispielen sowie
    \href{run:./files/Merkmale_neu_liste-WP4-2020-11-06.xlsx}{Merkmale\_neu\_liste
      WP4 (Version von 2020-11-06)}
    }
    % \frage{Ist das das gleiche wie
    %   \merkmalref{ZweckbestimmungWidmungskategorie1}? NEIN}
}




\merkmal{WidmungErsteEbene}{
  \erlaeuterung{Widmungsart der ersten Ebene wenn es mehrere Ebenen gibt.}
  \wert{Text}
  % \beispiel{}
  \rechtsmaterie{\href{https://www.ris.bka.gv.at/NormDokument.wxe?Abfrage=LrW&Gesetzesnummer=20000006&FassungVom=2020-11-05&Artikel=&Paragraf=4&Anlage=&Uebergangsrecht=}{WBO
      \S 4/3}:\\
  (3) Die Flächenwidmungspläne können für verschiedene übereinanderliegende Räume desselben Plangebietes gesonderte Widmungen ausweisen.}
\notiz{Nur für Widmungen in mehreren Ebenen. Für Widmungen mit
  nur einer Ebene insgesamt, siehe Merkmal \merkmalref{WidmungID}.} 
  \notiz{Jetzt absorbiert durch \merkmalref{WidmungInMehrerenEbenen}.}

\quelle{Aus \href{run:./files/200526_BRISE_RIM-IDM_Auszug
    AI-BB.xlsx}{200626\_BRISE\_RIM-IDM\_Auszug\_AI-BB}
}
% \frage{Schließt diese Merkmal das Merkmal \merkmalref{WidmungID}
%   bzw. \merkmalref{ZweckbestimmungWidmungskategorie1} aus? JA, HIER
%   NUR FUER MEHRERE EBENEN}
}

\merkmal{WidmungErsteEbeneBezugHoehe}{
  \erlaeuterung{Höhe ab welcher die erste Widmung erlaubt ist (Untergrenze).}
  \wert{Eine positive Zahl [m]}
  % \beispiel{}
  \rechtsmaterie{\href{https://www.ris.bka.gv.at/NormDokument.wxe?Abfrage=LrW&Gesetzesnummer=20000006&FassungVom=2020-11-05&Artikel=&Paragraf=4&Anlage=&Uebergangsrecht=}{WBO
      \S 4/3}:\\
  (3) Die Flächenwidmungspläne können für verschiedene übereinanderliegende Räume desselben Plangebietes gesonderte Widmungen ausweisen.}
\notiz{Für die Obergrenze, also die Grenze zwischen erster und
  zweiter Ebene siehe das Merkmal \merkmalref{WidmungZweiteEbeneBezugHoehe}.
}
  \notiz{Jetzt absorbiert durch \merkmalref{WidmungInMehrerenEbenen}.}
\quelle{Aus \href{run:./files/200526_BRISE_RIM-IDM_Auszug
    AI-BB.xlsx}{200626\_BRISE\_RIM-IDM\_Auszug\_AI-BB}
}
\frage{DONE Neu: Ist das die Ober- oder Untergrenze? (Haben immer nur die
  Obergrenze gesehen) => ist die Untergrenze => in Bemerkungen
  reinschreiben mit verweis auf zweite ebene}
}

\merkmal{WidmungErsteEbeneBezugObjekt}{
  \erlaeuterung{Gibt ein Objekt an, ab welchem die erste Widmung
    erlaubt ist (Untergrenze).}
  \wert{Text}
  % \beispiel{"`Für die mit BB 3 bezeichneten Flächen werden gesonderte Widmungen für zwei übereinander liegende Räume derart getroffen, daß der bis zur Deckenoberkante der unterirdischen Objekte reichende Raum dem Bauland/Wohngebiet und der Raum darüber dem Grünland/Erholungsgebiet Parkanlage, Grundfläche für öffentliche Zwecke zugeordnet wird."'\\(7020\_8)}
  % \rechtsmaterie{\href{https://www.ris.bka.gv.at/NormDokument.wxe?Abfrage=LrW&Gesetzesnummer=20000006&FassungVom=2020-11-05&Artikel=&Paragraf=4&Anlage=&Uebergangsrecht=}{WBO
  %     \S 4/3}:\\
  % (3) Die Flächenwidmungspläne können für verschiedene übereinanderliegende Räume desselben Plangebietes gesonderte Widmungen ausweisen.}
  \notiz{Für die Obergrenze, also die Grenze zwischen erster und
  zweiter Ebene siehe das Merkmal \merkmalref{WidmungZweiteEbeneBezugObjekt}.
  }
  \notiz{Jetzt absorbiert durch \merkmalref{WidmungInMehrerenEbenen}.}
\quelle{Aus Annotation Kleinworkshop 2020-11-27
}
\frage{DONE Neu: Vorschlag Merkmal => passt}
}

\merkmal{WidmungZweiteEbene}{
  \erlaeuterung{Widmungsart der zweiten Ebene.}
  \wert{Text}
  % \beispiel{}
  \rechtsmaterie{\href{https://www.ris.bka.gv.at/NormDokument.wxe?Abfrage=LrW&Gesetzesnummer=20000006&FassungVom=2020-11-05&Artikel=&Paragraf=4&Anlage=&Uebergangsrecht=}{WBO
      \S 4/3}:\\
  (3) Die Flächenwidmungspläne können für verschiedene übereinanderliegende Räume desselben Plangebietes gesonderte Widmungen ausweisen.}
\notiz{Nur für Widmungen in mehreren Ebenen. Für Widmungen mit
  nur einer Ebene insgesamt, siehe Merkmal \merkmalref{WidmungID}.} 
  \notiz{Jetzt absorbiert durch \merkmalref{WidmungInMehrerenEbenen}.}
\quelle{Aus \href{run:./files/200526_BRISE_RIM-IDM_Auszug
    AI-BB.xlsx}{200626\_BRISE\_RIM-IDM\_Auszug\_AI-BB}
}
}

\merkmal{WidmungZweiteEbeneBezugHoehe}{
  \erlaeuterung{Höhe ab welcher die zweite Widmung erlaubt ist (Untergrenze).}
  \wert{Eine positive Zahl [m]}
  \beispiel{"`Für die mit BB 14 bezeichnete Grundfläche werden
  gesonderte Bebauungsbestimmungen für zwei übereinanderliegende
  Räume derart getroffen, daß der Raum ab einer Tiefe von 1,0 m
  unter dem bestehenden Niveau dem Einstellen von Kraftfahrzeugen
  vorbehalten bleibt und der Raum darüber dem Parkschutzgebiet
  zugeordnet wird."'\\
  (7020\_45)
}
  \notiz{Jetzt absorbiert durch \merkmalref{WidmungInMehrerenEbenen}.}
  \rechtsmaterie{\href{https://www.ris.bka.gv.at/NormDokument.wxe?Abfrage=LrW&Gesetzesnummer=20000006&FassungVom=2020-11-05&Artikel=&Paragraf=4&Anlage=&Uebergangsrecht=}{WBO
      \S 4/3}:\\
  (3) Die Flächenwidmungspläne können für verschiedene übereinanderliegende Räume desselben Plangebietes gesonderte Widmungen ausweisen.}
\quelle{Aus \href{run:./files/200526_BRISE_RIM-IDM_Auszug
    AI-BB.xlsx}{200626\_BRISE\_RIM-IDM\_Auszug\_AI-BB}
}
}

\merkmal{WidmungZweiteEbeneBezugObjekt}{
  \erlaeuterung{Gibt ein Objekt an, ab welchem die zweite Widmung
    erlaubt ist (Untergrenze).}
  \wert{Text}
  \beispiel{"`Für die mit BB 3 bezeichneten Flächen werden gesonderte Widmungen für zwei übereinander liegende Räume derart getroffen, daß der bis zur Deckenoberkante der unterirdischen Objekte reichende Raum dem Bauland/Wohngebiet und der Raum darüber dem Grünland/Erholungsgebiet Parkanlage, Grundfläche für öffentliche Zwecke zugeordnet wird."'\\(7020\_8)}
  % \rechtsmaterie{\href{https://www.ris.bka.gv.at/NormDokument.wxe?Abfrage=LrW&Gesetzesnummer=20000006&FassungVom=2020-11-05&Artikel=&Paragraf=4&Anlage=&Uebergangsrecht=}{WBO
  %     \S 4/3}:\\
  % (3) Die Flächenwidmungspläne können für verschiedene übereinanderliegende Räume desselben Plangebietes gesonderte Widmungen ausweisen.}
  \notiz{Jetzt absorbiert durch \merkmalref{WidmungInMehrerenEbenen}.}
\quelle{Aus Annotation Kleinworkshop 2020-11-27
}
}

\merkmal{WidmungDritteEbene}{
  \erlaeuterung{Widmungsart der dritten Ebene}
  \wert{Text}
  % \beispiel{}
  \rechtsmaterie{\href{https://www.ris.bka.gv.at/NormDokument.wxe?Abfrage=LrW&Gesetzesnummer=20000006&FassungVom=2020-11-05&Artikel=&Paragraf=4&Anlage=&Uebergangsrecht=}{WBO
      \S 4/3}:\\
  (3) Die Flächenwidmungspläne können für verschiedene übereinanderliegende Räume desselben Plangebietes gesonderte Widmungen ausweisen.}
\notiz{Nur für Widmungen in mehreren Ebenen. Für Widmungen mit
  nur einer Ebene insgesamt, siehe Merkmal \merkmalref{WidmungID}.} 
  \notiz{Jetzt absorbiert durch \merkmalref{WidmungInMehrerenEbenen}.}
\quelle{Analog zu \merkmalref{WidmungErsteEbene}.}
}

\merkmal{WidmungDritteEbeneBezugHoehe}{
  \erlaeuterung{Höhe ab welcher die dritte Widmung erlaubt ist.}
  \wert{Eine positive Zahl [m]}
  % \beispiel{}
  \rechtsmaterie{\href{https://www.ris.bka.gv.at/NormDokument.wxe?Abfrage=LrW&Gesetzesnummer=20000006&FassungVom=2020-11-05&Artikel=&Paragraf=4&Anlage=&Uebergangsrecht=}{WBO
      \S 4/3}:\\
  (3) Die Flächenwidmungspläne können für verschiedene übereinanderliegende Räume desselben Plangebietes gesonderte Widmungen ausweisen.}
  \notiz{Jetzt absorbiert durch \merkmalref{WidmungInMehrerenEbenen}.}
\quelle{Analog zu \merkmalref{WidmungErsteEbeneBezugHoehe}.}
}

\merkmal{WidmungDritteEbeneBezugObjekt}{
  \erlaeuterung{Gibt ein Objekt an, ab welchem die dritte Widmung erlaubt ist.}
  \wert{Text}
  % \beispiel{"`Für die mit BB 3 bezeichneten Flächen werden gesonderte Widmungen für zwei übereinander liegende Räume derart getroffen, daß der bis zur Deckenoberkante der unterirdischen Objekte reichende Raum dem Bauland/Wohngebiet und der Raum darüber dem Grünland/Erholungsgebiet Parkanlage, Grundfläche für öffentliche Zwecke zugeordnet wird."'\\(7020\_8)}
  % \rechtsmaterie{\href{https://www.ris.bka.gv.at/NormDokument.wxe?Abfrage=LrW&Gesetzesnummer=20000006&FassungVom=2020-11-05&Artikel=&Paragraf=4&Anlage=&Uebergangsrecht=}{WBO
  %     \S 4/3}:\\
  % (3) Die Flächenwidmungspläne können für verschiedene übereinanderliegende Räume desselben Plangebietes gesonderte Widmungen ausweisen.}
  \notiz{Jetzt absorbiert durch \merkmalref{WidmungInMehrerenEbenen}.}
\quelle{Analog zu \merkmalref{WidmungErsteEbeneBezugObjekt}.
}
}

\merkmal{ZweckbestimmungWidmungskategorie}{
  \erlaeuterung{Festlegung des Zweckes der Gebäude oder Baulichkeiten.}
  \wert{Text}
  \beispiel{"`Auf der mit BB3 bezeichneten Fläche ist die Errichtung
    von Gebäuden für gastronomische Zwecke und für Zwecke der
    Bootsvermietung \etc\ % mit einer maximal bebauten Grundfläche von 550 m² und einer Gebäudehöhe von maximal 4,0 m
    zulässig."'\\(7545\_12)\smallskip\\
  "`Auf den mit BB1 bezeichneten Grundflächen sind die Gebäude der
  Zweckbestimmung Bildungseinrichtung zuzuführen."'\\(8008\_22)\smallskip\\
  "`Auf der mit G BB4 bezeichneten Fläche ist die Errichtung von
  Spielplätzen und eine damit verbundene Befestigung der Grundfläche
  im erforderlichen Ausmaß zulässig [\dots]."'\\(7356\_11)\smallskip\\
  "`Die mit G BB5 bezeichneten Flächen sind der Errichtung von
  baulichen Anlagen unter Niveau zum Einstellen von Kraftfahrzeugen
  vorbehalten."'\\(7356\_12)\smallskip\\
  "`Auf den mit BB2 bezeichneten Flächen ist die Errichtung von
  Baulichkeiten, die der Erholung und Gesundheit der Bevölkerung
  dienen, mit einer Gebäudehöhe von maximal 4,5 m
  zulässig."'\\(7548\_8)\smallskip\\
  "`Für die mit BB5 bezeichneten Grundflächen wird bestimmt: Die Gebäude dürfen nur für Einrichtungen der öffentlichen Versorgung und Sicherheit verwendet werden."'\\(8038\_23)
}
  \rechtsmaterie{\href{https://www.ris.bka.gv.at/NormDokument.wxe?Abfrage=LrW&Gesetzesnummer=20000006&FassungVom=2020-11-05&Artikel=&Paragraf=77&Anlage=&Uebergangsrecht=}{WBO
      \S 77/4/c}:\\
  (4) Über die Festsetzung nach Abs. 2 und 3 hinaus können die Bebauungspläne für Strukturen zusätzlich enthalten:\\
a)
weitere Bestimmungen über die Gebäudehöhe und den obersten Abschluss des Daches;\\
b)
verschiedene Widmungen der Grundflächen auf dem Bauplatz;\\
c)
die Zweckbestimmungen innerhalb der Widmungskategorie, denen die Gebäude zuzuführen sind.
}
\notiz{Es können auch mehrere Zweckbestimmungen angegeben
  sein. Für die Widmung allgemein, siehe \merkmalref{WidmungID},
  für Widmung in mehreren Ebenen siehe \merkmalref{WidmungErsteEbene}.}
  \notiz{Jetzt absorbiert durch \merkmalref{WidmungUndZweckbestimmung}.}
\quelle{Aus \href{run:./files/200526_BRISE_RIM-IDM_Auszug
    AI-BB.xlsx}{200626\_BRISE\_RIM-IDM\_Auszug\_AI-BB}
}
\frage{DONE NOTE: Haben es geändert von
  \mml{ZweckbestimmungWidmungskategorie1}, da das leichter zu
  trainieren ist. Werden dann aber bei den Werten mehrere Vorkommnisse
  annotieren.}
}



\subsection{Volumen}


\merkmal{AusnuetzbarkeitVolumenBaumasse}{
  \erlaeuterung{Volumenbezogene Ausnützbarkeit bezogen auf die Baumasse (m3).}
  \wert{Eine positive Zahl [$m^3$]}
  % \beispiel{}
  \rechtsmaterie{\href{https://www.ris.bka.gv.at/NormDokument.wxe?Abfrage=LrW&Gesetzesnummer=20000006&FassungVom=2020-11-05&Artikel=&Paragraf=5&Anlage=&Uebergangsrecht=}{WBO
      \S 5/4/d,e}:\\siehe Merkmal \merkmalref{AusnuetzbarkeitVolumenRelativ} % 
  }
  \notiz{Jetzt absorbiert durch \merkmalref{VolumenUndUmbaubarerRaum}.}
\quelle{Aus \href{run:./files/200526_BRISE_RIM-IDM_Auszug
    AI-BB.xlsx}{200626\_BRISE\_RIM-IDM\_Auszug\_AI-BB}
}
}


\merkmal{AusnuetzbarkeitVolumenBaumasseRelativ}{
  \erlaeuterung{Volumenbezogene Ausnützbarkeit bezogen auf die Baumasse (relativ/prozentual).}
  \wert{Eine positive Zahl [\%]}
  % \beispiel{}
  \rechtsmaterie{\href{https://www.ris.bka.gv.at/NormDokument.wxe?Abfrage=LrW&Gesetzesnummer=20000006&FassungVom=2020-11-05&Artikel=&Paragraf=5&Anlage=&Uebergangsrecht=}{WBO
      \S 5/4/d,e}:\\siehe Merkmal \merkmalref{AusnuetzbarkeitVolumenRelativ}% 
  }
  \notiz{Jetzt absorbiert durch \merkmalref{VolumenUndUmbaubarerRaum}.}
\quelle{Aus \href{run:./files/200526_BRISE_RIM-IDM_Auszug
    AI-BB.xlsx}{200626\_BRISE\_RIM-IDM\_Auszug\_AI-BB}
}
}


\merkmal{AusnuetzbarkeitVolumenRelativ}{
  \erlaeuterung{Volumenbezogene Ausnützbarkeit bezogen auf die Grundfläche (relativ/prozentual).}
  \wert{Eine positive Zahl [\%]}
  % \beispiel{}
  \rechtsmaterie{\href{https://www.ris.bka.gv.at/NormDokument.wxe?Abfrage=LrW&Gesetzesnummer=20000006&FassungVom=2020-11-05&Artikel=&Paragraf=5&Anlage=&Uebergangsrecht=}{WBO
      \S 5/4/d,e}: \\
(4) Über die Festsetzungen nach Abs. 2 und 3 hinaus können die
Bebauungspläne zusätzlich enthalten:\\
\etc\\
d)
Bestimmungen über die flächenmäßige beziehungsweise volumenbezogene Ausnützbarkeit der Bauplätze und der Baulose oder von Teilen davon; in Gebieten für geförderten Wohnbau Bestimmungen über den Anteil der Wohnnutzfläche der auf einem Bauplatz geschaffenen Wohnungen und Wohneinheiten in Heimen, die hinsichtlich der Grundkostenangemessenheit dem Wiener Wohnbauförderungs- und Wohnhaussanierungsgesetz – WWFSG 1989 entsprechen müssen;\\
e)
Bestimmungen über die bauliche Ausnützbarkeit von ländlichen Gebieten,
Parkanlagen, Freibädern, Parkschutzgebieten und Grundflächen für
Badehütten, bei Gewässern auch die Ausweisung der von jeder Bebauung
freizuhaltenden Uferzonen; Bestimmungen über die bauliche
Ausnützbarkeit von Sport- und Spielplätzen, bei Sportplätzen auch in
bezug auf Sporthallen, sowie eine höchstens zulässige bebaubare
Fläche, bezogen auf eine durch Grenzlinien bestimmte Grundfläche;
Bestimmungen über die Ausnützbarkeit der Sondernutzungsgebiete
hinsichtlich der Art, des Zweckes, ihres Umfanges und ihrer Abgrenzung
zu Nutzungen anderer Art sowie hinsichtlich der endgültigen Gestaltung
ihrer Oberflächen unter Festsetzung der beabsichtigten Wirkung auf das
örtliche Stadt- bzw. Landschaftsbild nach der endgültigen Widmung der
Widmungskategorie Grünland für die endgültige Nutzung der Grundflächen
durch Bestimmung von Geländehöhen (Überhöhungen und Vertiefungen),
Böschungswinkeln, Bepflanzungen der endgültigen baulichen
Ausnützbarkeit und ähnlichem; die Festsetzung eines Zeitpunktes für
die Herstellung der endgültigen Widmung ist zulässig;}
  \notiz{Jetzt absorbiert durch \merkmalref{VolumenUndUmbaubarerRaum}.}
\quelle{Aus \href{run:./files/200526_BRISE_RIM-IDM_Auszug
    AI-BB.xlsx}{200626\_BRISE\_RIM-IDM\_Auszug\_AI-BB}
}
}


\merkmal{UmbaubarerRaumBauplatzMax}{
  % \erlaeuterung{Beschränkung des maximal umbaubaren Raumes auf
  %   einem Bauplatz in Kubikmetern}
  \erlaeuterung{Beschränkt  das höchste zulässige Ausmaß des umbaubaren Raumes der Bauwerke auf dem Bauplatz.}
  % \erlaeuterung{Beschränkt im Strukturgebiet das höchste zulässige Ausmaß des umbaubaren Raumes der Bauwerke auf dem Bauplatz.}
  \wert{Eine positive Zahl [$m^3$].}
  \beispiel{
    "`Auf den mit BB7 bezeichneten Flächen darf das Ausmaß des umbauten Raumes insgesamt 65.000 $\mathrm{m}^3$ nicht überschreiten."'\\(6986e\_9)
  }
  \rechtsmaterie{
    \href{https://www.ris.bka.gv.at/NormDokument.wxe?Abfrage=LrW&Gesetzesnummer=20000006&FassungVom=2020-11-05&Artikel=&Paragraf=77&Anlage=&Uebergangsrecht=}{WBO
      \S 77/3/b}:\\
  (3) Über jede Struktureinheit hat der Bebauungsplan folgende Festsetzung zu enthalten:\\
  % a)
  % welche Teile des Bauplatzes unmittelbar bebaut werden dürfen;
  \etc \\
  b)
  das höchste zulässige Ausmaß des umbaubaren Raumes der
  Bauwerke auf dem Bauplatz;
  }
  \notiz{Jetzt absorbiert durch \merkmalref{VolumenUndUmbaubarerRaum}.}
  \quelle{Aus
    \href{run:./files/200526_BRISE_RIM-IDM_Auszug_AI-BB.xlsx}{200626\_BRISE\_RIM-IDM\_Auszug\_AI-BB}
  }
}


% \subsubsection{UmbaubarerRaumBauplatzMax}
% \label{sec:umbaRBM}


% \paragraph{Erläuterung}
% \paragraph{Wert} Positive Zahl [$m^3$]
% \paragraph{Beispiele}
% \paragraph{Rechtsmaterie}


\merkmal{UmbaubarerRaumGebaeudeMax}{
  \erlaeuterung{Ausmaß der Nutzflächen der einzelnen Räume und das Gesamtausmaß der Nutzfläche.}
  \wert{Eine positive Zahl [$m^2$]}
  % \beispiel{}
  \rechtsmaterie{\href{https://www.ris.bka.gv.at/NormDokument.wxe?Abfrage=LrW&Gesetzesnummer=20000006&FassungVom=2020-11-05&Artikel=&Paragraf=64&Anlage=&Uebergangsrecht=}{WBO
      \S 64/1/c}}:\\
  \S 64. (1) Die Baupläne haben zu enthalten:\\
  \etc\\
  c)
bei Bauführungen oder Widmungsänderungen, durch die Räume neu geschaffen, aufgelassen, geändert oder umgewidmet werden, das Ausmaß der Nutzflächen der einzelnen Räume und das Gesamtausmaß der Nutzfläche der einzelnen Benützungseinheiten (Wohnungen, Betriebe u. ä.); bei Neu- und Zubauten überdies das Ausmaß des umbauten Raumes der betroffenen Gebäude oder Gebäudeteile;
  \notiz{Jetzt absorbiert durch \merkmalref{VolumenUndUmbaubarerRaum}.}
\quelle{Aus \href{run:./files/200526_BRISE_RIM-IDM_Auszug
    AI-BB.xlsx}{200626\_BRISE\_RIM-IDM\_Auszug\_AI-BB}
}
\frage{Zwei Charakterisierungen, aber nur ein Wert? Ist das Volumen
  oder Fläche? Wenn Fläche, sollten wir es
  \texttt{UmbaubareFlaecheGebaeudeMax} nennen? ER SCHAUT NOCHMAL}
}


\merkmal{UmbaubarerRaumGebaeudeteilMax}{
  \erlaeuterung{Ausmaß der Nutzflächen der einzelnen Räume und das Gesamtausmaß der einzelnen Benützungseinheiten.}
  \wert{Eine positive Zahl [$m^2$]}
  % \beispiel{}
  \rechtsmaterie{\href{https://www.ris.bka.gv.at/NormDokument.wxe?Abfrage=LrW&Gesetzesnummer=20000006&FassungVom=2020-11-05&Artikel=&Paragraf=64&Anlage=&Uebergangsrecht=}{WBO \S 64/1/c}}:\\
  \S 64. (1) Die Baupläne haben zu enthalten:\\
  \etc\\
  c)
bei Bauführungen oder Widmungsänderungen, durch die Räume neu geschaffen, aufgelassen, geändert oder umgewidmet werden, das Ausmaß der Nutzflächen der einzelnen Räume und das Gesamtausmaß der Nutzfläche der einzelnen Benützungseinheiten (Wohnungen, Betriebe u. ä.); bei Neu- und Zubauten überdies das Ausmaß des umbauten Raumes der betroffenen Gebäude oder Gebäudeteile;
  \notiz{Jetzt absorbiert durch \merkmalref{VolumenUndUmbaubarerRaum}.}
\quelle{Aus \href{run:./files/200526_BRISE_RIM-IDM_Auszug
    AI-BB.xlsx}{200626\_BRISE\_RIM-IDM\_Auszug\_AI-BB}
}
\frage{Zwei Charakterisierungen, aber nur ein Wert? Ist das Volumen
  oder Fläche? Wenn Fläche, sollten wir es
  \texttt{UmbaubareFlaecheGebaeudeteilMax} nennen? ER SCHAUT NOCHMAL}
}


% \merkmal{WidmungAusnuetzbarkeitVolumen}{
%   % \erlaeuterung{}
%   \wert{Eine positive Zahl [$m^3$]}
%   % \beispiel{}
%   \rechtsmaterie{\href{https://www.ris.bka.gv.at/NormDokument.wxe?Abfrage=LrW&Gesetzesnummer=20000006&FassungVom=2020-11-05&Artikel=&Paragraf=5&Anlage=&Uebergangsrecht=}{WBO \S 5/4/d}: \\
% (4) Über die Festsetzungen nach Abs. 2 und 3 hinaus können die
% Bebauungspläne zusätzlich enthalten:\\
% \etc\\
% d)
% Bestimmungen über die flächenmäßige beziehungsweise volumenbezogene Ausnützbarkeit der Bauplätze und der Baulose oder von Teilen davon; in Gebieten für geförderten Wohnbau Bestimmungen über den Anteil der Wohnnutzfläche der auf einem Bauplatz geschaffenen Wohnungen und Wohneinheiten in Heimen, die hinsichtlich der Grundkostenangemessenheit dem Wiener Wohnbauförderungs- und Wohnhaussanierungsgesetz – WWFSG 1989 entsprechen müssen;}
%   \notiz{Aus WienBV\_Bebauungsbestimmungen}
%   \quelle{Aus \href{run:./files/200929_BRISE_IDM-REM-LOI.xlsx}{200929\_BRISE\_IDM REM LOI}
%   }
% \frage{Erläuterung? Ist das relevant? NEIN}
% }

 
}





\end{document}



%%% Local Variables:
%%% mode: latex
%%% TeX-master: t
%%% End:
